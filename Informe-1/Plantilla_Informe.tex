\documentclass[11pt,a4paper]{article}

\usepackage[utf8]{inputenc}
\usepackage[spanish]{babel}
\usepackage{amsmath}
\usepackage{amsfonts}
\usepackage{amssymb}
\usepackage{makeidx}
\usepackage{graphicx}
\usepackage{lmodern}
\usepackage{kpfonts}
\usepackage{wrapfig}
\usepackage{caption}
\usepackage{subcaption}
\usepackage{booktabs}
\usepackage[nottoc,numbib]{tocbibind} %agrega la bibliografia al índice.
\usepackage[font={small,it}]{caption}
%\usepackage{fourier}
\usepackage[left=2cm,right=2cm,top=2cm,bottom=2cm,headheight=13.6pt]{geometry}
\usepackage{fancyhdr}
\pagestyle{fancy}


%Para los gráficos en general, con las tablas...¡Ja!, arreglate.
%\begin{figure}[h!]
%\centering
%\includegraphics[width=0.7\textwidth]{} %nombre de la imagen, incluirla en el mismo directorio que este archivo.
%\caption*{} %rótulo, el asterico elimina la numeración automática. 
%\label{fig:} % para luego referirse con \ref{fig:}
%\end{figure}


\begin{document}


%%%%%%%%%%%%%%%%%%%%%%%%%%%%%%%%%%%%%%%%%%%%%%%%%%%%%%%%%%%%%%%%%%%%%%%%%%%%%%%%%%%%%%%%%%%%%%%%%%%%%%%%%%%%%%%%%%%%%%%%%%%%%%%%%
% 	TÍTULO
%%%%%%%%%%%%%%%%%%%%%%%%%%%%%%%%%%%%%%%%%%%%%%%%%%%%%%%%%%%%%%%%%%%%%%%%%%%%%%%%%%%%%%%%%%%%%%%%%%%%%%%%%%%%%%%%%%%%%%%%%%%%%%%%%

%%%%%%%%%%%%%%%%%%%%%%%%%%%%%%%%%%%%%%%%%
% University Assignment Title Page 
% LaTeX Template
% Version 1.0 (27/12/12)
%
% This template has been downloaded from:
% http://www.LaTeXTemplates.com
%
% Original author:
% WikiBooks (http://en.wikibooks.org/wiki/LaTeX/Title_Creation)
%
% License:
% CC BY-NC-SA 3.0 (http://creativecommons.org/licenses/by-nc-sa/3.0/)
% 
% Instructions for using this template:
% This title page is capable of being compiled as is. This is not useful for 
% including it in another document. To do this, you have two options: 
%
% 1) Copy/paste everything between \begin{document} and \end{document} 
% starting at \begin{titlepage} and paste this into another LaTeX file where you 
% want your title page.
% OR
% 2) Remove everything outside the \begin{titlepage} and \end{titlepage} and 
% move this file to the same directory as the LaTeX file you wish to add it to. 
% Then add \input{./title_page_1.tex} to your LaTeX file where you want your
% title page.
%
%%%%%%%%%%%%%%%%%%%%%%%%%%%%%%%%%%%%%%%%%

%----------------------------------------------------------------------------------------
%	PACKAGES AND OTHER DOCUMENT CONFIGURATIONS
%----------------------------------------------------------------------------------------

%\documentclass[12pt]{article}
%\usepackage[utf8]{inputenc}
%\usepackage[spanish]{babel}
%\begin{document}

\begin{titlepage}

\newcommand{\HRule}{\rule{\linewidth}{0.5mm}} % Defines a new command for the horizontal lines, change thickness here

\center % Center everything on the page
 
%----------------------------------------------------------------------------------------
%	HEADING SECTIONS
%----------------------------------------------------------------------------------------

\textsc{\Huge Universidad de Buenos Aires}\\[0.5cm]
\textsc{\LARGE Facultad de Ciencias Exactas y Naturales}\\[0.5cm] % Name of your university/college
\textsc{\Large Departamento de Física}\\[0.25cm] % Major heading such as course name

\begin{figure}[h]
  \centering
  \includegraphics[scale=0.15]{Logo_DF}
  \\[0.5cm]
\end{figure}

\textsc{\large Laboratorio 3}\\[0.25cm] % Minor heading such as course title

%----------------------------------------------------------------------------------------
%	TITLE SECTION
%----------------------------------------------------------------------------------------

\HRule \\[0.4cm]
{ \huge \bfseries Estudio de resonancia y antiresonancia en circuitos RLC, y filtros en circuitos RC utilizando señales sinusoidales }\\[0.2cm] % Title of your document
\HRule \\[1cm]
 
%----------------------------------------------------------------------------------------
%	AUTHOR SECTION
%----------------------------------------------------------------------------------------

\begin{minipage}{0.4\textwidth}
\begin{center} \large
\emph{Autores:}\\
\textsc{Andreu}, Gonzalo\\ % Your name
\textsc{Malpartida}, Bryan\\ % Your name
\textsc{Pugliese}, Facundo\\ % Your name


\end{center}
\end{minipage}
~ \\[1.25cm]
%\begin{minipage}{0.4\textwidth}
%\begin{flushright} \large
%\emph{Supervisor:} \\
%Dr. James \textsc{Smith} % Supervisor's Name
%\end{flushright}
%\end{minipage}\\[4cm]

% If you don't want a supervisor, uncomment the two lines below and remove the section above
%\Large \emph{Author:}\\
%John \textsc{Smith}\\[3cm] % Your name

%----------------------------------------------------------------------------------------
%	DATE SECTION
%----------------------------------------------------------------------------------------

%\vspace{\fill}


{\large 24 de Febrero de 2016}\\[1.75cm] % Date, change the \today to a set date if you want to be precise

%----------------------------------------------------------------------------------------
%	SUMMARY SECTION: No más de 15 renglones, no te zarpes
%----------------------------------------------------------------------------------------

\begin{center}
\large{\textbf{Resumen}}

\small{El objetivo del siguiente trabajo fue comprobar empíricamente los fenómenos de resonancia y anti-resonancia presente en circuitos eléctricos RLC, así como también el estudio de filtros pasa-bajos y filtros pasa-alto en circuitos RC y filtros pasa-banda en un circuito compuesto por dos RC conectados en paralelo. Para ello se observó, utilizando un osciloscopio digital, el comportamiento de la transferencia y desfasaje entre la señal de salida y una señal de entrada sinusoidal entregada por un fuente de alimentación programable.} % ACA VA EL RESUMEN


\end{center}


%----------------------------------------------------------------------------------------
%	LOGO SECTION
%----------------------------------------------------------------------------------------

%\includegraphics{Logo}\\[1cm] % Include a department/university logo - this will require the graphicx package
 
%----------------------------------------------------------------------------------------

\vfill % Fill the rest of the page with whitespace

\end{titlepage}
%\end{document} %incluir en el mismo directorio que este archivo. Equivalente a un copiar-pegar, nada de andar diciendo \begin{document} en la portada. Dejar el nombre de Caratula a la caratula.

%%%%%%%%%%%%%%%%%%%%%%%%%%%%%%%%%%%%%%%%%%%%%%%%%%%%%%%%%%%%%%%%%%%%%%%%%%%%%%%%%%%%%%%%%%%%%%%%%%%%%%%%%%%%%%%%%%%%%%%%%%%%%%%%%
% 	ENCABEZADO Y PIE DE PÁGINA.
%%%%%%%%%%%%%%%%%%%%%%%%%%%%%%%%%%%%%%%%%%%%%%%%%%%%%%%%%%%%%%%%%%%%%%%%%%%%%%%%%%%%%%%%%%%%%%%%%%%%%%%%%%%%%%%%%%%%%%%%%%%%%%%%%

\lhead{}
\chead{}
\rhead{Laboratorio 3}
\lfoot{}
\cfoot{}
\rfoot{\thepage}
\renewcommand{\headrulewidth}{1pt}
\renewcommand{\footrulewidth}{1pt}


%%%%%%%%%%%%%%%%%%%%%%%%%%%%%%%%%%%%%%%%%%%%%%%%%%%%%%%%%%%%%%%%%%%%%%%%%%%%%%%%%%%%%%%%%%%%%%%%%%%%%%%%%%%%%%%%%%%%%%%%%%%%%%%
% Página en blanco. Cita, agradecimiento, dedicación, lo que sea pero que sea algo.
%%%%%%%%%%%%%%%%%%%%%%%%%%%%%%%%%%%%%%%%%%%%%%%%%%%%%%%%%%%%%%%%%%%%%%%%%%%%%%%%%%%%%%%%%%%%%%%%%%%%%%%%%%%%%%%%%%%%%%%%%%%%%%%


%%%%%%%%%%%%%%%%%%%%%%%%%%%%%%%%%%%%%%%%%%%%%%%%%%%%%%%%%%%%%%%%%%%%%%%%%%%%%%%%%%%%%%%%%%%%%%%%%%%%%%%%%%%%%%%%%%%%%%%%%%%%%%%%%
% 	ÍNDICE
%%%%%%%%%%%%%%%%%%%%%%%%%%%%%%%%%%%%%%%%%%%%%%%%%%%%%%%%%%%%%%%%%%%%%%%%%%%%%%%%%%%%%%%%%%%%%%%%%%%%%%%%%%%%%%%%%%%%%%%%%%%%%%%%%

%\tableofcontents %compilar dos o tres veces para verlo bien. ¡Todo un índice en unas cuantas letras!
%\newpage

%%%%%%%%%%%%%%%%%%%%%%%%%%%%%%%%%%%%%%%%%%%%%%%%%%%%%%%%%%%%%%%%%%%%%%%%%%%%%%%%%%%%%%%%%%%%%%%%%%%%%%%%%%%%%%%%%%%%%%%%%%%%%%%
% 1. RESUMEN
%%%%%%%%%%%%%%%%%%%%%%%%%%%%%%%%%%%%%%%%%%%%%%%%%%%%%%%%%%%%%%%%%%%%%%%%%%%%%%%%%%%%%%%%%%%%%%%%%%%%%%%%%%%%%%%%%%%%%%%%%%%%%%%

\section{Resumen}
\label{sec:resumen}



%%%%%%%%%%%%%%%%%%%%%%%%%%%%%%%%%%%%%%%%%%%%%%%%%%%%%%%%%%%%%%%%%%%%%%%%%%%%%%%%%%%%%%%%%%%%%%%%%%%%%%%%%%%%%%%%%%%%%%%%%%%%%%%
% 2. INTRODUCCIÓN: ecuaciones aquí, luego se las cita.
%%%%%%%%%%%%%%%%%%%%%%%%%%%%%%%%%%%%%%%%%%%%%%%%%%%%%%%%%%%%%%%%%%%%%%%%%%%%%%%%%%%%%%%%%%%%%%%%%%%%%%%%%%%%%%%%%%%%%%%%%%%%%%%

\section{Introducción}\label{sec:intro}

El siguiente informe detalla los experimentos llevados a cabo para comprobar empíricamente dos de las herramientas fundamentales a la hora de resolver un circuito eléctrico: la ley de Ohm y el teorema de Thevenin. Ademas de verificar (también empíricamente) la ley de Joule, que determina la potencia disipada por una resistencia, y la eficacia de la misma.

La ley de Ohm dice que si una fuente $\varepsilon$ y una resistencia $R$ están conectados en serie, la diferencia de potencial entre las terminales $A$ y $B$ (en los extremos de este circuito respectivamente) es $V_{AB} = \varepsilon –iR$, donde $i$ es la corriente eléctrica generada por la fuente que pasa por la resistencia. En particular, si el circuito es cerrado se tiene $V_{AB}=0$ obteniendo entonces:

\begin{equation}\label{Ohm}
\ i= \frac{\varepsilon}{R}
\end{equation}

Por otro lado, el teorema de Thevenin enuncia que si se desea conocer el comportamiento de un circuito en una sola rama el resto del circuito puede reemplazarse por una rama que consiste de una resistencia equivalente $R_{th}$ y una fuente equivalente $\varepsilon_{th}$.

La resistencia $R_{th}$ se calcula cortando el circuito y resolviendo el sistema de resistencias que se obtiene. Mientras que la fuente equivalente $\varepsilon_{th}$ se obtiene calculando la diferencia de potencial entra $A$ y $B$ a circuito abierto utilizando, por ejemplo, el método de ramas.

Para el sistema utilizado (detallado mas adelante) se tiene:

\begin{equation}\label{Rth}
\ R_{th}= \frac{rR}{2r+R}
\end{equation}

\begin{equation}\label{Eth}
\ \varepsilon_{th}= \frac{2\varepsilon r}{2r+R}
\end{equation}

Por ultimo, para poder calcular la potencia disipada por una resistencia $R$, la ley de Joule determina la siguiente ecuación:

\begin{equation}\label{Joule}
\ P = i^2 R
\end{equation}

Donde la $i$ es la corriente que pasa por la resistencia. 

De esta forma, si se coloca una resistencia de carga $R_{v}$ que pueda variar en paralelo con una resistencia de un circuito, y calculando su respectivo equivalente de Thevenin, se puede obtener la potencia disipada por $R_{v}$ como:

\begin{equation}\label{Pot}
\ P_{R_{v}}=\frac{\varepsilon_{th}^2 R_{v}}{(R_{v}+R_{th})^2}
\end{equation}

Una curva acampanada cuyo máximo debería encontrarse en $R_{max} = R_{th}$.

Entonces, se define la eficacia como el cociente entre la potencia disipada por una resistencia $R$ y la potencia entregada por una fuente $\varepsilon$:

\begin{equation}\label{eficacia}
\ E=\frac{P_{R}}{P_{\varepsilon}}
\end{equation}

Por lo tanto, si se quiere conocer la eficacia de una resistencia $R$ se puede calcular su equivalente de Thevenin para obtener la relación:

\begin{equation}\label{efi}
\ E= \frac{\varepsilon_{th}}{\varepsilon}\frac{1}{1+\frac{R_{th}}{R}}
\end{equation}

%%%%%%%%%%%%%%%%%%%%%%%%%%%%%%%%%%%%%%%%%%%%%%%%%%%%%%%%%%%%%%%%%%%%%%%%%%%%%%%%%%%%%%%%%%%%%%%%%%%%%%%%%%%%%%%%%%%%%%%%%%%%%%%
% 3. DISPOSITIVO EXPERIMENTAL: armado del modelo, como se midio, consideraciones a la hora de medir.
%%%%%%%%%%%%%%%%%%%%%%%%%%%%%%%%%%%%%%%%%%%%%%%%%%%%%%%%%%%%%%%%%%%%%%%%%%%%%%%%%%%%%%%%%%%%%%%%%%%%%%%%%%%%%%%%%%%%%%%%%%%%%%%

\section{Desarrollo experimental}

\subsection{Ley de Ohm}

Como se ha dicho, una de las leyes basicas de los circuitos en general es la Ley de Ohm (\textbf{\ref{Ohm}}). Es por esto que la primer parte del trabajo se desarrolló en torno a su comprobación empírica para circuitos de corriente continua. El montaje del sistema fue simple, consistiendo en una Fuente de Alimentación Programable de 3 canales, un multimetro digital y una resistencia variable por decadas. La Fuente de Alimentación actuó como una fuente de voltaje constante, que se conectó en serie con el multimetro y la resistencia variable. Esto permitió generar un circuito de una única malla que consiste en una resistencia y una fuente de voltaje constante (Ver \textbf{Fig \ref{fig:circ_simp}}) 


\begin{figure}[h]
  \centering
  \includegraphics[scale= 0.6]{Circuito_simple}
  \caption{Circuito de malla única utilizado para la experiencia}
  \label{fig:circ_simp}
\end{figure}

La resistencia del multimetro utilizado en su función de amperímetro era $R_{A} = (11,8 \pm 0,4) \Omega$, la cual debió sumarse a la resistencia variable $R_v$ (pues están en serie) a la hora de calcular la resistencia total del circuito. La resistencia interna de la Fuente se asumió despreciable. Para empezar, se fijó la resistencia variable $R_v = (500 \pm 5)\Omega$ y se fue variando el voltaje de entrada $\varepsilon$, registrándose la corriente $I$ resultante para cada valor de $\varepsilon$. Posteriormente, utilizando el mismo circuito se planteó analizar la relación entre $I$ y $R$ fijando el voltaje de la fuente a $\varepsilon = (15,0 \pm 0,1)V$ y variando la resistencia total $R = R_v + R_A$ a través de $R_v$. 


\subsection{Equivalente de Thevenin y Ley de Joule}

Terminado lo anterior, se propuso montar otro circuito con dos mallas con el primer objetivo de medir sus corrientes de rama y compararlas con el resultado teórico esperado. Se utilizaron dos resistencias $R = (1000,0\pm0,5)\Omega$ y una resistencia $r = (100,00\pm0,05)\Omega$, mientras que la fuente $\varepsilon = (5,00\pm0,05)V$ fue generada nuevamente por la Fuente de Alimentación Programable, unidos por cables como se ve en el circuito de la \textbf{Fig \ref{fig:circ_mallas}}.

\begin{figure}[h]
  \centering
  \includegraphics[scale=0.55]{Mallas_sin_carga}
  \caption{Circuito de dos mallas utilizado para la experiencia, los $A_i$ representan los lugares donde se ubicó el multimetro}
  \label{fig:circ_mallas}
\end{figure}

Inicialmente, se utilizaron las leyes de Kirchoff y se resolvió el circuito mediante el método de ramas para obtener el valor de las corrientes $i_1$, $i_2$ e $i_3$ en base a los parámetros. 
 
Luego se ubicó el multimetro en las posiciones $A_1$, $A_2$ y $A_3$ (Ver \textbf{Fig \ref{fig:circ_mallas}}) para medir las corrientes $i_1$, $i_2$ e $i_3$ respectivamente.considerando despreciable la resistencia interna del multímetro.
 
 
%%En la \textbf{Tabla 1} puede verse que los valores obtenidos con ambos metodos coinciden dentro del error. 

%\begin{center}
%\begin{tabular}{||c|c|c||}
%\hline
%& \multicolumn{2}{c||}{\textbf{Valor}} \\ \hline
%\textbf{Corriente} & \textbf{Teórico (mA)} & \textbf{Medido (mA)} \\ \hline 
%$i_1$ & $8,33\pm0,09$ & $8,3\pm0,3$ \\ \hline 
%$i_2$ & $4,17\pm0,05$ & $4,2\pm0,2$ \\ \hline 
%$i_3$ & $4,17\pm0,05$ & $4,1\pm0,2$ \\ \hline 
%\end{tabular}\\

%\textit{Tabla 1: Valores obtenidos de las corrientes mediante cálculo (teórico) y medición directa}
%\end{center}

Posteriormente, se buscó calcular el equivalente de Thevenin para obtener la resistencia y la fuente equivalente vista desde las terminales A y B, entre las cuales se conectará una resistencia de carga $R_q$ (ver \textbf{Fig \ref{fig:circ_mallas_carga}}). 

\begin{figure}[h]
  \centering
  \includegraphics[scale=0.55]{Mallas_con_carga}
  \caption{Circuito de dos mallas con la resistencia de carga acoplada}
  \label{fig:circ_mallas_carga}
\end{figure}

%Primero, se calculó el equivalente de Thevenin a través de los parámetros $R$, $r$ del circuito y $\varepsilon$ y se obtuvieron los parametros equivalentes $\varepsilon_{th} = (0,83\pm0,08)V$ y $R_{th} = (83,3 \pm 0,7) \Omega$. Cabe aclarar que en este cálculo no intervinieron las corrientes previamente calculadas, si no que se utilizó directamente su dependencia con los parámetros del circuito. Alternativamente, se conectó 

Conectando el multimetro en paralelo en las terminales A y B se midió la diferencia de potencial $\Delta V_{AB}$ y la resistencia $R_{AB}$  entre ambas terminales.

%$\Delta\varepsilon = (0,84 \pm 0,06)V$ y la resistencia $R_{AB} = (83,4\pm0,6)\Omega$ entre ambas terminales. 

Una vez hecho esto, fue posible conectar una resistencia de carga $R_v$ representada por una resistencia variable por decadas en serie con el multimetro en modo amperimetro. El objetivo fue medir la corriente $I$ para distintos valores de $R_v$ y en base a ambas magnitudes calcular la potencia $P$ disipada por $R_q$ a través de (\textbf{\ref{Pot}}) buscando relevar la curva acampanada $P(R_v)$.
%cuyo máximo debería encontrarse en $R_{max} = R_{th}$ $= (83,3\pm0,7)\Omega$ segun (\textbf{POT MAX})

%Utilizamos el valor $R_{th}$ por convención dado que es fisicamente indistinguible (debido al error) de $R_{AB}$. Considerando que las resistencias de carga utilizadas rondaron $R_{max} = (83,3\pm0,7)\Omega$, la resistencia interna del multimetro en modo amperimetro $R_{A} = (11,8 \pm 0,4) \Omega$ no resulta despreciable, por lo que la resistencia de carga resulta $R_q = R_v + R_A$ donde $R_v$ es nuevamente la resistencia variable por decadas. 

%%%%%%%%%%%%%%%%%%%%%%%%%%%%%%%%%%%%%%%%%%%%%%%%%%%%%%%%%%%%%%%%%%%%%%%%%%%%%%%%%%%%%%%%%%%%%%%%%%%%%%%%%%%%%%%%%%%%%%%%%%%%%%%%
% 4.DISCUSIÓN Y RESULTADOS: todo lo que se obtuvo y explicación. Graficos, tablas.
%%%%%%%%%%%%%%%%%%%%%%%%%%%%%%%%%%%%%%%%%%%%%%%%%%%%%%%%%%%%%%%%%%%%%%%%%%%%%%%%%%%%%%%%%%%%%%%%%%%%%%%%%%%%%%%%%%%%%%%%%%%%%%%%

\section{Resultados}
\label{sec:discusion}

\subsection{Ley de Ohm}

Los resultados de la primer medición con la resistencia $R$ constante pueden verse en la \textbf{Figura \ref{fig:Ohm_lin}}, cuyo ajuste lineal arroja una pendiente $m_1 = (1,979.10^{-3} \pm 5.10^{-6})\Omega^{-1}$ y una ordenada $b_1 = (-0,23\pm 0,03)mA$ muy cercana a cero con un $R-square = 0,99995$, que asegura la bondad del ajuste. Siguiendo (\textbf{\ref{Ohm}}), se esperaría que la pendiente $m_1 = (1,979.10^{-3} \pm 5.10^{-6})\Omega^{-1}$ fuera igual a $\frac{1}{R}$ con $R = (512 \pm 5)\Omega$. Efectivamente, resulta $\frac{1}{m_1} = (505.3 \pm 1,3)\Omega$, el cual se acerca mucho a $R = (512 \pm 5)\Omega$.

\begin{figure}[h]
  \centering
  \includegraphics[scale=0.4]{Corriente_vs_Voltaje}
  \caption{Relacion entre Voltaje ($\varepsilon$) y Corriente (I)}
  \label{fig:Ohm_lin}
\end{figure}


%\textbf{OHM HIP}
%\begin{figure}[h]
  %\centering
  %\includegraphics[scale=0.15]{}
  %\\[1.0cm]
  %\label{fig:Ohm_hip}
  %\caption{Relacion entre Resistencia (R) y Corriente (I)}
%\end{figure}


\subsection{Equivalente de Thevenin y Ley de Joule}

 
De las mediciones y calculos hechos de las corrientes $A_1$, $A_2$ y $A_3$ puede verse, en la \textbf{Tabla 1}, que los valores obtenidos con ambos metodos coinciden dentro del error. 

\begin{center}
\begin{tabular}{||c|c|c||}
\hline
& \multicolumn{2}{c||}{\textbf{Valor}} \\ \hline
\textbf{Corriente} & \textbf{Teórico (mA)} & \textbf{Medido (mA)} \\ \hline 
$i_1$ & $8,33\pm0,09$ & $8,3\pm0,3$ \\ \hline 
$i_2$ & $4,17\pm0,05$ & $4,2\pm0,2$ \\ \hline 
$i_3$ & $4,17\pm0,05$ & $4,1\pm0,2$ \\ \hline 
\end{tabular}\\

\textit{Tabla 1: Valores obtenidos de las corrientes mediante cálculo (teórico) y medición directa}
\end{center}

Luego, se calculó el equivalente de Thevenin a través de los parámetros $R$, $r$ del circuito y $\varepsilon$ y se obtuvieron los parametros equivalentes $\varepsilon_{th} = (0,83\pm0,08)V$ y $R_{th} = (83,3 \pm 0,7) \Omega$. Cabe aclarar que en este cálculo no intervinieron las corrientes previamente calculadas, si no que se utilizó directamente su dependencia con los parámetros del circuito.

De los datos obtenidos se obtiene el gráfico de \textbf{Figura \ref{fig:Pot_Res}} donde se puede observar que el comportamiento de la potencia entregada por el circuito esta dentro de los parametros esperados. Luego, realizando un ajuste, se obtiene una resistencia $R = (78.5 \pm 0.8)\Omega$ y un voltaje $\epsilon = (0,74\pm 0,01)V$ ambos muy cercanos a los valores de Thevenin calculados anteriormente.

\begin{figure}[h]
  \centering
  \includegraphics[scale=0.45]{Potencia_vs_Resistencia_2}
  \label{fig:Pot_Res}
  \caption{Relacion entre la potencia disipada y el valor de la resistencia de carga}

\end{figure}

Finalmente, se calculo la eficacia con la cual se traspasaba la potencia de nuestro circuito sobre la carga. En el grafico de la \textbf{Figura \ref{fig:Efi_Res}} se ve que hay una tendencia creciente pero que en los ultimos puntos se empieza a dispersar. Esto puede deberse a queal trabajar con resistencias tan altas, las corrientes medidas quedan muy cerca del umbral de resolucion del multimetro, por lo cual ya las fluctuaciones propias del instrumento provocan incertezas muy altas. 

\begin{figure}[h]
  \centering
  \includegraphics[scale=0.45]{Eficiencia_vs_Resistencia}
  \label{fig:Efi_Res}
  \caption{Relación entre la eficacia de la entrega de potencia y el valor de la resistencia de carga}

\end{figure}

Desestimando los puntos en cuyos calculos contenian corrientes con incertezas comparables con sus mediciones, el ajuste realizado arroja como uno de sus parametros, la resistencia equivalente del circuito $R = (77.8 \pm 0.8)\Omega$ que es consistente con los resultados y mediciones anteriores.
\textbf{CAMPANA}
%\begin{figure}[h]
 % \centering
  %\includegraphics[scale=0.15]{}
  %\\[1.0cm]
  %\label{fig:campana}
  %\caption{Relacion entre la potencia disipada y el valor de la resistencia de carga}
%\end{figure}

%%%%%%%%%%%%%%%%%%%%%%%%%%%%%%%%%%%%%%%%%%%%%%%%%%%%%%%%%%%%%%%%%%%%%%%%%%%%%%%%%%%%%%%%%%%%%%%%%%%%%%%%%%%%%%%%%%%%%%%%%%%%%%%%
%	CONCLUSIONES
%%%%%%%%%%%%%%%%%%%%%%%%%%%%%%%%%%%%%%%%%%%%%%%%%%%%%%%%%%%%%%%%%%%%%%%%%%%%%%%%%%%%%%%%%%%%%%%%%%%%%%%%%%%%%%%%%%%%%%%%%%%%%%%%

\section{Conclusiones}
\label{sec:conclusiones}




%%%%%%%%%%%%%%%%%%%%%%%%%%%%%%%%%%%%%%%%%%%%%%%%%%%%%%%%%%%%%%%%%%%%%%%%%%%%%%%%%%%%%%%%%%%%%%%%%%%%%%%%%%%%%%%%%%%%%%%%%%%%%%%%%
%	APÉNDICE: esas cosas extras que simplemente no tuvieron lo suficiente como para ganarse una sección propia.
%%%%%%%%%%%%%%%%%%%%%%%%%%%%%%%%%%%%%%%%%%%%%%%%%%%%%%%%%%%%%%%%%%%%%%%%%%%%%%%%%%%%%%%%%%%%%%%%%%%%%%%%%%%%%%%%%%%%%%%%%%%%%%%%%



%%%%%%%%%%%%%%%%%%%%%%%%%%%%%%%%%%%%%%%%%%%%%%%%%%%%%%%%%%%%%%%%%%%%%%%%%%%%%%%%%%%%%%%%%%%%%%%%%%%%%%%%%%%%%%%%%%%%%%%%%%%%%%%%%
%	REFERENCIAS: libros, libros, libros.
%%%%%%%%%%%%%%%%%%%%%%%%%%%%%%%%%%%%%%%%%%%%%%%%%%%%%%%%%%%%%%%%%%%%%%%%%%%%%%%%%%%%%%%%%%%%%%%%%%%%%%%%%%%%%%%%%%%%%%%%%%%%%%%%%

%Ejemplo:
\begin{thebibliography}{1}
 \bibitem{Berkeley} Frank S. Crawford, \textit{Berkeley physics course 3: Ondas}, 1994, Editorial Reverte S.A.
\end{thebibliography}
%Para citar: blablabla \cite{Baird}
 
\end{document}





