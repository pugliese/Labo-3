\documentclass[11pt,a4paper]{article}

\usepackage[utf8]{inputenc}
\usepackage[export]{adjustbox}
\usepackage[spanish]{babel}
\usepackage{float}
\usepackage{amsmath}
\usepackage{amsfonts}
\usepackage{amssymb}
\usepackage{makeidx}
\usepackage{graphicx}
\usepackage{lmodern}
\usepackage{kpfonts}
\usepackage{wrapfig}
\usepackage{caption}
\usepackage{subcaption}
\usepackage{booktabs}
\usepackage[nottoc,numbib]{tocbibind} %agrega la bibliografia al índice.
\usepackage[font={small,it}]{caption}
%\usepackage{fourier}
\usepackage[left=2cm,right=2cm,top=2cm,bottom=2cm,headheight=13.6pt]{geometry}
\usepackage{fancyhdr}
\usepackage{multirow}
\pagestyle{fancy}
\usepackage{mathrsfs}
\usepackage{graphicx}

%Para los gráficos en general, con las tablas...¡Ja!, arreglate.
%\begin{figure}[h!]
%\centering
%\includegraphics[width=0.7\textwidth]{} %nombre de la imagen, incluirla en el mismo directorio que este archivo.
%\caption*{} %rótulo, el asterico elimina la numeración automática. 
%\label{fig:} % para luego referirse con \ref{fig:}
%\end{figure}


\begin{document}


%%%%%%%%%%%%%%%%%%%%%%%%%%%%%%%%%%%%%%%%%%%%%%%%%%%%%%%%%%%%%%%%%%%%%%%%%%%%%%%%%%%%%%%%%%%%%%%%%%%%%%%%%%%%%%%%%%%%%%%%%%%%%%%%%
% 	TÍTULO
%%%%%%%%%%%%%%%%%%%%%%%%%%%%%%%%%%%%%%%%%%%%%%%%%%%%%%%%%%%%%%%%%%%%%%%%%%%%%%%%%%%%%%%%%%%%%%%%%%%%%%%%%%%%%%%%%%%%%%%%%%%%%%%%%

%%%%%%%%%%%%%%%%%%%%%%%%%%%%%%%%%%%%%%%%%
% University Assignment Title Page 
% LaTeX Template
% Version 1.0 (27/12/12)
%
% This template has been downloaded from:
% http://www.LaTeXTemplates.com
%
% Original author:
% WikiBooks (http://en.wikibooks.org/wiki/LaTeX/Title_Creation)
%
% License:
% CC BY-NC-SA 3.0 (http://creativecommons.org/licenses/by-nc-sa/3.0/)
% 
% Instructions for using this template:
% This title page is capable of being compiled as is. This is not useful for 
% including it in another document. To do this, you have two options: 
%
% 1) Copy/paste everything between \begin{document} and \end{document} 
% starting at \begin{titlepage} and paste this into another LaTeX file where you 
% want your title page.
% OR
% 2) Remove everything outside the \begin{titlepage} and \end{titlepage} and 
% move this file to the same directory as the LaTeX file you wish to add it to. 
% Then add \input{./title_page_1.tex} to your LaTeX file where you want your
% title page.
%
%%%%%%%%%%%%%%%%%%%%%%%%%%%%%%%%%%%%%%%%%

%----------------------------------------------------------------------------------------
%	PACKAGES AND OTHER DOCUMENT CONFIGURATIONS
%----------------------------------------------------------------------------------------

%\documentclass[12pt]{article}
%\usepackage[utf8]{inputenc}
%\usepackage[spanish]{babel}
%\begin{document}

\begin{titlepage}

\newcommand{\HRule}{\rule{\linewidth}{0.5mm}} % Defines a new command for the horizontal lines, change thickness here

\center % Center everything on the page
 
%----------------------------------------------------------------------------------------
%	HEADING SECTIONS
%----------------------------------------------------------------------------------------

\textsc{\Huge Universidad de Buenos Aires}\\[0.5cm]
\textsc{\LARGE Facultad de Ciencias Exactas y Naturales}\\[0.5cm] % Name of your university/college
\textsc{\Large Departamento de Física}\\[0.25cm] % Major heading such as course name

\begin{figure}[h]
  \centering
  \includegraphics[scale=0.15]{Logo_DF}
  \\[0.5cm]
\end{figure}

\textsc{\large Laboratorio 3}\\[0.25cm] % Minor heading such as course title

%----------------------------------------------------------------------------------------
%	TITLE SECTION
%----------------------------------------------------------------------------------------

\HRule \\[0.4cm]
{ \huge \bfseries Estudio de resonancia y antiresonancia en circuitos RLC, y filtros en circuitos RC utilizando señales sinusoidales }\\[0.2cm] % Title of your document
\HRule \\[1cm]
 
%----------------------------------------------------------------------------------------
%	AUTHOR SECTION
%----------------------------------------------------------------------------------------

\begin{minipage}{0.4\textwidth}
\begin{center} \large
\emph{Autores:}\\
\textsc{Andreu}, Gonzalo\\ % Your name
\textsc{Malpartida}, Bryan\\ % Your name
\textsc{Pugliese}, Facundo\\ % Your name


\end{center}
\end{minipage}
~ \\[1.25cm]
%\begin{minipage}{0.4\textwidth}
%\begin{flushright} \large
%\emph{Supervisor:} \\
%Dr. James \textsc{Smith} % Supervisor's Name
%\end{flushright}
%\end{minipage}\\[4cm]

% If you don't want a supervisor, uncomment the two lines below and remove the section above
%\Large \emph{Author:}\\
%John \textsc{Smith}\\[3cm] % Your name

%----------------------------------------------------------------------------------------
%	DATE SECTION
%----------------------------------------------------------------------------------------

%\vspace{\fill}


{\large 24 de Febrero de 2016}\\[1.75cm] % Date, change the \today to a set date if you want to be precise

%----------------------------------------------------------------------------------------
%	SUMMARY SECTION: No más de 15 renglones, no te zarpes
%----------------------------------------------------------------------------------------

\begin{center}
\large{\textbf{Resumen}}

\small{El objetivo del siguiente trabajo fue comprobar empíricamente los fenómenos de resonancia y anti-resonancia presente en circuitos eléctricos RLC, así como también el estudio de filtros pasa-bajos y filtros pasa-alto en circuitos RC y filtros pasa-banda en un circuito compuesto por dos RC conectados en paralelo. Para ello se observó, utilizando un osciloscopio digital, el comportamiento de la transferencia y desfasaje entre la señal de salida y una señal de entrada sinusoidal entregada por un fuente de alimentación programable.} % ACA VA EL RESUMEN


\end{center}


%----------------------------------------------------------------------------------------
%	LOGO SECTION
%----------------------------------------------------------------------------------------

%\includegraphics{Logo}\\[1cm] % Include a department/university logo - this will require the graphicx package
 
%----------------------------------------------------------------------------------------

\vfill % Fill the rest of the page with whitespace

\end{titlepage}
%\end{document} %incluir en el mismo directorio que este archivo. Equivalente a un copiar-pegar, nada de andar diciendo \begin{document} en la portada. Dejar el nombre de Caratula a la caratula.

%%%%%%%%%%%%%%%%%%%%%%%%%%%%%%%%%%%%%%%%%%%%%%%%%%%%%%%%%%%%%%%%%%%%%%%%%%%%%%%%%%%%%%%%%%%%%%%%%%%%%%%%%%%%%%%%%%%%%%%%%%%%%%%%%
% 	ENCABEZADO Y PIE DE PÁGINA.
%%%%%%%%%%%%%%%%%%%%%%%%%%%%%%%%%%%%%%%%%%%%%%%%%%%%%%%%%%%%%%%%%%%%%%%%%%%%%%%%%%%%%%%%%%%%%%%%%%%%%%%%%%%%%%%%%%%%%%%%%%%%%%%%%

\lhead{}
\chead{}
\rhead{Laboratorio 3}
\lfoot{}
\cfoot{}
\rfoot{\thepage}
\renewcommand{\headrulewidth}{1pt}
\renewcommand{\footrulewidth}{1pt}
\newcommand\Sum{\displaystyle\sum}

%%%%%%%%%%%%%%%%%%%%%%%%%%%%%%%%%%%%%%%%%%%%%%%%%%%%%%%%%%%%%%%%%%%%%%%%%%%%%%%%%%%%%%%%%%%%%%%%%%%%%%%%%%%%%%%%%%%%%%%%%%%%%%%
% Página en blanco. Cita, agradecimiento, dedicación, lo que sea pero que sea algo.
%%%%%%%%%%%%%%%%%%%%%%%%%%%%%%%%%%%%%%%%%%%%%%%%%%%%%%%%%%%%%%%%%%%%%%%%%%%%%%%%%%%%%%%%%%%%%%%%%%%%%%%%%%%%%%%%%%%%%%%%%%%%%%%


%%%%%%%%%%%%%%%%%%%%%%%%%%%%%%%%%%%%%%%%%%%%%%%%%%%%%%%%%%%%%%%%%%%%%%%%%%%%%%%%%%%%%%%%%%%%%%%%%%%%%%%%%%%%%%%%%%%%%%%%%%%%%%%%%
% 	ÍNDICE
%%%%%%%%%%%%%%%%%%%%%%%%%%%%%%%%%%%%%%%%%%%%%%%%%%%%%%%%%%%%%%%%%%%%%%%%%%%%%%%%%%%%%%%%%%%%%%%%%%%%%%%%%%%%%%%%%%%%%%%%%%%%%%%%%

%\tableofcontents %compilar dos o tres veces para verlo bien. ¡Todo un índice en unas cuantas letras!
%\newpage

%%%%%%%%%%%%%%%%%%%%%%%%%%%%%%%%%%%%%%%%%%%%%%%%%%%%%%%%%%%%%%%%%%%%%%%%%%%%%%%%%%%%%%%%%%%%%%%%%%%%%%%%%%%%%%%%%%%%%%%%%%%%%%%
% 1. RESUMEN
%%%%%%%%%%%%%%%%%%%%%%%%%%%%%%%%%%%%%%%%%%%%%%%%%%%%%%%%%%%%%%%%%%%%%%%%%%%%%%%%%%%%%%%%%%%%%%%%%%%%%%%%%%%%%%%%%%%%%%%%%%%%%%%

%\section{Resumen}
%\label{sec:resumen}



%%%%%%%%%%%%%%%%%%%%%%%%%%%%%%%%%%%%%%%%%%%%%%%%%%%%%%%%%%%%%%%%%%%%%%%%%%%%%%%%%%%%%%%%%%%%%%%%%%%%%%%%%%%%%%%%%%%%%%%%%%%%%%%
% 2. INTRODUCCIÓN: ecuaciones aquí, luego se las cita.
%%%%%%%%%%%%%%%%%%%%%%%%%%%%%%%%%%%%%%%%%%%%%%%%%%%%%%%%%%%%%%%%%%%%%%%%%%%%%%%%%%%%%%%%%%%%%%%%%%%%%%%%%%%%%%%%%%%%%%%%%%%%%%%

\section{Introducción}\label{sec:intro}
Para llevar a cabo el objetivo de caracterizar los circuitos RC, RL y RCL es necesario comprender como responden a distintos tipos de combinaciones(capacitor $C$, resistencia $R$ e inductancia $L$) y fuentes de voltaje. Recordando que $\frac{dq(t)}{dt} = I(t)$ donde $q(t)$ es la carga en el capacitor $C$, las leyes que dictan la caida de potencial en cada elemento son:

\begin{equation}{\label{leyes}}
\begin{split}
\Delta V_C &= \frac{q}{C}\\
\Delta V_R &= IR\\
\Delta V_L &= \frac{dI}{dt}
\end{split}
\end{equation}

Considerando un circuito cerrado de una unica malla compuesto por una fuente de voltaje constante $V_{0}$, una resistencia $R$ y un capacitor $C$, las leyes de Kirchoff dictan que la ecuación que rige la evolución del sistema es $V_{0} = \frac{q(t)}{C}+IR$ \cite{Trelles}. Entonces, tomando como condición inicial $q(0)=0$ la solución de la ecuación es:
   
\begin{equation}{\label{RC_carga}}
q(t)=V_{0}C(1-e^\frac{-t}{RC})
\end{equation}

Y al ser $q(t)$ una exponencial creciente se conoce a este fenómeno como la \textit{carga} del capacitor.

Ahora si, utilizando una \textit{llave ideal}, se corta la fuente $V$ a tiempo $t_{0}$ la nueva solución del sistema es de la forma:

\begin{equation}{\label{RC_descarga}}
q(t-t_{0})=V_{0}C(1-e^\frac{-t_{0}}{RC})e^\frac{-t}{RC}
\end{equation}

Entonces se define a $\tau_{RC}=RC$ \textit{tiempo característico} del circuito, dado que para $t>\tau_{RC}=RC$ resulta $e^\frac{-t}{RC}<e^{-1}<0,40$ por lo que el termino exponencial se vuelve pequeño y virtualmente despreciable. Por lo tanto, si $\tau_{RC}<t_{0}$ se puede observar el efecto de \textit{descarga} del capacitor pues $q(t-t_{0})$ es una exponencial decreciente. Para la corriente $I(t) = \frac{dq(t)}{dt}$, las ecuaciones que describen su comportamiento en cada caso son:

\begin{equation}{\label{RC_carga_I}}
I(t)=\frac{V_{0}}{R}e^\frac{-t}{RC}
\end{equation}

\begin{equation}{\label{RC_descarga_I}}
I(t-t_{0}) = \frac{V_{0}}{R}(e^\frac{t_{0}}{RC}-1)e^\frac{-t}{RC}
\end{equation}

Por otro lado, si en el circuito anterior se reemplaza al capacitor $C$ por una inductancia $L$, la noción de carga desaparece y queda la de corriente $I(t) = \frac{dq(t)}{dt}$ en una ecuación diferencial de la forma $V_{0} = L.\frac{dI(t)}{dt}+R.I(t)$ \cite{Trelles}. La solución para el caso de $V= V_0$ es muy similar a la anterior:

\begin{equation}{\label{RL_carga_I}}
I(t)=\frac{V_0}{R}(1-e^\frac{-Rt}{L})
\end{equation}

\begin{equation}{\label{RL_carga}}
\dfrac{dI(t)}{dt}=\frac{V_{0}}{L}e^\frac{-Rt}{L}
\end{equation}

En este caso, el  \textit{tiempo característico} del circuito es $\tau_{RL}=\frac{L}{R}$ (y cumple las mismas propiedades que para el caso del RC) y puede verse que la corriente $I$ \textit{aumenta} hasta alcanzar su valor ideal (el valor de la corriente en un circuito sin inductancia).

Cuando se corta la fuente $V$ a tiempo $t_{0}$ se tiene el caso en que la corriente $I$ \textit{disminuye} hasta volverse nula. En esta situación, la solución del sistema es:

\begin{equation}{\label{RL_descarga_I}}
I(t-t_0)= \frac{V_{0}}{R}(1-e^\frac{-Rt_0}{L})e^\frac{-Rt}{L}
\end{equation}

\begin{equation}{\label{RL_descarga}}
\dfrac{dI(t-t_{0})}{dt}=-\frac{V_{0}}{L}(1-e^\frac{-Rt_{0}}{L})e^\frac{-Rt}{L}
\end{equation}


Un circuito más complicado puede generarse combinando los dos anteriores; una fuente de voltaje constante $V_0$, una resistencia $R$, un capacitor $C$ y una inductancia $L$. En este caso, la ecuación diferencial que rige la dependencia temporal de la carga $q(t)$ resulta \cite{Trelles}:

\begin{equation}{\label{ec_RCL}}
\frac{d^2q(t)}{dt^2}L+\frac{dq(t)}{dt}R+\frac{q(t)}{C} = V_0
\end{equation}

cuyo polinomio característico es $\lambda^2.L+\lambda.R+\frac{1}{C} = 0$ que tiene como raíces $\lambda_{1,2} = -\frac{R}{2L} \pm \sqrt{(\frac{R}{2L})^2-\frac{1}{LC}}$ donde $\omega_o^2 = \frac{1}{LC}>0$ se denomina \textit{frecuencia natural} del circuito. Esta frecuencia $\omega_o^2 = \frac{1}{LC}$ es la frecuencia de oscilinación del circuito en el caso en que no existe componente disipativo ($R=0$). Dependiendo del valor del discriminante $\omega^2 = \beta^2-\omega_o^2$ donde $\beta = \frac{R}{2L}>0$ las soluciones del polinomio característico serán reales, real doble o complejas no reales y la función $q(t)$ tendrá un comportamiento sobreamortiguado, subamortiguado o amortiguado, respectivamente. 

Para el caso $\beta < \omega_o$ resulta $\omega^2 < 0$ y por lo tanto las raíces del polinomio característico son complejas y distintas. Esto conlleva a una solución de la forma:

\begin{equation}{\label{amort}}
q(t) = a.e^{-\beta t}.cos(|\omega|(t+\phi)) + V_0C
\end{equation}

donde $a$ y $\phi$ se obtienen a través de las condiciones iniciales del sistema y $|\omega| = \sqrt{\omega_o^2-\beta^2}$. Este es el caso del \textit{oscilador amortiguado}.

Para el caso $\beta = \omega_o$ se tiene $\omega^2 = 0$ y la raíz del polinomio característica resulta doble y única. Esto conlleva a una solución de la forma:

\begin{equation}{\label{sub}}
q(t) = e^{-\beta t}(A+B.t) + V_0C
\end{equation}

donde A y B nuevamente se obtienen a través de las condiciones iniciales del sistema. Este es el caso límite conocido como \textit{oscilador subamortiguado}.

Finalmente, para el caso $\beta > \omega_o$ es $\omega^2>0$ y las raíces del polinomio característico son reales y distintas. Esto conlleva a una solución exponencial de la forma:

\begin{equation}{\label{sobre}}
q(t) = e^{-\beta t}[C.cosh(\omega t)+D.sinh(\omega t)]
\end{equation}

donde C y D se obtienen a través de las condiciones iniciales del sistema. Este caso corresponde al \textit{oscilador sobreamortiguado}, dado que al ser una combinación lineal de exponenciales decrecientes ($\beta > \omega > 0$) nunca puede completarse una oscilación. 

Si, como en el caso de los circuitos RC y RL, se utiliza una señal cuadrada para simular una \textit{llave ideal}, dado que todos los comportamientos cumplen $q(t\rightarrow\infty) = 0$, los períodos donde $V = 0$ resultan en una solución $q(t) = 0$ pues tanto las condiciones iniciales como la inhomogeneidad de la ecuación diferencial son nulas. Cuando la señal vuelve a adquirir el valor $V = V_0$, se recuperan las soluciones anteriores. Sin embargo, para esto se requiere una frecuencia de la señal cuadrada $\frac{\tau_f}{2} > \beta$ para asegurarse de que la carga $q(t)$ decaiga lo suficiente. 


%%%%%%%%%%%%%%%%%%%%%%%%%%%%%%%%%%%%%%%%%%%%%%%%%%%%%%%%%%%%%%%%%%%%%%%%%%%%%%%%%%%%%%%%%%%%%%%%%%%%%%%%%%%%%%%%%%%%%%%%%%%%%%%
% 3. DISPOSITIVO EXPERIMENTAL: armado del modelo, como se midio, consideraciones a la hora de medir.
%%%%%%%%%%%%%%%%%%%%%%%%%%%%%%%%%%%%%%%%%%%%%%%%%%%%%%%%%%%%%%%%%%%%%%%%%%%%%%%%%%%%%%%%%%%%%%%%%%%%%%%%%%%%%%%%%%%%%%%%%%%%%%%

\section{Desarrollo experimental}

Dado que toda la teoría previa se apoya sobre el concepto de \texit{llave ideal}, un generador de funciones capaces de generar ondas cuadradas debió ser utilizado para emular esta característica (variación instantánea de voltaje). Este generador es capaz de emitir frecuencias con un error relativo del $0,01\%$ en un rango entre $1\mu Hz$ y $5MHz$ cuyo voltaje pico-pico tiene un error relativo del $1\%$ para el rango de voltaje utilizado. Además, se utilizó una capacitancia fija $C = (100,0 \pm 0,2)nF$ y una resistencia variable por décadas cuyo error fue a priori desconocido. Usando un multimetro digital se midieron las resistencias utilizadas junto con su error, que era de la forma $\pm(1\%+2d)$ para el rango de resistencias utilizadas (mayores a $100\Omega$). La resistencia del capacitor resultó despreciable. También, se utilizó una inductancia fija $L = (1.000 \pm 0.002) H$ que poseía una resistencia interna (medida por el multímetro) $R_L = (294 \pm 3) \Omega$.

Finalmente, se utilizó un osciloscopio digital capaz de medir diferencias de potencial entre las dos terminales que dispone en un rango de 2mV a 5V con un error relativo del $3\%$. A la hora de medir voltaje, fue necesario asegurarse que el cable a tierra del osciloscopio estuvera conectado al cable a tierra el generador de funciones. 

\subsection{Circuito RC}

La primera parte del trabajo consistió en caracterizar un circuito RC. Para ello se montó un circuito utilizando el generador de funciones $\varepsilon$, la resistencia variable por décadas $R$ y el capacitor $C = (100,0 \pm 0,2)nF$, conectándose en serie para formar un circuito cerrado de una única malla como se ve en la \textbf{Figura \ref{fig:RC}}. 

\begin{figure}[h]
\centering
  \includegraphics[scale=0.7]{Circuito-RC}
  \caption{Circuito RC con una fuente de onda cuadrada}
  \label{fig:RC}
\end{figure}

El objetivo fue medir la carga y descarga del capacitor $C$ y la corriente sobre la resistencia $R$ (que resultan proporcionales a la caida de potencial en cada elemento, respectivamente) para obtener el tiempo característico $\tau_{RC} = RC$del circuito presente en las ecuaciones \eqref{RC_carga}, \eqref{RC_descarga}, \eqref{RC_carga_I} y \eqref{RC_descarga_I}.
 
Se programó la fuente para que emitiera una señal cuadrada con un voltaje máximo $\varepsilon = (2.00 \pm 0.02)V$ y un voltaje mínimo nulo cuya frecuencia era variable. Sin embargo, esta frecuencia $f = \frac{\omega_f}{2\pi}$ debió elegirse de forma tal que la carga $q$ llegará a su máximo o mínimo (y por ende la corriente $I$ se anulase) en medio período de la señal $\frac{\tau_f}{2} = \frac{pi}{\omega_f} > \tau_{RC}$ para poder analizar todo el comportamiento.

Para medir la diferencia de tension se conectó un osciloscopio en paralelo al elemento que se quería medir, como se puede ver en la \textbf{Figura \ref{fig:RC}} donde muestra el caso de la resistencia. A su vez, el generador de funciones estaba conectado al osciloscopio para poder visualizar la forma funcional de la fuente. Se lo utilizó además como \textit{trigger externo} para poder apreciar que tanto su señal como la caida de potencial en el elemento se encontraban en fase. Y utilizando un software de recopilación de datos se pudo importar a una computadora las mediciones registradas por el osciloscopio para ser analizadas posteriormente. Este proceso se realizó para disintos valores de $R$, considerandose despreciable la resistencia aportada por $C$.

\subsection{Circuito RL}

De manera análoga, la siguiente parte del trabajo consistió en caracterizar un circuito RL. En este caso se reemplazo el capacitor $C$ por la inductancia $L = (1.000 \pm 0.002) H$. La forma del circuito se puede ver en la \textbf{Figura \ref{fig:RL}}.

\begin{figure}[h]
\centering
\includegraphics[scale=0.7]{Circuito-RL}
  \caption{Circuito RL con una fuente de onda cuadrada}
  \label{fig:RL}
\end{figure}

El método utilizado para obtener el tiempo característico $\tau_{RL} = \frac{L}{R}$, fue el mismo que para el circuito RC exceptuando el voltaje entregado por la fuente cuyo valor fue $\varepsilon = (8.00 \pm 0.08)V$ y la resistencia de la inductancia $R_L = (294 \pm 3) \Omega$, que en este caso no pudo ser despreciada, debió sumarse a la resistencia $R$ pues estaba en serie. Se midieron las caidas de potencial en la resistencia $R$ (proporcional a la corriente $I$) y en la inductancia $L$ (proporcional a $\frac{dI(t)}{dt}$).

Nuevamente, se buscaron frecuencias $f = \frac{\omega_f}{2\pi}$ tal que la corriente se estabilizara (llegara a un máximo o mínimo) en medio período de la señal $\frac{\tau_f}{2} = \frac{pi}{\omega_f} > \tau_{RL}$ para poder analizar todo el comportamiento.

\subsection{Circuito RCL}

Por ultimo, se buscó estudiar el comportamiento de un circuito RCL. Para ello se montó un circuito cerrado que tenia en serie la fuente programable $\varepsilon$, la resistencia variable por decadas $R$, una inductancia variable $L$ y un capacitor $C$ como muestra la \textbf{Figura \ref{fig:RCL}}.  

\begin{figure}[h]
\centering
\includegraphics[scale=0.7]{Circuito-RCL}
  \caption{Circuito RCL con una fuente de onda cuadrada}
  \label{fig:RCL}
\end{figure}

Para caracterizar el circuito, la ecuación diferencia \eqref{ec_RCL} determina distintos \textit{comportamientos} que dependen de los parámetros involucrados en el circuito.

Utilizando el método de adquisición de datos ya explicado, se midió la diferencia de potencial sobre la resistencia (que resulta proporcional a la corriente $I$). Esta diferencia de potencial era generada por una señal cuadrada de $(3.08 \pm 0.04)V$.

Por lo tanto se estudió el circuito para disintos valores de $R$,$C$ y $L$ con el fin de encontrar cada uno de estos \textit{comportamientos}. Para ello, se fijaban dos parametros y se buscaba que el tercero cumpliese las condiciones  que determinan cada \textit{comportamiento} \eqref{amort}, \eqref{sub} y \eqref{sobre}.

Para cada comportamiento, como se dijo en la \textbf{Introducción}, se buscó una frecuencia $f$ tal que la corriente $I$ se anulara y, por lo tanto, se descargara el capacitor $C$ ($q=0$) en un medio período de la señal $\frac{\tau_f}{2} = \frac{pi}{\omega_f} > \beta$.

\newpage

%%%%%%%%%%%%%%%%%%%%%%%%%%%%%%%%%%%%%%%%%%%%%%%%%%%%%%%%%%%%%%%%%%%%%%%%%%%%%%%%%%%%%%%%%%%%%%%%%%%%%%%%%%%%%%%%%%%%%%%%%%%%%%%%
% 4.DISCUSIÓN Y RESULTADOS: todo lo que se obtuvo y explicación. Graficos, tablas.
%%%%%%%%%%%%%%%%%%%%%%%%%%%%%%%%%%%%%%%%%%%%%%%%%%%%%%%%%%%%%%%%%%%%%%%%%%%%%%%%%%%%%%%%%%%%%%%%%%%%%%%%%%%%%%%%%%%%%%%%%%%%%%%%

\section{Resultados}
\label{sec:discusion}

\subsection{Circuito RC}
Para la primer configuracion del circuito RC, se dispuso una resistencia de $R = (3,00\pm0,02)k\Omega$, se midio la caida de potencial sobre la resistencia y mediante el analizador de datos se obtuvo un grafico que ilusta la \textbf{Figura \ref{fig:RC-CR}}.

\begin{figure}[H]
\centering
\includegraphics[scale=0.45]{RC-Caida_en_Resistencia}
  \caption{Caida de potencial sobre la resistencia en función del tiempoen un circuito RC}
  \label{fig:RC-CR}
\end{figure}

Sabiendo que la variacion en el voltaje medida sobre la resistencia es proporcional a la corriente, se utilizaron las ecuaciones \eqref{RC_carga_I} y  \eqref{RC_descarga_I} para realizar ajustes sobre cada uno de los picos que aparecian en el grafico y así obtener un valor de $\tau_{RC}$ para cada uno. Y luego se realizo un intervalo de confianza de nivel $0.95$ \textbf{(Ver apendice)} con los todos los valores de $\tau_{RC}$, para obtener un valor representativo del tiempo caracteristico de este circuito, que resultó $\tau_{RC}=(0,315 \pm 0,004) ms$.

Cuando la medicion se realizo sobre el capacitor $C$ se obtuvo el grafico ilustrado por la \textbf{Figura \ref{fig:RC-CC}}, en el cual se utilizaron las ecuaciones \eqref{RC_carga} y \eqref{RC_descarga}; y realizandose un análisis idéntico al realizado con los valores de la \textbf{Figura \ref{fig:RC-CR}}, se obtuvo un $\tau_{RC}=(0,316 \pm 0,005) ms$.

\begin{figure}[H]
\centering
\includegraphics[scale=0.45]{RC-Caida_en_Capacitor}
  \caption{Caida de potencial sobre el capacitor en función del tiempoen un circuito RC}
  \label{fig:RC-CC}
\end{figure}

Se puede ver que los resultados de los obtenidos de las distintas mediciones son indistigibles, ya que sus intervalos de incerteza se solapan, lo cual resulta consistente ya que el tiempo caracteristico depende intrinsecamente de la configuracion del circuito y no del lugar de la medicion.

Posteriormente se realizo el mismo analisis para configuraciones con resistencias de $(5,00\pm0,03)k\Omega$, $(7,00\pm0,04)k\Omega$, $(9,00\pm0,05)k\Omega$ y $(11,00\pm0,05)k\Omega$, sin variar los demas parametros y se obtuvieron los valores presentados en la \textbf{Tabla 1}, donde se puede observar lo resaltado anteriormente; para cada par de resultados correspondientes a una misma configuracion, estos son indistingibles.\\ 

\begin{center}
\begin{tabular}{||c|c|c||}
\hline
& \multicolumn{2}{c||}{\textbf{Valor}} \\ \hline
\textbf{Resistencia (k$\Omega$)} & \textbf{$\tau_{R}$ (ms)} & \textbf{$\tau_{C}$ (ms)} \\ \hline 
$3,00\pm0,02$ & $0,315 \pm 0,004$ & $0,316 \pm 0,005$ \\ \hline 
$5,00\pm0,03$ & $0,516\pm 0,004$ & $0,525 \pm 0,007$ \\ \hline 
$7,00\pm0,03$ & $0,715\pm 0,007$ & $0,725\pm 0,006$ \\ \hline 
$9,00\pm0,03$ & $0,916 \pm 0,006$ & $0,926 \pm 0,009$ \\ \hline 
$11,00\pm0,03$ & $1,117\pm 0,007$ & $1,125\pm 0,005$ \\ \hline 
\end{tabular}\\[0.3cm]
 
\textit{Tabla 1: Valores obtenidos de las tiempos caracteristicos en el circuito RC}
\end{center}

Finalmente, teniendo en cuenta que segun el modelo propuesto, el tiempo caracteristico de un circuito tiene una correspondencia lineal con la resistencia se construyeron dos de gráficos que relacionan estos valores y se realizo un ajuste sobre los mismos como se puede ver en la \textbf{Figura \ref{fig:RCvsR}}.

\begin{figure}[H]

\begin{subfigure}{0.5\textwidth}
\includegraphics[scale=0.35]{Tau-RCvsR_Resistencia}
  \caption{Medicion sobre la resistencia}
  \label{subfig:RCvsRR}
\end{subfigure}
\begin{subfigure}{0.5\textwidth}
\includegraphics[scale=0.35]{Tau-RCvsR_Capacitor}
  \caption{Medicion sobre el capacitor}
  \label{subfig:RCvsRC}
\end{subfigure}
  \caption{Variacion del tiempo caracteristico en funcion de la resistencia}
  \label{fig:RCvsR}
\end{figure}

El resultado esperado seria que la pendiente de ambos graficos sea equivalente al valor de la capacitancia $C = (100 \pm 0,2)nF$. Efectivamente para la \textbf{Figura \ref{subfig:RCvsRR}} la pendiente arroja un valor $m_{R} = (100.18 \pm 0,09)nF$ una ordenada $b_R = (14,6 \pm 0,7)\mu s$ con un coeficiente $R-Square = 1$ mientras que para la \textbf{Figura \ref{subfig:RCvsRC}} la pendiente es $m_{C} = (101,0 \pm 0,5)nF$ y a ordenada $b_C = (15,6 \pm 0,4)ms$ con un coeficiente $R-Square = 0.99991$. Los $R-square$ aseguran la bondad del ajuste y las ordenadas $b_C$, $B_R$ resultan despreciables frente a los valores $\tau_{RC}$ manejados. 


\subsection{Circuito RL}

El procedimiento realizado con el circuito RL es absolutamente analogo al realizado con el RC. Se realizaron mediciones para distintas configuraciones que relevaban multiples curvas como muestra la \textbf{Figura \ref{fig:RL_C}}. En el caso de los graficos de tipo \textbf{\ref{subfig:RL_CR}}, como corresponden a las mediciones sobre la resistencia, se utilizaron las ecuaciones \eqref{RL_carga_I} y \eqref{RL_descarga_I} para realizar los ajustes, ya que la caida de potencial medida en ese lugar es proporcional a la corriente. En cambio, para los graficos de tipo \textbf{\ref{subfig:RL_CI}}, correspondientes a las mediciones sobre la inductancia, se utilizaron las ecuaciones \eqref{RL_carga} y \eqref{RL_descarga} ya que, en este caso, es proporcional a la derivada de la corriente.

\begin{figure}[H]

\begin{subfigure}{0.5\textwidth}
\includegraphics[scale=0.3]{RL-Caida_en_Inductancia}
  \caption{Sobre la Inductancia}
  \label{subfig:RL_CI}
\end{subfigure}
\begin{subfigure}{0.5\textwidth}
\includegraphics[scale=0.3]{RL-Caida_en_Resistencia}
  \caption{Sobre la Resistencia}
  \label{subfig:RL_CR}
\end{subfigure}
  \caption{Caida de potencial en funcion del tiempo en un circuito RL}
  \label{fig:RL_C}
\end{figure}

Posteriormente, con los resultados provistos por los ajustes, se realizo el mismo análisis, obteniendose los resultados explayados en la \textbf{Tabla 2}.

\begin{center}
\begin{tabular}{||c|c|c||}
\hline
\multicolumn{3}{||c||}{\textbf{Valor}} \\ \hline
\textbf{Posicion osciloscopio} & \textbf{Resistencia (k$\Omega$)} & \textbf{$\tau_{RL}$ (ms)}\\ \hline 
\multirow {5}{2cm}{Resistencia} & $0.600\pm0.003$ & $1.373 \pm 0.006$ \\ \cline {2-3}
& $0.800\pm0,004$ & $1.04\pm 0.01$ \\ \cline {2-3} 
& $1.000\pm0,005$ & $0.770 \pm 0.003$ \\ \cline {2-3}
& $1.200\pm0,006$ & $0.656 \pm 0.004$ \\ \cline {2-3}
& $1.400\pm0,007$ & $0.570 \pm 0.002$ \\ \hline
\multirow {5}{2cm}{Inductancia} & $3,00\pm0,02$ & $0.27 \pm 0.03$ \\ \cline {2-3}
& $5.00\pm0,03$ & $0.157 \pm 0.002$ \\ \cline {2-3}
& $7.00\pm0,04$ & $0.114 \pm 0.002$ \\ \cline {2-3}
& $9.00\pm0,05$ & $0.091 \pm 0.005$ \\ \cline {2-3} 
& $11.00\pm0,05$ & $0.071 \pm 0.008$ \\ \hline
\end{tabular}\\[0.3cm]

 \textit{Tabla 2: Valores obtenidos de los tiempos caracteristicos en el circuito RL}
\end{center}

Luego se costruyo el \textbf{grafico \ref{fig:T-RL}} que relaciona las duplas, se realizo un ajuste esperando que se respete la relacion $\tau_{RL} = L/R$ y que, ademas, se consiga un valor de $L = (1.000 \pm 0.002) H$.

\begin{figure}[H]
\centering
\includegraphics[scale=0.4]{TauRCvsR_Todo_en_Serie}
  \caption{Variacion del tiempo caracteristico en funcion de la resistencia}
  \label{fig:T-RL}
\end{figure}

Efectivamente, una vez realizado ese proceso, se obtiene un $R-Square = 0.99819$ que ratifica la bondad del ajuste pero el valor de la inductancia calculado por este metodo fue $L = (0.80 \pm 0.01) H$, que es ligeramente inferior. Se desconocen las causas de esa diferencia.


\subsection{Circuito RCL}

Para realizar la primer medición, se fijó una resistencia $R= (400 \pm 4) \Omega$, una inductancia $L = (1,000 \pm 0.001) H$ que añadia una resistencia de $R_{L} = (296 \pm 3) \Omega$, y una capacitancia $C = (100,0 \pm 0.2) nF$. Bajo esos parametros se esperaba una respuesta regida por la ecuación \eqref{amort} y, efectivamente, se puede ver en la \textbf{Figura \ref{fig:RLC-A}} que el ajuste propuesto es correcto, pues arroja un $R-Square = 0.99827$. Cabe destacar, que para este tipo de circuito, no se realizó el analisis utilizado en los dos casos anteriores, por lo que se se priorizó obtener la maxima resolucion posible de un único pico con el osciloscopio.

\begin{figure}[H]
\centering
\includegraphics[scale=0.4]{RLC-Amortiguado(1H)}
  \caption{Variacion del potencial en funcion del tiempo en un circuito RLC (Oscilador amortiguado)}
  \label{fig:RLC-A}
\end{figure}

Del ajuste se puede extraer el valor de la frecuencia de oscilacion $f_{A} = (501.3 \pm 0.7) Hz$ el cual es consistente con la calculada de manera teorica, $f_{teo} = (500.2 \pm 0.6)Hz$, ya que como sus intervalos de incerteza se solapan, son indistigibles.

Luego, se propusieron parametros para lograr una respuesta regida por la ecuación \eqref{sobre}. Se dispuso una resistencia $R= (1110 \pm 11)\Omega$, una inductancia $L = (30,00 \pm 0.03) mH$ , y una capacitancia $C = (100,0 \pm 0.2) nF$. Y luego mediante el osciloscopio se obtuvo la \textbf{Figura \ref{fig:RLC-SA}}.

\begin{figure}[H]
\centering
\includegraphics[scale=0.45]{RLC-SobreAmortiguado}
  \caption{Variacion del potencial en funcion del tiempo en un circuito RLC (Oscilador Sobreamortiguado)}
  \label{fig:RLC-SA}
\end{figure}

El ajuste realizado sobre la misma arroja un $R-Square = 0.99829$, y aporta valores para la frecuencia \textbf{$f_{A} = (477 \pm 31)Hz$} y para el tiempo caracteristico \textbf{$\tau_{A)} = (28.05 \pm 0.02) \mu s$}. Si se los compara con los resultados teoricos, \textbf{$f_{teo} = (475 \pm 9)Hz$} y \textbf{$\tau_{teo} = (27,0 \pm 0.3) \mu s$}, puede verse que las frecuencias coinciden, y que, aunque la diferencia es pequeña frente a las magnitudes manejadas, los $\tau$ no.

Finalmente se intento construir un circuito para el caso critico, para el cual se fijo una inductancia $L= (1 \pm 0.001) H$ y una capacitancia de  $C = (100 \pm 0.2) nF$, y se se fijo una resistencia $R = (6.30 \pm 0.03) k\Omega$ la cual contenia dentro de su margen de incerteza el resultado de calcular el valor para el cual el polinomio caracteristico de la ecuacion \eqref{ec_RCL} diera una raiz doble, y asi la evolucion del sistema estuviera regido por la ecuacion \eqref{sub}. Asi se obtuvo el grafico de la \textbf{Figura \ref{fig:RLC-sA}}.

\begin{figure}[H]
\centering
\includegraphics[scale=0.45]{RLC-Amortiguado_Critico(MasPiola)}
  \caption{Variacion del potencial en funcion del tiempo en un circuito RLC (Oscilador Subamortiguado) ajustado como un oscilador sobreamortiguado}
  \label{fig:RLC-sA}
\end{figure}

Durante el analisis fue imposible realizar un ajuste a partir de la ecuacion \eqref{sub} y solo se pudo obtener un ajuste viable cuando se utilizo la ecuacion del oscilador sobreamortiguado \eqref{sobre}. El $R-Square = 0.99948$ confirma la bondad del ajuste, por lo cual es posible afirmar que la \textbf{Figura \ref{fig:RLC-sA}} no corresponde al caso critico.



%%%%%%%%%%%%%%%%%%%%%%%%%%%%%%%%%%%%%%%%%%%%%%%%%%%%%%%%%%%%%%%%%%%%%%%%%%%%%%%%%%%%%%%%%%%%%%%%%%%%%%%%%%%%%%%%%%%%%%%%%%%%%%%%
%	CONCLUSIONES
%%%%%%%%%%%%%%%%%%%%%%%%%%%%%%%%%%%%%%%%%%%%%%%%%%%%%%%%%%%%%%%%%%%%%%%%%%%%%%%%%%%%%%%%%%%%%%%%%%%%%%%%%%%%%%%%%%%%%%%%%%%%%%%%

\section{Conclusiones}
\label{sec:conclusiones}

Para comenzar, los datos arrojados para el caso del circuito RC son más que consistentes. Los tiempos caracteristicos $\tau_R$ y $\tau_C$ mostrados en la \textbf{Tabla 1} resultan indistinguibles dentro del error, comprobando las relacion entre las ecuaciones \eqref{RC_carga}, \eqref{RC_descarga}, \eqref{RC_carga_I} y \eqref{RC_descarga_I}. Ante todo, los ajustes lineales de la \textbf{Figura \ref{fig:RCvsR}} permiten comprobar con enorme seguridad la dependencia del tiempo característico con los parámetros $C$ y $R$, dado que ambas pendientes $m_R = (100,18 \pm 0,09)nF$ y $m_C = (101,0 \pm 0,5)nF$ resultan indistinguibles.

Sin embargo, para el caso del circuito RL, dado que la metodologia de medición y analisis utilizado fueron los mismos que para el circuito RC se obtuvo una inconsistencia en los resultado. Esto se debe a que de los valores registrados en la \textbf{Tabla 2} se obtiene un ajuste bondadoso de la relacion $\tau_{RL}$, el cual se ve en la \textbf{Figura 8}, pero que difiere del valor esperado. Y puesto que el marco teorico del trabajo se comprobó correcto en la sección del RC al igual que tanto los metodos de medición como de analisis; se puede atribuir esta inconsistencia a alguna caracteristica de los elementos no tenidos en cuenta o a un descuido experimental a la hora de las mediciones.

Por ultimo; el estudio de los circuitos RCL se pudieron observar eficientemente los casos de \textit{oscilador amortiguado} y \textit{oscilador sobreamortiguado}, los cuales se muestra en las \textbf{Figura 9} y \textbf{10} respectivamente. Esto se puede concluir a partir de que los ajustes realizados en cada caso se hicieron utilizando las funciones esperadas, y los estimadores de bondad aseguraron dichos ajustes. Por otro lado; de las mediciones realizadas para el caso del \textit{oscilador subamortiguado}, en los cuales se dispusieron los valores de los elementos de manera que coincidiesen con las condiciones teóricas, no se pudo ajustar la curva esperada para este caso, sin embargo si se logró utilizando el ajuste del caso \textit{oscilador sobreamortiguado}, el cual debido a la bondad de su ajuste, se pudo garantizar que efectivamente corresponde a dicho caso. Es por ello que de los distintos parámetros utilizados, ninguna configuración correspondió al caso crítico. Esto es esperable considerando que este caso es muy fino y que la precisión de los parámetros debe ser mayor que fue utilizada en la experiencia.


%%%%%%%%%%%%%%%%%%%%%%%%%%%%%%%%%%%%%%%%%%%%%%%%%%%%%%%%%%%%%%%%%%%%%%%%%%%%%%%%%%%%%%%%%%%%%%%%%%%%%%%%%%%%%%%%%%%%%%%%%%%%%%%%%
%	APÉNDICE: esas cosas extras que simplemente no tuvieron lo suficiente como para ganarse una sección propia.
%%%%%%%%%%%%%%%%%%%%%%%%%%%%%%%%%%%%%%%%%%%%%%%%%%%%%%%%%%%%%%%%%%%%%%%%%%%%%%%%%%%%%%%%%%%%%%%%%%%%%%%%%%%%%%%%%%%%%%%%%%%%%%%%%

\section{Apéndice}
\label{sec:apendice}

A la hora de analizar largas tiras de datos, los Intervalos de Confianza son herramientas muy útiles. Matemáticamente, si $X_1,..X_n$ son variables aleatorias identicamente distribuidas tal que su esperanza $E(X_i) = \mu$ y varianza $V(X_i) = \sigma^2 > 0$ es posible construir un intervalo de confianza del promedio $\overline{X}_n = \Sigma_{i=1}^{n} \frac{X_i}{n}$. Fisicamente, asumiendo las $X_1,..X_n$ como distintas mediciones de una misma magnitud $X$, es posible considerar $\mu = X$ y construir un intervalo de confianza para este parámetro $\overline{X}_n$. Sin embargo, es necesario conocer la distribución $F(\theta_1,..,\theta_m)$ tal que $X_i \sim F(\theta_1,..\theta_m)$. 

Se define $(a;b)_{\theta}$ como un intervalo de confianza de nivel $1-\alpha$ para un parámetro $\theta$ según:

\begin{equation}\label{inter}
\ P(\theta\in(a;b)_{\theta})= 1-\alpha
\end{equation}

Según el Teorema Central del Límite, cuando $n\rightarrow\infty$ resulta $\overline{X}_n \sim N(\mu, \frac{\sigma}{\sqrt{n}})$ donde $N$ representa la distribución normal. De esta forma, a la hora de estimar $\mu$ es posible, asumiendo $n$ suficientemente grande, utilizar un intervalo de confianza sobre la variable aleatoria $\overline{X}_n$. En particular, definiendo la variable aleatoria $Z \sim N(0,1)$, vale asintóticamente que $\overline{X}_n = \mu + \frac{\sigma}{\sqrt{n}}Z$ de forma tal que $\Phi(x) = \int_{-\infty}^x \frac{e^{\frac{x^2}{2}}}{\sqrt{2\pi}} dx = P(Z<x)$. Por otro lado, definiendo $z_{p}$ tal que $\Phi(z_{p}) = 1-p$ y recordando que $\Phi(-x) = 1-\Phi(x)$ es posible, a través de \eqref{inter}, definir un intervalo de confianza ($-z_{\frac{\alpha}{2}}$; $z_{\frac{\alpha}{2}}$):

\begin{equation}\label{inter2}
\ P(|Z|<z_{\frac{\alpha}{2}})= 1-\alpha
\end{equation}

De esta forma, asumiendo conocida la varianza $V(\overline{X}_n) = \sigma^2$ y utilizando que $Z = \frac{\overline{X}_n-\mu}{\sigma}.\sqrt{n}$ es posible utilizar \eqref{inter2} para obtener un intervalo de confianza de nivel $1-\alpha$ para $\mu$. Sin embargo, dado que $\sigma$ no es un parámetro conocido, es posible simplemente aproximarlo por el \textit{desvio estandar muestral} tal que $\sigma^2 \simeq \Sigma_{i=1}^{n}\frac{(x_i-\overline{x}_n)^2}{n-1}$. Definiendo $\sigma$ de esta forma y $\overline{x}_n = \Sigma_{i=1}^{n} \frac{x_i}{n}$ con $x_i$ el valor obtenido en la i-esima medición., se puede calcular un intervalo de confianza de nivel $1-\alpha$ de la forma:

\begin{equation}\label{inter_final}
\ \mu_{\alpha} = \overline{x}_n \pm z_{\alpha/2}.\frac{\sigma}{\sqrt{n}} \simeq \overline{x}_n \pm \frac{z_{\alpha/2}}{n} \sqrt{\Sigma_{i=1}^{n}(x_i-\overline{x}_n)^2}
\end{equation}


%%%%%%%%%%%%%%%%%%%%%%%%%%%%%%%%%%%%%%%%%%%%%%%%%%%%%%%%%%%%%%%%%%%%%%%%%%%%%%%%%%%%%%%%%%%%%%%%%%%%%%%%%%%%%%%%%%%%%%%%%%%%%%%%%
%	REFERENCIAS: libros, libros, libros.
%%%%%%%%%%%%%%%%%%%%%%%%%%%%%%%%%%%%%%%%%%%%%%%%%%%%%%%%%%%%%%%%%%%%%%%%%%%%%%%%%%%%%%%%%%%%%%%%%%%%%%%%%%%%%%%%%%%%%%%%%%%%%%%%%

%Ejemplo:
\begin{thebibliography}{1}
 \bibitem{Trelles} Felix Rodriguez Trelles, \textit{Temas de Electricidad y Magnetismo}, 1984, Editorial EUDEBA
\end{thebibliography}
%Para citar: blablabla \cite{Baird}
 
\end{document}





