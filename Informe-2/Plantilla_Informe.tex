\documentclass[11pt,a4paper]{article}

\usepackage[utf8]{inputenc}
\usepackage[spanish]{babel}
\usepackage{amsmath}
\usepackage{amsfonts}
\usepackage{amssymb}
\usepackage{makeidx}
\usepackage{graphicx}
\usepackage{lmodern}
\usepackage{kpfonts}
\usepackage{wrapfig}
\usepackage{caption}
\usepackage{subcaption}
\usepackage{booktabs}
\usepackage[nottoc,numbib]{tocbibind} %agrega la bibliografia al índice.
\usepackage[font={small,it}]{caption}
%\usepackage{fourier}
\usepackage[left=2cm,right=2cm,top=2cm,bottom=2cm,headheight=13.6pt]{geometry}
\usepackage{fancyhdr}
\usepackage{multirow}
\pagestyle{fancy}
\usepackage{mathrsfs}

%Para los gráficos en general, con las tablas...¡Ja!, arreglate.
%\begin{figure}[h!]
%\centering
%\includegraphics[width=0.7\textwidth]{} %nombre de la imagen, incluirla en el mismo directorio que este archivo.
%\caption*{} %rótulo, el asterico elimina la numeración automática. 
%\label{fig:} % para luego referirse con \ref{fig:}
%\end{figure}


\begin{document}


%%%%%%%%%%%%%%%%%%%%%%%%%%%%%%%%%%%%%%%%%%%%%%%%%%%%%%%%%%%%%%%%%%%%%%%%%%%%%%%%%%%%%%%%%%%%%%%%%%%%%%%%%%%%%%%%%%%%%%%%%%%%%%%%%
% 	TÍTULO
%%%%%%%%%%%%%%%%%%%%%%%%%%%%%%%%%%%%%%%%%%%%%%%%%%%%%%%%%%%%%%%%%%%%%%%%%%%%%%%%%%%%%%%%%%%%%%%%%%%%%%%%%%%%%%%%%%%%%%%%%%%%%%%%%

%%%%%%%%%%%%%%%%%%%%%%%%%%%%%%%%%%%%%%%%%
% University Assignment Title Page 
% LaTeX Template
% Version 1.0 (27/12/12)
%
% This template has been downloaded from:
% http://www.LaTeXTemplates.com
%
% Original author:
% WikiBooks (http://en.wikibooks.org/wiki/LaTeX/Title_Creation)
%
% License:
% CC BY-NC-SA 3.0 (http://creativecommons.org/licenses/by-nc-sa/3.0/)
% 
% Instructions for using this template:
% This title page is capable of being compiled as is. This is not useful for 
% including it in another document. To do this, you have two options: 
%
% 1) Copy/paste everything between \begin{document} and \end{document} 
% starting at \begin{titlepage} and paste this into another LaTeX file where you 
% want your title page.
% OR
% 2) Remove everything outside the \begin{titlepage} and \end{titlepage} and 
% move this file to the same directory as the LaTeX file you wish to add it to. 
% Then add \input{./title_page_1.tex} to your LaTeX file where you want your
% title page.
%
%%%%%%%%%%%%%%%%%%%%%%%%%%%%%%%%%%%%%%%%%

%----------------------------------------------------------------------------------------
%	PACKAGES AND OTHER DOCUMENT CONFIGURATIONS
%----------------------------------------------------------------------------------------

%\documentclass[12pt]{article}
%\usepackage[utf8]{inputenc}
%\usepackage[spanish]{babel}
%\begin{document}

\begin{titlepage}

\newcommand{\HRule}{\rule{\linewidth}{0.5mm}} % Defines a new command for the horizontal lines, change thickness here

\center % Center everything on the page
 
%----------------------------------------------------------------------------------------
%	HEADING SECTIONS
%----------------------------------------------------------------------------------------

\textsc{\Huge Universidad de Buenos Aires}\\[0.5cm]
\textsc{\LARGE Facultad de Ciencias Exactas y Naturales}\\[0.5cm] % Name of your university/college
\textsc{\Large Departamento de Física}\\[0.25cm] % Major heading such as course name

\begin{figure}[h]
  \centering
  \includegraphics[scale=0.15]{Logo_DF}
  \\[0.5cm]
\end{figure}

\textsc{\large Laboratorio 3}\\[0.25cm] % Minor heading such as course title

%----------------------------------------------------------------------------------------
%	TITLE SECTION
%----------------------------------------------------------------------------------------

\HRule \\[0.4cm]
{ \huge \bfseries Estudio de resonancia y antiresonancia en circuitos RLC, y filtros en circuitos RC utilizando señales sinusoidales }\\[0.2cm] % Title of your document
\HRule \\[1cm]
 
%----------------------------------------------------------------------------------------
%	AUTHOR SECTION
%----------------------------------------------------------------------------------------

\begin{minipage}{0.4\textwidth}
\begin{center} \large
\emph{Autores:}\\
\textsc{Andreu}, Gonzalo\\ % Your name
\textsc{Malpartida}, Bryan\\ % Your name
\textsc{Pugliese}, Facundo\\ % Your name


\end{center}
\end{minipage}
~ \\[1.25cm]
%\begin{minipage}{0.4\textwidth}
%\begin{flushright} \large
%\emph{Supervisor:} \\
%Dr. James \textsc{Smith} % Supervisor's Name
%\end{flushright}
%\end{minipage}\\[4cm]

% If you don't want a supervisor, uncomment the two lines below and remove the section above
%\Large \emph{Author:}\\
%John \textsc{Smith}\\[3cm] % Your name

%----------------------------------------------------------------------------------------
%	DATE SECTION
%----------------------------------------------------------------------------------------

%\vspace{\fill}


{\large 24 de Febrero de 2016}\\[1.75cm] % Date, change the \today to a set date if you want to be precise

%----------------------------------------------------------------------------------------
%	SUMMARY SECTION: No más de 15 renglones, no te zarpes
%----------------------------------------------------------------------------------------

\begin{center}
\large{\textbf{Resumen}}

\small{El objetivo del siguiente trabajo fue comprobar empíricamente los fenómenos de resonancia y anti-resonancia presente en circuitos eléctricos RLC, así como también el estudio de filtros pasa-bajos y filtros pasa-alto en circuitos RC y filtros pasa-banda en un circuito compuesto por dos RC conectados en paralelo. Para ello se observó, utilizando un osciloscopio digital, el comportamiento de la transferencia y desfasaje entre la señal de salida y una señal de entrada sinusoidal entregada por un fuente de alimentación programable.} % ACA VA EL RESUMEN


\end{center}


%----------------------------------------------------------------------------------------
%	LOGO SECTION
%----------------------------------------------------------------------------------------

%\includegraphics{Logo}\\[1cm] % Include a department/university logo - this will require the graphicx package
 
%----------------------------------------------------------------------------------------

\vfill % Fill the rest of the page with whitespace

\end{titlepage}
%\end{document} %incluir en el mismo directorio que este archivo. Equivalente a un copiar-pegar, nada de andar diciendo \begin{document} en la portada. Dejar el nombre de Caratula a la caratula.

%%%%%%%%%%%%%%%%%%%%%%%%%%%%%%%%%%%%%%%%%%%%%%%%%%%%%%%%%%%%%%%%%%%%%%%%%%%%%%%%%%%%%%%%%%%%%%%%%%%%%%%%%%%%%%%%%%%%%%%%%%%%%%%%%
% 	ENCABEZADO Y PIE DE PÁGINA.
%%%%%%%%%%%%%%%%%%%%%%%%%%%%%%%%%%%%%%%%%%%%%%%%%%%%%%%%%%%%%%%%%%%%%%%%%%%%%%%%%%%%%%%%%%%%%%%%%%%%%%%%%%%%%%%%%%%%%%%%%%%%%%%%%

\lhead{}
\chead{}
\rhead{Laboratorio 3}
\lfoot{}
\cfoot{}
\rfoot{\thepage}
\renewcommand{\headrulewidth}{1pt}
\renewcommand{\footrulewidth}{1pt}
\newcommand\Sum{\displaystyle\sum}

%%%%%%%%%%%%%%%%%%%%%%%%%%%%%%%%%%%%%%%%%%%%%%%%%%%%%%%%%%%%%%%%%%%%%%%%%%%%%%%%%%%%%%%%%%%%%%%%%%%%%%%%%%%%%%%%%%%%%%%%%%%%%%%
% Página en blanco. Cita, agradecimiento, dedicación, lo que sea pero que sea algo.
%%%%%%%%%%%%%%%%%%%%%%%%%%%%%%%%%%%%%%%%%%%%%%%%%%%%%%%%%%%%%%%%%%%%%%%%%%%%%%%%%%%%%%%%%%%%%%%%%%%%%%%%%%%%%%%%%%%%%%%%%%%%%%%


%%%%%%%%%%%%%%%%%%%%%%%%%%%%%%%%%%%%%%%%%%%%%%%%%%%%%%%%%%%%%%%%%%%%%%%%%%%%%%%%%%%%%%%%%%%%%%%%%%%%%%%%%%%%%%%%%%%%%%%%%%%%%%%%%
% 	ÍNDICE
%%%%%%%%%%%%%%%%%%%%%%%%%%%%%%%%%%%%%%%%%%%%%%%%%%%%%%%%%%%%%%%%%%%%%%%%%%%%%%%%%%%%%%%%%%%%%%%%%%%%%%%%%%%%%%%%%%%%%%%%%%%%%%%%%

%\tableofcontents %compilar dos o tres veces para verlo bien. ¡Todo un índice en unas cuantas letras!
%\newpage

%%%%%%%%%%%%%%%%%%%%%%%%%%%%%%%%%%%%%%%%%%%%%%%%%%%%%%%%%%%%%%%%%%%%%%%%%%%%%%%%%%%%%%%%%%%%%%%%%%%%%%%%%%%%%%%%%%%%%%%%%%%%%%%
% 1. RESUMEN
%%%%%%%%%%%%%%%%%%%%%%%%%%%%%%%%%%%%%%%%%%%%%%%%%%%%%%%%%%%%%%%%%%%%%%%%%%%%%%%%%%%%%%%%%%%%%%%%%%%%%%%%%%%%%%%%%%%%%%%%%%%%%%%

%\section{Resumen}
%\label{sec:resumen}



%%%%%%%%%%%%%%%%%%%%%%%%%%%%%%%%%%%%%%%%%%%%%%%%%%%%%%%%%%%%%%%%%%%%%%%%%%%%%%%%%%%%%%%%%%%%%%%%%%%%%%%%%%%%%%%%%%%%%%%%%%%%%%%
% 2. INTRODUCCIÓN: ecuaciones aquí, luego se las cita.
%%%%%%%%%%%%%%%%%%%%%%%%%%%%%%%%%%%%%%%%%%%%%%%%%%%%%%%%%%%%%%%%%%%%%%%%%%%%%%%%%%%%%%%%%%%%%%%%%%%%%%%%%%%%%%%%%%%%%%%%%%%%%%%

\section{Introducción}\label{sec:intro}
Para llevar a cabo el objetivo de caracterizar los circuitos RC, RL y RCL es necesario definir los siguientes conceptos.

Si consideramos un circuito cerrado de una unica malla compuesto por una fuente de alimentación constante $V_{0}$, una resistencia $R$ y un capacitor $C$, la ecuación que rige la evolución del sistema es $V_{0} = \frac{q(t)}{C}+\dfrac{dq(t)}{dt}R$ donde $q(t)$ es la carga eléctrica del sistema. Entonces, si tomamos como condición inicial $q(0)=0$ la solución de la ecuación es:
   
\begin{equation}{\label{RC carga}}
q(t)=V_{0}C(1-e^\frac{-t}{RC})
\end{equation}

Y al ser $q(t)$ una exponencial creciente se denomina a este fenómeno como la carga del capacitor.

Ahora si, utilizando una \textit{llave ideal}, se corta la fuente $V$ a tiempo $t_{0}$ la nueva solución del sistema es de la forma:

\begin{equation}{\label{RC descarga}}
q(t-t_{0})=V_{0}C(e^\frac{t_{0}}{RC}-1)e^\frac{-t}{RC}
\end{equation}

Entonces se define a $\tau_{RC}=RC$ \textit{tiempo característico} del circuito. Por lo tanto, si $\tau_{RC}<t_{0}$ se puede observar el efecto de descarga del capacitor pues $q(t-t_{0})$ es una exponencial decreciente. 

\begin{equation}{\label{RL carga}}
\dfrac{dI(t)}{dt}=\frac{V_{0}}{L}e^\frac{-Rt}{L}
\end{equation}

\begin{equation}{\label{RL descarga}}
\dfrac{dI(t-t_{0})}{dt}=-\frac{V_{0}}{L}(e^\frac{Rt_{0}}{L}-1)e^\frac{-tR}{L}
\end{equation}

\begin{equation}
q(t)= Ae^{-\alpha t + \beta t} + Be^{-\alpha t - \beta t}
\end{equation}

$\alpha = \frac{R}{2L}$,  $\beta^2 = (\frac{R}{2L})^2-\frac{1}{LC}$

Si $\beta^2 < 0$

$q(t)= Ae^{-\alpha t} \sin(\sqrt{-\beta^2} + \phi_{0})$

Si $\beta^2 = 0$

$q(t) = (A+Bt)e^{-\alpha t}$ 

Si $\beta^2 > 0$
$q(t)= Ae^{-\alpha t} \sinh(\beta t + \phi_{0})$
%%%%%%%%%%%%%%%%%%%%%%%%%%%%%%%%%%%%%%%%%%%%%%%%%%%%%%%%%%%%%%%%%%%%%%%%%%%%%%%%%%%%%%%%%%%%%%%%%%%%%%%%%%%%%%%%%%%%%%%%%%%%%%%
% 3. DISPOSITIVO EXPERIMENTAL: armado del modelo, como se midio, consideraciones a la hora de medir.
%%%%%%%%%%%%%%%%%%%%%%%%%%%%%%%%%%%%%%%%%%%%%%%%%%%%%%%%%%%%%%%%%%%%%%%%%%%%%%%%%%%%%%%%%%%%%%%%%%%%%%%%%%%%%%%%%%%%%%%%%%%%%%%

\section{Desarrollo experimental}

\subsection{Circuito RC}

La primera parte del trabajo consistió en caracterizar un circuito RC. Para ello se montó un circuito utilizando un generador de funciones $\varepsilon$, una resistencia variable por décadas $R$ y un capacitor $C = (100 \pm 0,2)nF$, conectándose en serie para formar un circuito cerrado de una única malla como se ve en la \textbf{Figura \ref{fig:RC}}. 

\begin{figure}[h]
\centering
  \includegraphics[scale=0.7]{Circuito-RC}
  \caption{Circuito RC con una fuente de onda cuadrada}
  \label{fig:RC}
\end{figure}

El objetivo fue medir la carga y descarga del capacitor $C$, y la corriente sobre la resistencia $R$ para obtener el tiempo característico $\tau_{RC}$ del circuito determinado por (\textbf{ecucaciones de RC}).
 
Con el fin de recrear el efecto de la $llave$ $ideal$ se programó la fuente para que emitiera una señal cuadrada con un voltaje máximo $\varepsilon = (2.00 \pm 0.02)V$ y un voltaje mínimo nulo cuya frecuencia era variable.

Para medir la diferencia de tension se conectó un osciloscopio en paralelo al elemento que se quería medir, como se puede ven en la \textbf{Figura \ref{fig:RC}} donde muestra el caso de la resistencia. A su vez, el generador de funciones estaba conectado al osciloscopio para poder visualizar la forma funcional de la fuente. Se lo utilizó además como \textit{trigger externo} para poder apreciar que tanto su señal como la caida de potencial en el elemento se encontraban en fase. Y utilizando un software de recopilación de datos se pudo importar a una computadora las mediciones registradas por el osciloscopio para ser analizadas posteriormente. Este proceso se realizó para disintos valores de $R$, considerandose despreciable la resistencia aportada por $C$.

\subsection{Circuito RL}

De manera análoga, la siguiente parte del trabajo consistió en caracterizar un circuito RL. En este caso se reemplazo el capacitor $C$ por una inductancia $L = (1.000 \pm 0.002) H$. La forma del circuito se puede ver en la \textbf{Figura \ref{fig:RL}}.

\begin{figure}[h]
\centering
\includegraphics[scale=0.7]{Circuito-RL}
  \caption{Circuito RL con una fuente de onda cuadrada}
  \label{fig:RL}
\end{figure}

El método utilizado para obtener el tiempo característico $\tau_{RL}$, determinado por (\textbf{Ecuaciones de RL}), fue el mismo que para el circuito RC exceptuando el voltaje entregado por la fuente cuyo valor fue $\varepsilon = (8.00 \pm 0.08)V$ y la resistencia de la inductancia $R_L = (294 \pm 3) \Omega$, que en este caso no pudo ser despreciada, debió sumarse a la resistencia $R$ pues estaba en serie.

\subsection{Circuito RCL}

Por ultimo, se buscó estudiar el comportamiento de un circuito RCL. Para ello se montó un circuito cerrado que tenia en serie la fuente programable $\varepsilon$, la resistencia variable por decadas $R$, una inductancia variable $L$ y un capacitor $C$ como muestra la \textbf{Figura \ref{fig:RCL}}.  

\begin{figure}[h]
\centering
\includegraphics[scale=0.7]{Circuito-RCL}
  \caption{Circuito RCL con una fuente de onda cuadrada}
  \label{fig:RCL}
\end{figure}

Para caracterizar el circuito, la ecuación (\textbf{ecuaciones de RCL}) determina distintos \textit{comportamientos} que dependen de los parámetros que se usen.

Utilizando el método de adquisición de datos ya explicado, se midió la diferencia de potencial sobre la resistencia. Esta diferencia de potencial era generada por una señal cuadrada de $(3.08 \pm 0.04)V$.

Por lo tanto se estudió el circuito para disintos valores de $R$,$C$ y $L$ con el fin de encontrar cada uno de estos \textit{comportamientos}. Para ello, se fijaban dos parametros y se buscaba que el tercero cumpliese las condiciones (\textbf{discriminante del polinomio que sale de la ecuacion diferencial}) que determinan cada \textit{comportamiento}.


%%%%%%%%%%%%%%%%%%%%%%%%%%%%%%%%%%%%%%%%%%%%%%%%%%%%%%%%%%%%%%%%%%%%%%%%%%%%%%%%%%%%%%%%%%%%%%%%%%%%%%%%%%%%%%%%%%%%%%%%%%%%%%%%
% 4.DISCUSIÓN Y RESULTADOS: todo lo que se obtuvo y explicación. Graficos, tablas.
%%%%%%%%%%%%%%%%%%%%%%%%%%%%%%%%%%%%%%%%%%%%%%%%%%%%%%%%%%%%%%%%%%%%%%%%%%%%%%%%%%%%%%%%%%%%%%%%%%%%%%%%%%%%%%%%%%%%%%%%%%%%%%%%

\section{Resultados}
\label{sec:discusion}

\subsection{Circuito RC}


\subsection{Circuito RL}


\subsection{Circuito RCL}



%%%%%%%%%%%%%%%%%%%%%%%%%%%%%%%%%%%%%%%%%%%%%%%%%%%%%%%%%%%%%%%%%%%%%%%%%%%%%%%%%%%%%%%%%%%%%%%%%%%%%%%%%%%%%%%%%%%%%%%%%%%%%%%%
%	CONCLUSIONES
%%%%%%%%%%%%%%%%%%%%%%%%%%%%%%%%%%%%%%%%%%%%%%%%%%%%%%%%%%%%%%%%%%%%%%%%%%%%%%%%%%%%%%%%%%%%%%%%%%%%%%%%%%%%%%%%%%%%%%%%%%%%%%%%

\section{Conclusiones}
\label{sec:conclusiones}




%%%%%%%%%%%%%%%%%%%%%%%%%%%%%%%%%%%%%%%%%%%%%%%%%%%%%%%%%%%%%%%%%%%%%%%%%%%%%%%%%%%%%%%%%%%%%%%%%%%%%%%%%%%%%%%%%%%%%%%%%%%%%%%%%
%	APÉNDICE: esas cosas extras que simplemente no tuvieron lo suficiente como para ganarse una sección propia.
%%%%%%%%%%%%%%%%%%%%%%%%%%%%%%%%%%%%%%%%%%%%%%%%%%%%%%%%%%%%%%%%%%%%%%%%%%%%%%%%%%%%%%%%%%%%%%%%%%%%%%%%%%%%%%%%%%%%%%%%%%%%%%%%%

\section{Apéndice}
\label{sec:apendice}

A la hora de analizar largas tiras de datos, los Intervalos de Confianza son herramientas muy útiles. Matemáticamente, si $X_1,..X_n$ son variables aleatorias identicamente distribuidas tal que su esperanza $E(X_i) = \mu$ y varianza $V(X_i) = \sigma^2 > 0$ es posible construir un intervalo de confianza del promedio $\overline{X}_n = \Sigma_{i=1}^{n} \frac{X_i}{n}$. Fisicamente, asumiendo las $X_1,..X_n$ como distintas mediciones de una misma magnitud $X$, es posible considerar $\mu = X$ y construir un intervalo de confianza para este parámetro $\overline{X}_n$. Sin embargo, es necesario conocer la distribución $F(\theta_1,..\theta_m)$ tal que $X_i \sim F(\theta_1,..\theta_m)$. 

Se define $(a;b)_{\theta}$ como un intervalo de confianza de nivel $1-\alpha$ para un parámetro $\theta$ según:

\begin{equation}\label{inter}
\ P(\theta\in(a;b)_{\theta})= 1-\alpha
\end{equation}

Según el Teorema Central del Límite, cuando $n\longrightarrow\infty$ resulta $\overline{X}_n \sim N(\mu, \frac{\sigma}{\sqrt{n}})$ donde $N$ representa la distribución normal cuya funcion de densidad se define como:

\begin{equation}\label{normal}
\ f_N(x) := \frac{e^{\frac{(x-\mu)^2}{2\sigma^2}}}{\sqrt{2\pi}\sigma}
\end{equation}

De esta forma, a la hora de estimar $\mu$ es posible, asumiendo $n$ suficientemente grande, utilizar un intervalo de confianza sobre la variable aleatoria $\overline{X}_n$. En particular, definiendo la variable aleatoria $Z \sim N(0,1)$, vale asintóticamente que $\overline{X}_n = \mu + \frac{\sigma}{\sqrt{n}}Z$ de forma tal que $\Phi(x) = \int_{-\infty}^x \frac{e^{\frac{x^2}{2}}}{\sqrt{2\pi}} dx = P(Z<x)$. Por otro lado, definiendo $z_{p}$ tal que $\Phi(z_{p}) = 1-p$ y recordando que $\Phi(-x) = 1-\Phi(x)$ es posible, a través de \eqref{inter}, definir un intervalo de confianza ($-z_{\frac{\alpha}{2}}$; $z_{\frac{\alpha}{2}}$):

\begin{equation}\label{inter2}
\ P(|Z|<z_{\frac{\alpha}{2}})= 1-\alpha
\end{equation}

De esta forma, asumiendo conocida la varianza $V(\overline{X}_n) = \sigma^2$ y utilizando que $Z = \frac{\overline{X}_n-\mu}{\sigma}.\sqrt{n}$ se obtiene utilizando \eqref{inter2} un intervalo de confianza de nivel $1-\alpha$ para $\mu$:

\begin{equation}\label{casi}
\ I_{\mu} = (\overline{x}_n-z_{\alpha/2}\frac{\sigma}{\sqrt{n}}; \overline{x}_n+z_{\alpha/2}\frac{\sigma}{\sqrt{n}})
\end{equation}

Donde $\overline{x}_n = \Sigma_{i=1}^{n} \frac{x_i}{n}$ con $x_i$ el valor obtenido en la i-esima medición. Dado que $\sigma$ no es un parámetro conocido, es posible simplemente aproximarla por el \textit{desvio estandar muestral} tal que $\sigma^2 \simeq \Sigma_{i=1}^{n}\frac{(x_i-\overline{x}_n)^2}{n-1}$. Definiendo $\sigma$ de esta forma, se puede calcular un intervalo de confianza de nivel $1-\alpha$ de la forma:

\begin{equation}\label{inter_final}
\ \mu_{\alpha} = \overline{x}_n \pm z_{\alpha/2}.\frac{\sigma}{\sqrt{n}} \simeq \overline{x}_n \pm \frac{z_{\alpha/2}}{n} \sqrt{\Sigma_{i=1}^{n}(x_i-\overline{x}_n)^2}
\end{equation}


%%%%%%%%%%%%%%%%%%%%%%%%%%%%%%%%%%%%%%%%%%%%%%%%%%%%%%%%%%%%%%%%%%%%%%%%%%%%%%%%%%%%%%%%%%%%%%%%%%%%%%%%%%%%%%%%%%%%%%%%%%%%%%%%%
%	REFERENCIAS: libros, libros, libros.
%%%%%%%%%%%%%%%%%%%%%%%%%%%%%%%%%%%%%%%%%%%%%%%%%%%%%%%%%%%%%%%%%%%%%%%%%%%%%%%%%%%%%%%%%%%%%%%%%%%%%%%%%%%%%%%%%%%%%%%%%%%%%%%%%

%Ejemplo:
\begin{thebibliography}{1}
 \bibitem{Berkeley} Frank S. Crawford, \textit{Berkeley physics course 3: Ondas}, 1994, Editorial Reverte S.A.
\end{thebibliography}
%Para citar: blablabla \cite{Baird}
 
\end{document}





