%%%%%%%%%%%%%%%%%%%%%%%%%%%%%%%%%%%%%%%%%
% University Assignment Title Page 
% LaTeX Template
% Version 1.0 (27/12/12)
%
% This template has been downloaded from:
% http://www.LaTeXTemplates.com
%
% Original author:
% WikiBooks (http://en.wikibooks.org/wiki/LaTeX/Title_Creation)
%
% License:
% CC BY-NC-SA 3.0 (http://creativecommons.org/licenses/by-nc-sa/3.0/)
% 
% Instructions for using this template:
% This title page is capable of being compiled as is. This is not useful for 
% including it in another document. To do this, you have two options: 
%
% 1) Copy/paste everything between \begin{document} and \end{document} 
% starting at \begin{titlepage} and paste this into another LaTeX file where you 
% want your title page.
% OR
% 2) Remove everything outside the \begin{titlepage} and \end{titlepage} and 
% move this file to the same directory as the LaTeX file you wish to add it to. 
% Then add \input{./title_page_1.tex} to your LaTeX file where you want your
% title page.
%
%%%%%%%%%%%%%%%%%%%%%%%%%%%%%%%%%%%%%%%%%

%----------------------------------------------------------------------------------------
%	PACKAGES AND OTHER DOCUMENT CONFIGURATIONS
%----------------------------------------------------------------------------------------

%\documentclass[12pt]{article}
%\usepackage[utf8]{inputenc}
%\usepackage[spanish]{babel}
%\begin{document}

\begin{titlepage}

\newcommand{\HRule}{\rule{\linewidth}{0.5mm}} % Defines a new command for the horizontal lines, change thickness here

\center % Center everything on the page
 
%----------------------------------------------------------------------------------------
%	HEADING SECTIONS
%----------------------------------------------------------------------------------------

\textsc{\Huge Universidad de Buenos Aires}\\[0.5cm]
\textsc{\LARGE Facultad de Ciencias Exactas y Naturales}\\[0.5cm] % Name of your university/college
\textsc{\Large Departamento de Física}\\[0.25cm] % Major heading such as course name

\begin{figure}[h]
  \centering
  \includegraphics[scale=0.15]{Logo_DF}
  \\[0.5cm]
\end{figure}

\textsc{\large Laboratorio 3}\\[0.25cm] % Minor heading such as course title

%----------------------------------------------------------------------------------------
%	TITLE SECTION
%----------------------------------------------------------------------------------------

\HRule \\[0.4cm]
{ \huge \bfseries Estudio de resonancia y antiresonancia en circuitos RLC, y filtros en circuitos RC utilizando señales sinusoidales }\\[0.2cm] % Title of your document
\HRule \\[1cm]
 
%----------------------------------------------------------------------------------------
%	AUTHOR SECTION
%----------------------------------------------------------------------------------------

\begin{minipage}{0.4\textwidth}
\begin{center} \large
\emph{Autores:}\\
\textsc{Andreu}, Gonzalo\\ % Your name
\textsc{Malpartida}, Bryan\\ % Your name
\textsc{Pugliese}, Facundo\\ % Your name


\end{center}
\end{minipage}
~ \\[1.25cm]
%\begin{minipage}{0.4\textwidth}
%\begin{flushright} \large
%\emph{Supervisor:} \\
%Dr. James \textsc{Smith} % Supervisor's Name
%\end{flushright}
%\end{minipage}\\[4cm]

% If you don't want a supervisor, uncomment the two lines below and remove the section above
%\Large \emph{Author:}\\
%John \textsc{Smith}\\[3cm] % Your name

%----------------------------------------------------------------------------------------
%	DATE SECTION
%----------------------------------------------------------------------------------------

%\vspace{\fill}


{\large 24 de Febrero de 2016}\\[1.75cm] % Date, change the \today to a set date if you want to be precise

%----------------------------------------------------------------------------------------
%	SUMMARY SECTION: No más de 15 renglones, no te zarpes
%----------------------------------------------------------------------------------------

\begin{center}
\large{\textbf{Resumen}}

\small{El objetivo del siguiente trabajo fue comprobar empíricamente los fenómenos de resonancia y anti-resonancia presente en circuitos eléctricos RLC, así como también el estudio de filtros pasa-bajos y filtros pasa-alto en circuitos RC y filtros pasa-banda en un circuito compuesto por dos RC conectados en paralelo. Para ello se observó, utilizando un osciloscopio digital, el comportamiento de la transferencia y desfasaje entre la señal de salida y una señal de entrada sinusoidal entregada por un fuente de alimentación programable.} % ACA VA EL RESUMEN


\end{center}


%----------------------------------------------------------------------------------------
%	LOGO SECTION
%----------------------------------------------------------------------------------------

%\includegraphics{Logo}\\[1cm] % Include a department/university logo - this will require the graphicx package
 
%----------------------------------------------------------------------------------------

\vfill % Fill the rest of the page with whitespace

\end{titlepage}
%\end{document}