\documentclass[11pt,a4paper]{article}

\usepackage[utf8]{inputenc}
\usepackage[spanish]{babel}
\usepackage{amsmath}
\usepackage{amsfonts}
\usepackage{amssymb}
\usepackage{makeidx}
\usepackage{graphicx}
\usepackage{lmodern}
\usepackage{kpfonts}
\usepackage{wrapfig}
\usepackage{caption}
\usepackage{subcaption}
\usepackage{booktabs}
\usepackage[nottoc,numbib]{tocbibind} %agrega la bibliografia al índice.
\usepackage[font={small,it}]{caption}
%\usepackage{fourier}
\usepackage[left=2cm,right=2cm,top=2cm,bottom=2cm,headheight=13.6pt]{geometry}
\usepackage{fancyhdr}
\usepackage{multirow}
\pagestyle{fancy}


%Para los gráficos en general, con las tablas...¡Ja!, arreglate.
%\begin{figure}[h!]
%\centering
%\includegraphics[width=0.7\textwidth]{} %nombre de la imagen, incluirla en el mismo directorio que este archivo.
%\caption*{} %rótulo, el asterico elimina la numeración automática. 
%\label{fig:} % para luego referirse con \ref{fig:}
%\end{figure}


\begin{document}


%%%%%%%%%%%%%%%%%%%%%%%%%%%%%%%%%%%%%%%%%%%%%%%%%%%%%%%%%%%%%%%%%%%%%%%%%%%%%%%%%%%%%%%%%%%%%%%%%%%%%%%%%%%%%%%%%%%%%%%%%%%%%%%%%
% 	TÍTULO
%%%%%%%%%%%%%%%%%%%%%%%%%%%%%%%%%%%%%%%%%%%%%%%%%%%%%%%%%%%%%%%%%%%%%%%%%%%%%%%%%%%%%%%%%%%%%%%%%%%%%%%%%%%%%%%%%%%%%%%%%%%%%%%%%

%%%%%%%%%%%%%%%%%%%%%%%%%%%%%%%%%%%%%%%%%
% University Assignment Title Page 
% LaTeX Template
% Version 1.0 (27/12/12)
%
% This template has been downloaded from:
% http://www.LaTeXTemplates.com
%
% Original author:
% WikiBooks (http://en.wikibooks.org/wiki/LaTeX/Title_Creation)
%
% License:
% CC BY-NC-SA 3.0 (http://creativecommons.org/licenses/by-nc-sa/3.0/)
% 
% Instructions for using this template:
% This title page is capable of being compiled as is. This is not useful for 
% including it in another document. To do this, you have two options: 
%
% 1) Copy/paste everything between \begin{document} and \end{document} 
% starting at \begin{titlepage} and paste this into another LaTeX file where you 
% want your title page.
% OR
% 2) Remove everything outside the \begin{titlepage} and \end{titlepage} and 
% move this file to the same directory as the LaTeX file you wish to add it to. 
% Then add \input{./title_page_1.tex} to your LaTeX file where you want your
% title page.
%
%%%%%%%%%%%%%%%%%%%%%%%%%%%%%%%%%%%%%%%%%

%----------------------------------------------------------------------------------------
%	PACKAGES AND OTHER DOCUMENT CONFIGURATIONS
%----------------------------------------------------------------------------------------

%\documentclass[12pt]{article}
%\usepackage[utf8]{inputenc}
%\usepackage[spanish]{babel}
%\begin{document}

\begin{titlepage}

\newcommand{\HRule}{\rule{\linewidth}{0.5mm}} % Defines a new command for the horizontal lines, change thickness here

\center % Center everything on the page
 
%----------------------------------------------------------------------------------------
%	HEADING SECTIONS
%----------------------------------------------------------------------------------------

\textsc{\Huge Universidad de Buenos Aires}\\[0.5cm]
\textsc{\LARGE Facultad de Ciencias Exactas y Naturales}\\[0.5cm] % Name of your university/college
\textsc{\Large Departamento de Física}\\[0.25cm] % Major heading such as course name

\begin{figure}[h]
  \centering
  \includegraphics[scale=0.15]{Logo_DF}
  \\[0.5cm]
\end{figure}

\textsc{\large Laboratorio 3}\\[0.25cm] % Minor heading such as course title

%----------------------------------------------------------------------------------------
%	TITLE SECTION
%----------------------------------------------------------------------------------------

\HRule \\[0.4cm]
{ \huge \bfseries Estudio de resonancia y antiresonancia en circuitos RLC, y filtros en circuitos RC utilizando señales sinusoidales }\\[0.2cm] % Title of your document
\HRule \\[1cm]
 
%----------------------------------------------------------------------------------------
%	AUTHOR SECTION
%----------------------------------------------------------------------------------------

\begin{minipage}{0.4\textwidth}
\begin{center} \large
\emph{Autores:}\\
\textsc{Andreu}, Gonzalo\\ % Your name
\textsc{Malpartida}, Bryan\\ % Your name
\textsc{Pugliese}, Facundo\\ % Your name


\end{center}
\end{minipage}
~ \\[1.25cm]
%\begin{minipage}{0.4\textwidth}
%\begin{flushright} \large
%\emph{Supervisor:} \\
%Dr. James \textsc{Smith} % Supervisor's Name
%\end{flushright}
%\end{minipage}\\[4cm]

% If you don't want a supervisor, uncomment the two lines below and remove the section above
%\Large \emph{Author:}\\
%John \textsc{Smith}\\[3cm] % Your name

%----------------------------------------------------------------------------------------
%	DATE SECTION
%----------------------------------------------------------------------------------------

%\vspace{\fill}


{\large 24 de Febrero de 2016}\\[1.75cm] % Date, change the \today to a set date if you want to be precise

%----------------------------------------------------------------------------------------
%	SUMMARY SECTION: No más de 15 renglones, no te zarpes
%----------------------------------------------------------------------------------------

\begin{center}
\large{\textbf{Resumen}}

\small{El objetivo del siguiente trabajo fue comprobar empíricamente los fenómenos de resonancia y anti-resonancia presente en circuitos eléctricos RLC, así como también el estudio de filtros pasa-bajos y filtros pasa-alto en circuitos RC y filtros pasa-banda en un circuito compuesto por dos RC conectados en paralelo. Para ello se observó, utilizando un osciloscopio digital, el comportamiento de la transferencia y desfasaje entre la señal de salida y una señal de entrada sinusoidal entregada por un fuente de alimentación programable.} % ACA VA EL RESUMEN


\end{center}


%----------------------------------------------------------------------------------------
%	LOGO SECTION
%----------------------------------------------------------------------------------------

%\includegraphics{Logo}\\[1cm] % Include a department/university logo - this will require the graphicx package
 
%----------------------------------------------------------------------------------------

\vfill % Fill the rest of the page with whitespace

\end{titlepage}
%\end{document} %incluir en el mismo directorio que este archivo. Equivalente a un copiar-pegar, nada de andar diciendo \begin{document} en la portada. Dejar el nombre de Caratula a la caratula.

%%%%%%%%%%%%%%%%%%%%%%%%%%%%%%%%%%%%%%%%%%%%%%%%%%%%%%%%%%%%%%%%%%%%%%%%%%%%%%%%%%%%%%%%%%%%%%%%%%%%%%%%%%%%%%%%%%%%%%%%%%%%%%%%%
% 	ENCABEZADO Y PIE DE PÁGINA.
%%%%%%%%%%%%%%%%%%%%%%%%%%%%%%%%%%%%%%%%%%%%%%%%%%%%%%%%%%%%%%%%%%%%%%%%%%%%%%%%%%%%%%%%%%%%%%%%%%%%%%%%%%%%%%%%%%%%%%%%%%%%%%%%%

\lhead{}
\chead{}
\rhead{Laboratorio 3}
\lfoot{}
\cfoot{}
\rfoot{\thepage}
\renewcommand{\headrulewidth}{1pt}
\renewcommand{\footrulewidth}{1pt}


%%%%%%%%%%%%%%%%%%%%%%%%%%%%%%%%%%%%%%%%%%%%%%%%%%%%%%%%%%%%%%%%%%%%%%%%%%%%%%%%%%%%%%%%%%%%%%%%%%%%%%%%%%%%%%%%%%%%%%%%%%%%%%%
% Página en blanco. Cita, agradecimiento, dedicación, lo que sea pero que sea algo.
%%%%%%%%%%%%%%%%%%%%%%%%%%%%%%%%%%%%%%%%%%%%%%%%%%%%%%%%%%%%%%%%%%%%%%%%%%%%%%%%%%%%%%%%%%%%%%%%%%%%%%%%%%%%%%%%%%%%%%%%%%%%%%%


%%%%%%%%%%%%%%%%%%%%%%%%%%%%%%%%%%%%%%%%%%%%%%%%%%%%%%%%%%%%%%%%%%%%%%%%%%%%%%%%%%%%%%%%%%%%%%%%%%%%%%%%%%%%%%%%%%%%%%%%%%%%%%%%%
% 	ÍNDICE
%%%%%%%%%%%%%%%%%%%%%%%%%%%%%%%%%%%%%%%%%%%%%%%%%%%%%%%%%%%%%%%%%%%%%%%%%%%%%%%%%%%%%%%%%%%%%%%%%%%%%%%%%%%%%%%%%%%%%%%%%%%%%%%%%

%\tableofcontents %compilar dos o tres veces para verlo bien. ¡Todo un índice en unas cuantas letras!
%\newpage

%%%%%%%%%%%%%%%%%%%%%%%%%%%%%%%%%%%%%%%%%%%%%%%%%%%%%%%%%%%%%%%%%%%%%%%%%%%%%%%%%%%%%%%%%%%%%%%%%%%%%%%%%%%%%%%%%%%%%%%%%%%%%%%
% 1. RESUMEN
%%%%%%%%%%%%%%%%%%%%%%%%%%%%%%%%%%%%%%%%%%%%%%%%%%%%%%%%%%%%%%%%%%%%%%%%%%%%%%%%%%%%%%%%%%%%%%%%%%%%%%%%%%%%%%%%%%%%%%%%%%%%%%%

%\section{Resumen}
%\label{sec:resumen}



%%%%%%%%%%%%%%%%%%%%%%%%%%%%%%%%%%%%%%%%%%%%%%%%%%%%%%%%%%%%%%%%%%%%%%%%%%%%%%%%%%%%%%%%%%%%%%%%%%%%%%%%%%%%%%%%%%%%%%%%%%%%%%%
% 2. INTRODUCCIÓN: ecuaciones aquí, luego se las cita.
%%%%%%%%%%%%%%%%%%%%%%%%%%%%%%%%%%%%%%%%%%%%%%%%%%%%%%%%%%%%%%%%%%%%%%%%%%%%%%%%%%%%%%%%%%%%%%%%%%%%%%%%%%%%%%%%%%%%%%%%%%%%%%%

\section{Introducción}\label{sec:intro}

Para un circuito RLC con una fuente de corriente alterna de la forma $\varepsilon(t) = E_{0}cos(\omega t)$ con una resistencia $R$, una inductancia $L$ y una capacitancia $C$ en serie la ecuacion diferencial que rige la evolución del sistema está dada por $\frac{d^2q}{dt^2}L+\frac{dq}{dt}R+\frac{q}{C} = \varepsilon = E_{0}cos(\omega t)$ donde $q = q(t)$ representa la carga del capacitor $C$. En particular, es posible buscar la solución compleja de la ecuación $\frac{d^2Q}{dt^2}L+\frac{dQ}{dt}R+\frac{Q}{C} = \varepsilon = E_{0}e^{i\omega t}$ donde Q es tal que $Re(Q) = q$. Para este sistema, recordando que $\mathbb{I}(t) = \frac{dQ}{dt}$, la solución resulta $\mathbb{I}(t) = \frac{E_0e^{i\omega t}}{Z}$ donde $Z = R +i(\omega L-\frac{1}{\omega C})$ es la \textit{impedancia} total del circuito. Para esta corriente $\mathbb{I}(t)$, la caida de potencial sobre la resistencia $R$ resulta, según la Ley de Ohm: 

\begin{equation}
\mathbb{V}_R(t) = R\mathbb{I}(t) = \frac{RE_0e^{i\omega t}}{R +i(\omega L-\frac{1}{\omega C})}
\label{eq:ohm}
\end{equation}

Definiendo la \textit{amplitud de la corriente} a $I_0 = \frac{E_0}{|Z|}$ y la \textit{amplitud de la caida} $V_R = R.I_0$ resulta una campana de la forma

\begin{equation}
V_R(\omega) = \frac{E_{0}R}{\sqrt{R^2 + (\omega L - \frac{1}{\omega C})^2}}
\label{eq:1}
\end{equation}

Por otro lado, dado que la impedancia $Z$ es compleja, es posible escribirla como $Z = |Z|e^{i\phi}$ de forma tal que resulta $\mathbb{I}(t) = \frac{E_0e^{i\omega t-\phi}}{|Z|}$ donde $\phi$ representa el desfasaje de la corriente respecto a la fuente y tiene la forma

\begin{equation}
tan(\phi) = \frac{L\omega}{R} - \frac{1}{RC\omega} 
\label{eq:2}
\end{equation}

Volviendo a \eqref{eq:1}, dado que $E_0$ y $R$ son constantes, el valor de $\omega$ que maximiza la caida de potencial $V_R$ es el que minimiza el denominador, anulando la parte reactiva $(\omega L - \frac{1}{\omega C})$. Este $\omega_0$ que maximiza $V_R$ es llamado \textit{frecuencia de resonancia} y coincide con la \textit{frecuencia natural} de un circuito LC no resistivo (donde $R=0$). En particular, es de la forma 

\begin{equation}
\omega_{0}^2= \frac{1}{LC}
\label{eq:F_res}
\end{equation}

Para este valor, resulta $V_R(\omega_0) = E_{0}$. Definiendo una \textit{banda de resonancia} $[\omega_1;\omega_2]$ tal que $\frac{E_0}{\sqrt{2}} \leq V_R(\omega) \leq E_{0}$ $\forall$ $\omega \in [\omega_1;\omega_2]$ y el \textit{ancho de banda} $\Delta\omega = \omega_2 - \omega_1$ resulta

\begin{equation}
\Delta\omega = \frac{R}{L}
\label{deltaomega}
\end{equation}\\

Ahora, si se colocan el capacitor $C$ en paralelo a la inductancia $L$ y estas en serie a una fuente de corriente alterna y a la resistencia $R$ la impedancia tiene un nuevo valor $Z = R+ i(\frac{1}{\omega C - \frac{1}{\omega L}})$. Usando el razonamiento y las definiciones anteriores, si se mide la caida de potencial sobre la resistencia, la amplitud de la misma (proporcional a la amplitud de la corriente) es de la forma

\begin{equation}
V_{R} (\omega) = \frac{E_0R}{\sqrt{R^2 + \frac{1}{(\omega C - \frac{1}{\omega L})^2}}}
\label{eq:3}
\end{equation}

Nuevamente, es posible definir $Z = |Z|e^{i\phi}$ tal que $\mathbb{I}(t) = \frac{E_0e^{i\omega t-\phi}}{|Z|}$ donde $\phi$ tendrá esta vez la forma

\begin{equation}
tan(\phi) = \frac{L}{R}\frac{\omega}{(1- LC\omega^2)} 
\label{eq:4}
\end{equation}

En este caso, $V_{R} (\omega)$ no tiene máximo dado que la parte reactiva $\frac{1}{\omega C - \frac{1}{\omega L}}$ se anula asintóticamente para $\omega \longrightarrow 0$ o $\omega \longrightarrow \infty$  tal que $V_R(\omega \rightarrow 0) = E_0 = V_R(\omega \rightarrow \infty)$. No obstante, $V_{R} (\omega)$ tiene un mínimo en $\omega_0^2 = \frac{1}{LC}$ donde la parte reactiva diverge y resulta $V_{R} (\omega_0)=0$. Es por esto que $\omega_0$ es la \textit{frecuencia de antiresonancia} del circuito, dado que cualquier señal de frecuencia $\omega_0$ será completamente absorvida por el circuito. Definiendo nuevamente una \textit{banda de antiresonancia} $[\omega_1;\omega_2]$ tal que $\frac{E_0}{\sqrt{2}} \leq V_R(\omega) \leq E_{0}$ $\forall$ $\omega \in [\omega_1;\omega_2]$ y el ancho de banda $\Delta\omega$ resulta

\begin{equation}
\Delta\omega = \frac{1}{RC}
\label{deltaomega2}
\end{equation}\\ \\

Como puede verse, los circuitos con inductancias $L$ y capacitancias $C$ tienen tal interacción con las frecuencias de una fuente de corriente alterna que es posible combinarlos para que atenuen determinadas frecuencias dejando pasar otras. Esto es lo que se denomina como \textit{filtro}. Los filtros más básicos que existen son el Pasa-Altos y el Pasa-Bajos, que no son más que circuitos RC serie identicos. Lo que varía es la ubicación de las terminales con las que se conectan al resto del circuito. 

Usando una fuente de corriente alterna $\varepsilon(t) = E_{0}e^{i\omega t}$, para un circuito RC de resistencia $R$ y capacitancia $C$ la impendancia es de la forma $Z = R-\frac{i}{\omega C}$. Por lo tanto, la corriente resulta $I(t) = \frac{\varepsilon(t)}{Z}$. A la constante $E_0$ se la llama \textit{voltaje de entrada}. Nuevamente, la solución compleja es de la forma $\mathbb{I}(t) = I_0e^{i\omega t-\phi}$ donde $I_0 = \frac{E_0}{|Z|}$. A la amplitud de la caida de potencial sobre el elemento en cuestión (resistencia $R$ o capacitor $C$) se le llama \textit{voltaje de salida} $V_e$. Para normalizar la relación entre $V_e$ y $V_s$ se define la \textit{transmisión} como $T=\frac{|V_s|}{|V_e|}=\frac{|V_s|}{E_0}$. Además, se define la \textit{frecuencia de corte} de una transmisión como el $\omega_c$ tal que $T(\omega_c) = \frac{\sqrt{2}}{2}$.

Si se mide la caida de potencial sobre la resistencia $R$, la ley de ohm dicta que $\mathbb{V}_R = R.\mathbb{I} = RI_0e^{i\omega t -\phi}$ donde se define nuevamente $I_0 = \frac{E_0}{|Z|}$ de forma tal que resulta $V_s(\omega) = \frac{RE_0}{\sqrt{R^2+(\omega C)^2}}$. Por lo tanto, la transmisión tiene la forma

\begin{equation}
T_R(\omega) = \frac{R}{\sqrt{R^2+\frac{1}{(\omega C)^2}}}
\label{trans_PA}
\end{equation}

donde puede verse que $T_R$ crece en función de $\omega$ tal que $T_R(\omega \rightarrow \infty) = 1$ y $T_R(\omega \rightarrow 0) = 0$, por lo que se le llama a este sistema filtro Pasa-Altos. Para esta transmisión, la frecuencia de corte resulta $\omega_c = \frac{1}{RC}$.

Por otro lado, el desfasaje $\phi$ de $\mathbb{V}_R = R.\mathbb{I} = \frac{RE_0e^{i\omega t - \phi}}{\sqrt{R^2+(\omega C)^2}}$ resulta de la forma

\begin{equation}
tan(\phi_R) = \frac{1}{\omega R C}
\label{des_PA}
\end{equation}

tal que $\phi(\omega \rightarrow 0) = \frac{\pi}{2}$ y $\phi(\omega \rightarrow \infty) = 0$.

Alternativamente, si se mide la caida de potencial sobre el capacitor, usando las leyes de Kirchoff resulta $\mathbb{V}_C = E_0e^{i\omega t} - R.\mathbb{I}$ tal que $\mathbb{V}_C = -\frac{iE_0}{\omega C|Z|} = \frac{E_0e^{i\omega t -(\phi+\frac{pi}{2})}}{\sqrt{1+(\omega RC)^2}}$. Por lo tanto, la transmisión resulta

\begin{equation}
T_C(\omega) = \frac{1}{\sqrt{1+(\omega RC)^2}}
\label{trans_PBaj}
\end{equation}

donde $T_C$ es una función decreciente de $\omega$ tal que $T_R(\omega \rightarrow \infty) = 0$ y $T_R(\omega \rightarrow 0) = 1$, por lo que se le llama filtro Pasa-Bajos. Como antes, la frecuencia de corte resulta $\omega_c = \frac{1}{RC}$. Sin embargo, el desfasaje $\phi_C$ de este sistema resulta de la forma 

\begin{equation}
\phi_C = \frac{\pi}{2}+\phi_R = \frac{\pi}{2} - tan^{-1}(\frac{1}{\omega R C})
\label{des_PBaj}
\end{equation}

Finalmente, si se une un filtro Pasa-Altos con un Pasa-Bajos, se obtiene un filtro Pasa-Banda. Este circuito consiste en un circuito RC de resistencia $R_1$ y capacitancia $C_1$ con terminales alrededor este último que se unen a las terminales de otro circuito RC sin fuente y de resistencia $R_2$ y capacitancia $C_2$ como muestra la \textbf{Figura \ref{fig:intro}}.

\begin{figure}[h]
\centering
\includegraphics[scale=0.75]{Figura_Intro}
  \caption{Filtro Pasa-Banda compuesto por un filtro Pasa-Bajos (izquierda) unido a un filtro Pasa-Altos (derecha)}
  \label{fig:intro}
\end{figure}

donde el voltaje de salida $V_s$ es la caída de potencial sobre la resistencia $R$. Resolviendo el circuito, la transmición $T(\omega)$ resulta una campana de la forma

\begin{equation}
T(\omega) = [(R_1+R_2)^2(1+\frac{C_1}{C_2})^2+(R_1R_2C_1\omega-\frac{1}{\omega C_2})^2]^{-1/2}
\label{trans_PBanda}
\end{equation}

cuyo máximo se encuentra en $\omega_m^2 = \frac{1}{R_1R_2C_1C_2} = \omega_{PB}.\omega_{PA}$ donde $\omega_{PB}$ y $\omega_{PA}$ son las frecuencias de corte del Pasa-Bajos y Pasa-Altos, respectivamente. La máxima transmisión para esa frecuencia resulta $T(\omega_m) = [(1+\frac{R_1}{R_2})(1+\frac{C_1}{C_2})]^{-1} = T_{max}$. Definiendo las \textit{frecuencias de corte} $\omega_1$ y $\omega_2$ tales que $T(\omega_1) = \frac{T_{max}}{\sqrt{2}} = T(\omega_2)$, resultan

\begin{equation}
\omega_{1,2} = \sqrt{[\frac{1}{2}(\frac{1}{R_1}+\frac{1}{R_2})(\frac{1}{C_1}+\frac{1}{C_2})]^2+\frac{1}{R_1R_2C_1C_2}} \pm \frac{1}{2}(\frac{1}{R_1}+\frac{1}{R_2})(\frac{1}{C_1}+\frac{1}{C_2})
\label{corte_PBanda}
\end{equation}

Por otro lado, el desfasaje $\phi$ entre $V_s$ y $V_e$ tiene la forma

\begin{equation}
tan(\phi) = \frac{(\frac{\omega}{\omega_m})^2-1}{\omega (R_1+R_2)(C_1+C_2)}
\label{des_PBanda}
\end{equation}

%%%%%%%%%%%%%%%%%%%%%%%%%%%%%%%%%%%%%%%%%%%%%%%%%%%%%%%%%%%%%%%%%%%%%%%%%%%%%%%%%%%%%%%%%%%%%%%%%%%%%%%%%%%%%%%%%%%%%%%%%%%%%%%
% 3. DISPOSITIVO EXPERIMENTAL: armado del modelo, como se midio, consideraciones a la hora de medir.
%%%%%%%%%%%%%%%%%%%%%%%%%%%%%%%%%%%%%%%%%%%%%%%%%%%%%%%%%%%%%%%%%%%%%%%%%%%%%%%%%%%%%%%%%%%%%%%%%%%%%%%%%%%%%%%%%%%%%%%%%%%%%%%

\section{Desarrollo experimental}

Durante esta experiencia se utilizó como fuente un generador de funciones que se programó para que generara un diferencia de potencial que variara en el tiempo con la forma $\varepsilon = E_{0}cos(\omega t)$, donde $E_{0}$ es la amplitud maxima y en el informe se referira a ella simplemente como \textit{amplitud}. Este generador es capaz de emitir frecuencias con un error relativo del $0,01\%$ en un rango entre $1\mu Hz$ y $5MHz$ cuyo voltaje pico-pico tiene un error relativo del $1\%$ para el rango de voltaje utilizado ($2V-20V$). Además, se utilizó una resistencia variable por décadas cuyo error fue a priori desconocido. Usando un multimetro digital se midieron las resistencias utilizadas junto con su error, que era de la forma $\pm(1\%+2d)$ para el rango de resistencias utilizadas (mayores a $100\Omega$). La resistencia del capacitor resultó despreciable. También, se utilizó una inductancia fija $L = (1.000 \pm 0.002) H$ que poseía una resistencia interna (medida por el multímetro) $R_L = (294 \pm 3) \Omega$.
Finalmente, se utilizó un osciloscopio digital que en sus dos canales de entrada era capaz de medir diferencias de potencial entre las dos terminales que dispone en un rango de 2mV a 5V con un error relativo del $3\%$. A la hora de medir voltaje, fue necesario asegurarse que el cable a tierra del osciloscopio estuvera conectado al cable a tierra el generador de funciones. 



\subsection{Resonancia y Anti-Resonancia}
Durante la experiencia se estudió el comportamiento de circuitos RCL sometidos a corrientes con distintas frecuencias. En primer lugar se quiso estudiar el efecto de resonacia, por lo cual se construyó un circuito cerrado que constaba de la fuente $\varepsilon$, la resistencia variable por decadas fijada en un valor $R = (5.00 \pm 0,05)K\Omega$, la inductancia $L = (1.000 \pm 0.002)H$ la cual imponia una resistencia $R_{L}= (296\pm 3)\Omega$, y una capacitacia $C = (9.95 \pm 0.07)nF$, conectados en serie como muestra la \textbf{Figura \ref{fig:RCL-Res}}. Cabe destacar que, previamente a la construccion el dispositivo,  se utilizó el multimetro para asegurar la continuidad de todos los cables utilizados, y que esta misma no se viera comprometida por movimientos aleatorios, a fin de poder reducir una fuente de posibles incertezas.

\begin{figure}[h]
\centering
\includegraphics[scale=0.7]{Circuito-RCL-Resonante}
  \caption{Circuito RCL resonante con una fuente sinusoidal de amplitud (voltaje de entrada), capacitancia e inductancia fijas. Se obtuvo desfasaje y voltaje de salida en función de la frecuencia para dos resistencias distintas.}
  \label{fig:RCL-Res}
\end{figure}

Para medir la diferencia de tension se conectó, en paralelo, un canal del osciloscopio a la resistencia. Se utilizo una ficha T para conectar en paralelo a la fuente el segundo canal del osciloscopio, logrando de esta manera, que se desplegaran en la pantalla las dos señales al mismo tiempo y fuera posibile medir la diferencia de fase entre las señales. Además, tambien se utilizó la frecuencia de esa segunda señal como \textit{trigger externo} para asegurar una imagen estatica en la pantalla del osciloscopio. Se fijo una amplitud $E_{0} = (8,00 \pm 0,08)V$ y se fue variando la frecuencia. Para cada frecuencia se registró el desfasaje y la amplitud medida por el osciloscopio para la señal de salida. Una vez terminada la adquisicion de datos, se repitio el proceso con un cicuito con los mismos parametros, a excepcion de la resistencia, a la cual se le cambio el valor a $R = (500 \pm 5)$. Vale la pena aclarar que para este segundo circuito, la resistencia producida por la inductancia no era despreciable y fue tenida en cuenta.

Luego, se estudió el caso de la anti-resonancia, y para esto se diseñó un circuito RCL similar al anterior, con la salvedad que en este caso el capacitor se encontraba conectado en paralelo a la inductancia como ilustra la \textbf{Figura \ref{fig:RCL-ARes}}. De la misma forma que en el caso de resonancia, se realizaron dos juegos de mediciones y para ambas se utilizó una inductancia con un valor $L = (1.000 \pm0.002)H$ y una capacitacia $C = (9.95 \pm 0.07)nF$ mientras que la fuente se fijó en una amplitud $E_{0} = (8,00 \pm 0,08)V$ y la resistencia tuvo un valor $R = (5.00 \pm 0,05)K\Omega$ durante la primer medición, y en $R = (1.00 \pm 0,01)K\Omega$ durante la segunda. En este caso se consideraron despreciable las resistencias de $C$ y $L$.

\begin{figure}[h]
\centering
\includegraphics[scale=0.7]{Circuito-RCL-AntiResonante}
  \caption{Circuito RCL antiresonante con una fuente sinusoidal de amplitud (voltaje de entrada), capacitancia e inductancia fijas. Se obtuvo desfasaje y voltaje de salida en función de la frecuencia para dos resistencias distintas.}
  \label{fig:RCL-ARes}
\end{figure}

De manera analoga al metodo utilizado con el circuito resonante, el osciloscopio se conectó de forma paralela a la resistencia y a la fuente, y se tomaron nota de los desfasajes y amplitudes de la corriente de salida.


\subsection{Filtros}

El objetivo de esta parte fue experimentar con distintos tipos de filtros, para esto se construyó un circuito que constaba de una resistencia $R = (5.00 \pm 0.05)k\Omega$, una capacitancia $C = (10.00 \pm 0.03)nF$ y un generador de funciones con una amplitud $E_{0} = (9.4 \pm 0.2)V$, colocados en serie. Para un primer análisis, se utilizaron como terminales de salida los extremos del capacitor obteniendo así, un filtro Pasa-Bajos, como se muestra en la \textbf{Figura \ref{fig:RC-PB}}. Cabe destacar, que de la misma manera que se hizo durante los experimentos de resonancia, se aseguro la continuidad de todos los cables a utilizar.

\begin{figure}[h]
\centering
\includegraphics[scale=0.9]{Circuito-RC-Pasa-Bajos}
  \caption{Circuito RC con un osciloscopio conectado en paralelo a la fuente y al capacitor para medir desfasaje y tensión sobre este último para distintas frecuencias. Tanto la resistencia como la capacitancia son fijos}
  \label{fig:RC-PB}
\end{figure}

Antes de realizar las mediciones se calculó la frecuencia de corte del filtro para asegurar que se relevaran datos que correspondieran tanto a las señales que eran atenuadas como a las que no. Una vez tomado esto en consideración, se utilizó el mismo procedimiento de medición que se utilizo con el circuito resonante. 

Una vez finalizadas las mediciones sobre el dispositivo, se procedió a ver el caso del filtro Pasa-Altos; para lo cual se cambiaron las terminales de salida, colocándose en los extremos de la resistencia como ilustra la \textbf{Figura \ref{fig:RC-PA}}. No se realizaron variaciones en ninguno de los parámetros, pero si se cambió la posición de la descarga a tierra del generador de funciones para que coincidiera con la del osciloscopio.

\begin{figure}[h]
\centering
\includegraphics[scale=0.9]{Circuito-RC-Pasa-Altos}
  \caption{Circuito RC con un osciloscopio conectado en paralelo a la fuente y a la resistencia para medir desfasaje y tensión sobre esta última para distintas frecuencias. Tanto la resistencia como la capacitancia son fijos}
  \label{fig:RC-PA}
\end{figure}

El Procedimiento de medicion para este circuito, fue completamente análogo al realizado anteriormente con el circuito Pasa-Bajos. 

Finalmente se procedió a combinar ambos filtros para formar un Pasa-Banda como se puede ver en la \textbf{Figura \ref{fig:RC-PBD}}. En este caso si se cambiaron los parametros. Se fijarom ambas resistencias en un valor $R = (1.000 \pm 0.009)k\Omega$ y una capacitancia $C_{1} = (10.02 \pm 0.02)nF$ y una $C_{2} = (100.0 \pm 0.3)nF$. Cabe destacar que estos valores se eligieron de forma tal que la frecuencia de corte del pasabajos sea un orden de magnitud mayor que la del pasa altos y asi poder apreciar la campana de transferencia. 

\begin{figure}[h]
\centering
\includegraphics[scale=0.8]{Circuito-RC-Pasa-Banda}
  \caption{Circuito Pasa-Banda de amplitud fija cuyas resistencias tienen el mismo valor y la capacitancia $c_{1}$ es 10 veces menor que $c_{2}$. Para distintas frecuencias se midió el desfasaje y la tensión de salida con el osciloscopio.}
  \label{fig:RC-PBD}
\end{figure}

Una vez finalizada la construccion del dispositivo, se procedio a medir de la manera detallada anteriormente.


%%%%%%%%%%%%%%%%%%%%%%%%%%%%%%%%%%%%%%%%%%%%%%%%%%%%%%%%%%%%%%%%%%%%%%%%%%%%%%%%%%%%%%%%%%%%%%%%%%%%%%%%%%%%%%%%%%%%%%%%%%%%%%%%
% 4.DISCUSIÓN Y RESULTADOS: todo lo que se obtuvo y explicación. Graficos, tablas.
%%%%%%%%%%%%%%%%%%%%%%%%%%%%%%%%%%%%%%%%%%%%%%%%%%%%%%%%%%%%%%%%%%%%%%%%%%%%%%%%%%%%%%%%%%%%%%%%%%%%%%%%%%%%%%%%%%%%%%%%%%%%%%%%

\section{Resultados}


\subsection{Resonancia y Anti-resonancia}

Sabiendo que, sobre la resistencia, la diferencia de potencial $V_R$ es proporcional a la amplitud corriente $I_0$; se midieron estas difencia de potencial sobre la resistencia $R$ de cada circuito para poder ajustarlo con la ecuación \textbf{\eqref{eq:1}} en los circuitos resonantes y \textbf{\eqref{eq:3}} en los circuitos anti-resonantes.  

Puesto que para el estudio de los circuitos resonante y anti-resonantes se utilizaron una capacitancia $C= (9.95 \pm 0.07)nF$ y una inductancia $L= (1.000 \pm 0.002) H$, la valor teórica de la frecuencia angular de resonancia dado por \textbf{\eqref{eq:F_res}} es $\omega_{0}=(10030 \pm 150) Hz$ para todos los circuitos.

En primer lugar se hizo el análisis de resonancia. Para ello se utilizo el sistema que tiene en serie la inductancia $L$, la capacitancia $C$ y la resistencia $R$; por lo que, al conciderarse despreciable la resistencia de $C$, la resistencia total del circuito era $R_{total} = R+R_{L}$ donde $R_{L} = (296 \pm 3)$ es la resistencia que impone la inductancia.

De las mediciones obtenidas para los circuitos con resistencia $R=(5.00 \pm 0.05)K\Omega$ y $R=(500.0 \pm 0.2)\Omega$ se pudieron esbozar los gráficos que muestran las \textbf{Figuras \ref{subfig:RES_I.a}} y \textbf{\ref{subfig:RES_I.b}} respectivamente.

\begin{figure}[h]

\begin{subfigure}{0.5\textwidth}
\includegraphics[scale=0.3]{RLC_RES_5000_VvsF}
  \caption{Resonancia 5000 $\Omega$}
  \label{subfig:RES_I.a}
\end{subfigure}
\begin{subfigure}{0.5\textwidth}
\includegraphics[scale=0.32]{RLC_RES_500_VvsF}
  \caption{Resonancia 500 $\Omega$}
  \label{subfig:RES_I.b}
\end{subfigure}
  \caption{Circuitos RLC resonancia: $I_{ef}$}
  \label{fig:RES_I}
\end{figure}

Para el primer caso, para $R=(5.00 \pm 0.05)K\Omega$, se realizó un ajuste según \eqref{eq:1}. De manera que el coeficiente $R-Square = 0.99985$ garantizó la bondad de dicho ajuste, y utilizando los parámetros obtenidos se calcularon el máximo de potencial $V_{max}=(4.6 \pm 0.8)$ para una frecuencia $\omega_{res} = (10425 \pm 2500)Hz$, resultando entonces una frecuencia idéntica al valor teórico $\omega_0 = (10030 \pm 150)Hz$. A su vez se obtuvo un ancho de banda $\Delta\omega = (5590 \pm 1200) Hz$, y usando \textbf{\eqref{deltaomega}} se obtiene un valor teórico $\Delta\omega_{teo} = (5300 \pm 90)Hz$ teniendo entonces dos resultados indistinguibles al coincidir en un intervalo. 


Luego, para $R=(500 \pm 2)\Omega$, el coeficiente $R-Square = 0.99981$ garantizó la bondad del ajuste dado por \textbf{\eqref{eq:1}}. Por lo que se pudo calcular la frecuencia de resonancia $\omega_{res}= (10753 \pm 1000) Hz$ con un máximo de amplitud $V_{max}=(2.4 \pm 0.4) V$. Se puede ver entonces que la frecuencia de resonancia obtenida coincide con el resultado teórico $\omega_0 = (10030 \pm 150) Hz$. Ademas se calculó un ancho de banda $\Delta\omega = (697 \pm 200)Hz$, y utilizando \textbf{\eqref{deltaomega}} se obtuvo su valor teórico $\Delta\omega_{teo} = (729 \pm 6) Hz$ por lo cual, al coincidir en un intervalor, se concideran indistinguibles.

Por otro lado también se registraron los valores de la diferencia de fase entre la señal de entrada y la señal de salida, para obtener los gráficos que muestran las \textbf{Figuras \ref{subfig:RES_D.a}}, para $R=(5.00 \pm 0.05) K\Omega$, y \textbf{\ref{subfig:RES_D.b}} para $R=(500 \pm 2)\Omega$.

\begin{figure}[h]
\begin{subfigure}{0.5\textwidth}
\includegraphics[scale=0.34]{RLC_RES_5000_Tan(fase)vsF}
  \caption{Resonancia 5000 $\Omega$}
  \label{subfig:RES_D.a}
\end{subfigure}
\begin{subfigure}{0.5\textwidth}
\includegraphics[scale=0.34]{RLC_RES_500_Tan(fase)vsF}
  \caption{Resonancia 500 $\Omega$}
  \label{subfig:RES_D.b}
\end{subfigure}
  \caption{Circuitos RLC resonancia: \textit{Desfasaje}}
  \label{fig:RES_D}
\end{figure}

Por lo que, al obtener un $R-Square = 0.99885$ se garantizó la bondad del ajuste de la \textbf{Figura \ref{subfig:RES_D.a}} dado por \textbf{\eqref{eq:2}}. Entonces, despejando de los parametros, se encontró que las dos señales están en fase cuando $\omega = ( 9978 \pm 90) Hz$ coincidiendo con la frecuencia de resonancia $\omega_{0} = (10030 \pm 150) Hz$.

De la misma manera, el ajuste de \textbf{\ref{subfig:RES_D.b}} por la ecuación \eqref{eq:2} dio un $R-Square = 0.99091$, nuevamente garantizando la bondad del ajuste. Y se obtuvo entonces que las dos señales estaban en fase cuando $\omega = (10036 \pm 220) Hz$ siendo indistinguible de la frecuencia de resonancia $\omega_{0} = (10030 \pm 150) Hz$.

\bigskip

De manera análoga se hizo el análisis de anti-resonancia. En este caso se consideraron despreciables las resistencia impuestas por $C$ como por $L$.

A partir de las mediciones obtenidas para los circuitos con resistencia $R=(5.00 \pm 0.05) K\Omega$ y $R=(1.00 \pm 0.01)K\Omega$ se realizaron los gráficos que muestran las \textbf{Figuras \ref{subfig:ARES_I.a}} y \textbf{\ref{subfig:ARES_I.b}} respectivamente.

\begin{figure}[h]

\begin{subfigure}{0.5\textwidth}
\includegraphics[scale=0.31]{RLC_ARES_5000_VvsF}
  \caption{Anti-resonancia 5000 $\Omega$}
  \label{subfig:ARES_I.a}
\end{subfigure}
\begin{subfigure}{0.5\textwidth}
\includegraphics[scale=0.3]{RLC_ARES_1000_VvsF}
  \caption{Anti-resonancia 1000 $\Omega$}
  \label{subfig:ARES_I.b}
\end{subfigure}
  \caption{Circuitos RLC anti-resonancia: $I_{ef}$}
  \label{fig:ARES_I}
\end{figure}

Para el circuito con $R=(5.00 \pm 0.05)K\Omega$ del ajuste la \textbf{Figura \ref{subfig:ARES_I.a}} según \eqref{eq:3} se obtuvo un coeficiente $R-Square = 0.99903$ garantizando que el ajuste es el correcto. Por lo cual se pudo obtener un mínimo de potencial $V_{min} = (0.0 \pm 0.1)V$ en la frecuencia $\omega_{anti-res} = (9991 \pm 300) Hz$, teniendo entonces que la frecuencia de anti-resonancia es idéntica 
a su valor teórico $\omega_0 = (10030 \pm 150) Hz$. Tambien se calculó un ancho de banda $\Delta\omega = (20708 \pm 2500) Hz$ el cual coincide con el valor teorico, obtenido a partir de \textbf{\eqref{deltaomega2}}, $\Delta\omega_{teo}=(20100 \pm 350)$.

De igual manera, para $R=(1.00 \pm 0.01)K\Omega$, se ajustó la \textbf{Figura \ref{subfig:ARES_I.b}} por \textbf{\eqref{eq:3}}, confirmando la bondad de este ajuste con un coeficiente $R-Square = 0.99978$. En este caso para una frecuencia $\omega_{anti-res} =(10065 \pm 80)Hz$ se obtuvo un mimimo de potencia $V_{min} = (0.0 \pm 0.2)V$, la cual coincide con el valor teórico $\omega_0 = (10030 \pm 150) Hz$. Y el ancho de banda obtenido es $\Delta\omega = (101285 \pm 3000)Hz$, mientras que el valor teórico dado por \textbf{\eqref{deltaomega2}} es $\Delta\omega = (10052 \pm 1500)Hz$, y puesto que coinciden en un intervalos se los considera indistinguibles.

\bigskip

Del registro de la diferencia de fase entre las señales de entrada y salida se obtuvieron los graficos que muestras las \textbf{Figuras \ref{subfig:ARES_D.a}}, para $R=(5.00 \pm 0.05)K\Omega$, y \textbf{\ref{subfig:ARES_D.b}} para $R=(1.00 \pm 0.01)K\Omega$.

\begin{figure}[h]

\begin{subfigure}{0.5\textwidth}
\includegraphics[scale=0.32]{RLC_ARES_5000_Tan(fase)vsF}
  \caption{Anti-resonancia 5000 $\Omega$}
  \label{subfig:ARES_D.a}
\end{subfigure}
\begin{subfigure}{0.5\textwidth}
\includegraphics[scale=0.32]{RLC_ARES_1000_Tan(fase)vsF}
  \caption{Anti-resonancia 1000 $\Omega$}
  \label{subfig:ARES_D.b}
\end{subfigure}
  \caption{Circuitos RLC anti-resonancia: \textit{Desfasaje}}
  \label{fig:ARES_D}
\end{figure}

Como se puede ver en ambas figuras, para frecuencia mayores a la frecuencia de resonancia el ajuste propuesto por \eqref{eq:4} fue efectivamente la correcta al obtener un coeficiente $R-Square=0.99072$ para $R=(5.00 \pm 0.05)K\Omega$ y $R-Square=0.99743$ para $R=(1.00 \pm 0.01)K\Omega$. Sin embargo para frecuencias menores no se pudo realizar un ajuste correcto, pues para $R=(1.00 \pm 0.01)K\Omega$ no se obtuvieron suficientes mediciones y para $R=(5.00 \pm 0.05)K\Omega$ se obtuvo un coeficiente $R=(0.79899)$ el cual se consideró insuficiente para garantizar la bondad del ajuste.

\newpage

\subsection{Filtros}

De las mediciones realizadas para el estudio del filtro Pasa-Bajo se obtuvieron los gráficos que muestran la \textbf{Figura \ref{subfig:PB.a}} y \textbf{\ref{subfig:PB.b}}, transferencia y desfasaje respectivamente.

\begin{figure}[h]
\begin{subfigure}{0.5\textwidth}
\includegraphics[scale=0.34]{PB_TRANSFERENCIA}
  \caption{Circuito pasa-bajos: Transferencia vs frecuencia }
  \label{subfig:PB.a}
\end{subfigure}
\begin{subfigure}{0.5\textwidth}
\includegraphics[scale=0.34]{PB_DESFASAJE}
  \caption{Circuito pasa-bajo: Desfasaje vs frecuencia}
  \label{subfig:PB.b}
\end{subfigure}
  \caption{Resultados obtenidos a partir del circuito pasa-bajos}
  \label{fig:PB}
\end{figure}


De ambos gráficos se obtuvo un ajuste satisfactorio. En primer lugar, del ajuste de la \textbf{Figura \ref{subfig:PB.a}} dado por \eqref{trans_PA} se obtuve un $R-Square = 0.99944$ el cual garantizaba la bondad del ajuste. A partir del cual se obtuvo una frecuencia de corte $\omega_C = (20151 \pm 140) hz$, mientras que el valor teórico es $\omega_c = (20000 \pm 250)$, por lo que al coincidir en un intervalo se los considera indistinguibles. Luego, del ajuste de \textbf{Figura \ref{subfig:PB.b}} se obtuvo un $R-Square=0.99708$, asegurando entonces que el ajuste dado por \eqref{des_PBaj} es el correcto, garantizando la relación entre fase y frecuencia para este circuito pasa-bajos.


\bigskip

Por otro lado, a partir de las mediciones hechas para el filtro Pasa-Altso se pudieron obtener los gráficos que muestran la \textbf{Figura \ref{subfig:PA.a}} y \textbf{\ref{subfig:PA.b}}, transferencia y desfasaje respectivamente.

\begin{figure}[h]
\begin{subfigure}{0.5\textwidth}
\includegraphics[scale=0.34]{PA_TRANSFERENCIA}
  \caption{Circuito pasa-altos: Transferencia vs frecuencia }
  \label{subfig:PA.a}
\end{subfigure}
\begin{subfigure}{0.5\textwidth}
\includegraphics[scale=0.34]{PA_DESFASAJE}
  \caption{Circuito pasa-altos: Desfasaje vs frecuencia}
  \label{subfig:PA.b}
\end{subfigure}
  \caption{Resultados obtenidos a partir del circuito pasa-altos}
  \label{fig:PA}
\end{figure}

Nuevamente, los ajustes resultaron satisfactorios. Del gráfico de la transferencia en la \textbf{Figura (\ref{subfig:PA.a})} se realizó un ajuste según \eqref{trans_PA}, obteniendo un $R-Square = 0.99953$. Con lo cual se pudó obtener una frecuencia de corte $\omega_C = (20344 \pm 125)Hz$, que al igual que el caso anterior se lo resulta idéntico al valor teórico $\omega_c = (20000 \pm 250)Hz$. Y por otro lado, del gráfico del desfasaje \textbf{Figura (\ref{subfig:PA.b})} se ajustó correctamente utilizando la ecuación \eqref{des_PBaj} al obtener un $R-Square = 0.99832$, garantizando la relación entre la fase y la frecuencia para este circuito pasa-altos.

\bigskip

Por ultimo, de las mediciones obtenidas del circuito Pasa-Bandas se realizaron los gráficos que muestran la \textbf{Figura \ref{subfig:PBAN.a}} y \textbf{\ref{subfig:PBAN.b}}, transferencia y desfasaje respectivamente.

\begin{figure}[h]
\begin{subfigure}{0.5\textwidth}
\includegraphics[scale=0.34]{PBANDA_TRANSFERENCIA}
  \caption{Circuito pasa-bandas: Transferencia vs frecuencia }
  \label{subfig:PBAN.a}
\end{subfigure}
\begin{subfigure}{0.5\textwidth}
\includegraphics[scale=0.34]{PBANDA_DESFASAJE}
  \caption{Circuito pasa-bandas: Desfasaje vs frecuencia}
  \label{subfig:PBAN.b}
\end{subfigure}
  \caption{Resultados obtenidos a partir del circuito pasa-banda}
  \label{fig:PBANDA}
\end{figure}

Entonces, el gráfico de la \textbf{Figura \ref{subfig:PBAN.a}} se lo ajustó usando la ecuación \eqref{trans_PBanda}. Al obtener un $R-Square= 0.99988$ se garantizó dicho ajuste, y utilizando los parámetros obtenidos se despejaron unas frecuencias de corte $\omega_1 = (4491 \pm 1200)Hz$ y $\omega_2 = (216970 \pm 1200)Hz$, mientras que al utilizar \eqref{corte_PBanda} se obtienen los valor teóricos $\omega_1 = (4455 \pm 250) Hz$ y $\omega_2 = (224455 \pm 250) Hz$. Teniendo entonces que las frecuencias $\omega_1$ son identicas al coincidir en un intervalo; sin embargo esto no sucede para las $\omega_2$.
Por otro lado, se ajustó el gráfico de la \textbf{Figura \ref{subfig:PBAN.b}} usando \eqref{des_PBanda}. Al obtener un $R-Square = 0.9996$ se considero correcto el ajuste, por lo cual se pudo garantizar la relación entre fase y frecuencia para el circuito pasa-banda.

\label{sec:discusion}



%%%%%%%%%%%%%%%%%%%%%%%%%%%%%%%%%%%%%%%%%%%%%%%%%%%%%%%%%%%%%%%%%%%%%%%%%%%%%%%%%%%%%%%%%%%%%%%%%%%%%%%%%%%%%%%%%%%%%%%%%%%%%%%%
%	CONCLUSIONES
%%%%%%%%%%%%%%%%%%%%%%%%%%%%%%%%%%%%%%%%%%%%%%%%%%%%%%%%%%%%%%%%%%%%%%%%%%%%%%%%%%%%%%%%%%%%%%%%%%%%%%%%%%%%%%%%%%%%%%%%%%%%%%%%

\section{Conclusiones}

En primer lugar, los resultados obtenidos del estudio las frecuencias de resonancia en el circuito resonante y anti-resonante fueron satisfactorios. A partir de los ajustes realizados en los correspondientes circuitos, se pudieron calcular las frecuencias de resonancia las cuales coincidían con el valor esperado $\omega_0 = (10030 \pm 150)hz$. Se puede ver esto en el caso de resonancia donde para $R=(5.00 \pm 0.05)K\Omega$ se obtuvo $\omega_res = (10425 \pm 2500)Hz$, para $R= (500 \pm 2)$  se obtuvo $\omega_res = (10753 \pm 1000)Hz$; mientras que en el caso de anti-resonancia para $R=(5.00 \pm 0.05)K\Omega$ se obtuvo $\omega_{anti-res} = (9991 \pm 300)Hz$ y para $R=(1.00 \pm 0.01)K\Omega$ se obtuvo $\omega_{anti-res} = (10065 \pm 80)Hz$. Teniendo entonces que los valores obtenidos coinciden en un intervalo, por lo tanto se consideran identicos al valor teorico. 

Por otro lado, para el circuito resonante se pudo comprobar la relación entre fase y frecuencia propuesta por \eqref{des_PBaj}; por lo cual a partir de los correspondientes gráficos se observó que la frecuencia en la que ambas señales se ponen en fase es $\omega = (9978 \pm 90)$ para $R=(5.00 \pm 0.05)K\Omega$, y $\omega = (10036 \pm 220)$ para $R=(500 \pm 2)\Omega$. Y al coincidir cada frecuencia de fase con su correspondiente frecuencia de resonancia, se pudo verificar que en los circuitos resonantes la señal de entrada y salida están en fase para la frecuencia de resonancia. Sin embargo, no fue asi para el circuito anti-resonante, donde no se pudieron realizar los ajuste correspondientes para frecuencias menores a la de resonancia. Esto puede deberse que en el marco teórico se considera despreciable la resistencia que impone la inductancia $L$, resistencia que puede considerarse despreciable para frecuencias altas pero no es así para frecuencias bajas; de modo que cambiaría el valor de la impedancia afectando directamente el valor de la fase. Por otro lado, en la \textbf{Figura \ref{subfig:ARES_D.a}} se puede observar una clara dispersión de los valores obtenidos alrededor del ajuste propuesto lo que indica un posible error de medición, y en la \textbf{Figura \ref{subfig:ARES_D.b}} al tener pocas mediciones no se pudo obtener una tendencia clara por lo que no se pudo realizar un ajuste correcto; entonces estos factores pueden también responder a lo inconcluso de estos resultados

Del estudio de los filtros pasa-bajos y pasa-altos se obtuvieron, también, resultados satisfactorios. En ambos casos se obtuvieron frecuencias de corte $\omega_C= (20151 \pm 140) Hz$ en el filtro pasa-bajos, y $\omega_C= (20344\pm125)Hz$; las cuales coinciden con la frecuencia de corte teorica $\omega_c = (20000 \pm 250) Hz$, siendo entonces idénticas al coincidir en un intervalo. De la misma manera, los ajustes realizados para los desfasajes fueron los correctos, por lo que se pudieron corroborar las correspondientes relaciones entre fase y frecuencia. Para el caso del filtro pasa-bajos, en la \textbf{Figura \ref{subfig:PB.b}} se puede ver que para frecuencias muchos mayores de la frecuencia de corte la fase tiene a $\frac{\pi}{2}$, teniendo entonces que la señal de entrada y salida están en contra-fase;  y para el filtro pasa-altos en la \textbf{Figura \ref{subfig:PA.b}} se observa que para frecuencias mucho mayores que la frecuencia de corte la fase tiende a cero, por lo que la señal de entrada y salida están en fase.

Por ultimo, a pesar de que el ajuste realizado para el filtro pasa-banda fueron correctos, de las frecuencias de corte obtenidas del mismo solo resultaron identicos los valores $\omega_{C-1} = (4491 \pm 1200)Hz$ con su equivalente teórico $\omega_1 = (4455 \ 250)Hz$, mientras que $\omega_{C-2} = (216870 \pm 1200)Hz$ no coincidió con $\omega_2 = (224455 \pm 250)hz$. Aun asi, el ajuste hecho sobre la \textbf{Figura \ref{subfig:PBAN.b}} fue el correcto, garantizando entonces la relación entre fase y frecuencia para el filtro pasa-banda. Entonces se puede observar, que para frecuencias mucho mas chicas que $\omega_1$ y muchos mas grandes que $\omega_2$ la fase tiene a $\frac{-\pi}{2}$ y $\frac{\pi}{2}$ respectivamente, teniendo entonces que para dichas frecuencias la señal de entrada y salida estan en contra-fase.

\label{sec:conclusiones}




%%%%%%%%%%%%%%%%%%%%%%%%%%%%%%%%%%%%%%%%%%%%%%%%%%%%%%%%%%%%%%%%%%%%%%%%%%%%%%%%%%%%%%%%%%%%%%%%%%%%%%%%%%%%%%%%%%%%%%%%%%%%%%%%%
%	APÉNDICE: esas cosas extras que simplemente no tuvieron lo suficiente como para ganarse una sección propia.
%%%%%%%%%%%%%%%%%%%%%%%%%%%%%%%%%%%%%%%%%%%%%%%%%%%%%%%%%%%%%%%%%%%%%%%%%%%%%%%%%%%%%%%%%%%%%%%%%%%%%%%%%%%%%%%%%%%%%%%%%%%%%%%%%



%%%%%%%%%%%%%%%%%%%%%%%%%%%%%%%%%%%%%%%%%%%%%%%%%%%%%%%%%%%%%%%%%%%%%%%%%%%%%%%%%%%%%%%%%%%%%%%%%%%%%%%%%%%%%%%%%%%%%%%%%%%%%%%%%
%	REFERENCIAS: libros, libros, libros.
%%%%%%%%%%%%%%%%%%%%%%%%%%%%%%%%%%%%%%%%%%%%%%%%%%%%%%%%%%%%%%%%%%%%%%%%%%%%%%%%%%%%%%%%%%%%%%%%%%%%%%%%%%%%%%%%%%%%%%%%%%%%%%%%%

%Ejemplo:
%\begin{thebibliography}{1}
 %\bibitem{Berkeley} Frank S. Crawford, \textit{Berkeley physics course 3: Ondas}, 1994, Editorial Reverte S.A.
%\end{thebibliography}
%Para citar: blablabla \cite{Baird}
 
\end{document}





