\documentclass[11pt,a4paper]{article}

\usepackage[utf8]{inputenc}
\usepackage[spanish]{babel}
\usepackage{amsmath}
\usepackage{amsfonts}
\usepackage{amssymb}
\usepackage{makeidx}
\usepackage{graphicx}
\usepackage{lmodern}
\usepackage{kpfonts}
\usepackage{wrapfig}
\usepackage{caption}
\usepackage{subcaption}
\usepackage{booktabs}
\usepackage[nottoc,numbib]{tocbibind} %agrega la bibliografia al índice.
\usepackage[font={small,it}]{caption}
%\usepackage{fourier}
\usepackage[left=2cm,right=2cm,top=2cm,bottom=2cm,headheight=13.6pt]{geometry}
\usepackage{fancyhdr}
\usepackage{multirow}
\pagestyle{fancy}


%Para los gráficos en general, con las tablas...¡Ja!, arreglate.
%\begin{figure}[h!]
%\centering
%\includegraphics[width=0.7\textwidth]{} %nombre de la imagen, incluirla en el mismo directorio que este archivo.
%\caption*{} %rótulo, el asterico elimina la numeración automática. 
%\label{fig:} % para luego referirse con \ref{fig:}
%\end{figure}


\begin{document}


%%%%%%%%%%%%%%%%%%%%%%%%%%%%%%%%%%%%%%%%%%%%%%%%%%%%%%%%%%%%%%%%%%%%%%%%%%%%%%%%%%%%%%%%%%%%%%%%%%%%%%%%%%%%%%%%%%%%%%%%%%%%%%%%%
% 	TÍTULO
%%%%%%%%%%%%%%%%%%%%%%%%%%%%%%%%%%%%%%%%%%%%%%%%%%%%%%%%%%%%%%%%%%%%%%%%%%%%%%%%%%%%%%%%%%%%%%%%%%%%%%%%%%%%%%%%%%%%%%%%%%%%%%%%%

%%%%%%%%%%%%%%%%%%%%%%%%%%%%%%%%%%%%%%%%%
% University Assignment Title Page 
% LaTeX Template
% Version 1.0 (27/12/12)
%
% This template has been downloaded from:
% http://www.LaTeXTemplates.com
%
% Original author:
% WikiBooks (http://en.wikibooks.org/wiki/LaTeX/Title_Creation)
%
% License:
% CC BY-NC-SA 3.0 (http://creativecommons.org/licenses/by-nc-sa/3.0/)
% 
% Instructions for using this template:
% This title page is capable of being compiled as is. This is not useful for 
% including it in another document. To do this, you have two options: 
%
% 1) Copy/paste everything between \begin{document} and \end{document} 
% starting at \begin{titlepage} and paste this into another LaTeX file where you 
% want your title page.
% OR
% 2) Remove everything outside the \begin{titlepage} and \end{titlepage} and 
% move this file to the same directory as the LaTeX file you wish to add it to. 
% Then add \input{./title_page_1.tex} to your LaTeX file where you want your
% title page.
%
%%%%%%%%%%%%%%%%%%%%%%%%%%%%%%%%%%%%%%%%%

%----------------------------------------------------------------------------------------
%	PACKAGES AND OTHER DOCUMENT CONFIGURATIONS
%----------------------------------------------------------------------------------------

%\documentclass[12pt]{article}
%\usepackage[utf8]{inputenc}
%\usepackage[spanish]{babel}
%\begin{document}

\begin{titlepage}

\newcommand{\HRule}{\rule{\linewidth}{0.5mm}} % Defines a new command for the horizontal lines, change thickness here

\center % Center everything on the page
 
%----------------------------------------------------------------------------------------
%	HEADING SECTIONS
%----------------------------------------------------------------------------------------

\textsc{\Huge Universidad de Buenos Aires}\\[0.5cm]
\textsc{\LARGE Facultad de Ciencias Exactas y Naturales}\\[0.5cm] % Name of your university/college
\textsc{\Large Departamento de Física}\\[0.25cm] % Major heading such as course name

\begin{figure}[h]
  \centering
  \includegraphics[scale=0.15]{Logo_DF}
  \\[0.5cm]
\end{figure}

\textsc{\large Laboratorio 3}\\[0.25cm] % Minor heading such as course title

%----------------------------------------------------------------------------------------
%	TITLE SECTION
%----------------------------------------------------------------------------------------

\HRule \\[0.4cm]
{ \huge \bfseries Estudio de resonancia y antiresonancia en circuitos RLC, y filtros en circuitos RC utilizando señales sinusoidales }\\[0.2cm] % Title of your document
\HRule \\[1cm]
 
%----------------------------------------------------------------------------------------
%	AUTHOR SECTION
%----------------------------------------------------------------------------------------

\begin{minipage}{0.4\textwidth}
\begin{center} \large
\emph{Autores:}\\
\textsc{Andreu}, Gonzalo\\ % Your name
\textsc{Malpartida}, Bryan\\ % Your name
\textsc{Pugliese}, Facundo\\ % Your name


\end{center}
\end{minipage}
~ \\[1.25cm]
%\begin{minipage}{0.4\textwidth}
%\begin{flushright} \large
%\emph{Supervisor:} \\
%Dr. James \textsc{Smith} % Supervisor's Name
%\end{flushright}
%\end{minipage}\\[4cm]

% If you don't want a supervisor, uncomment the two lines below and remove the section above
%\Large \emph{Author:}\\
%John \textsc{Smith}\\[3cm] % Your name

%----------------------------------------------------------------------------------------
%	DATE SECTION
%----------------------------------------------------------------------------------------

%\vspace{\fill}


{\large 24 de Febrero de 2016}\\[1.75cm] % Date, change the \today to a set date if you want to be precise

%----------------------------------------------------------------------------------------
%	SUMMARY SECTION: No más de 15 renglones, no te zarpes
%----------------------------------------------------------------------------------------

\begin{center}
\large{\textbf{Resumen}}

\small{El objetivo del siguiente trabajo fue comprobar empíricamente los fenómenos de resonancia y anti-resonancia presente en circuitos eléctricos RLC, así como también el estudio de filtros pasa-bajos y filtros pasa-alto en circuitos RC y filtros pasa-banda en un circuito compuesto por dos RC conectados en paralelo. Para ello se observó, utilizando un osciloscopio digital, el comportamiento de la transferencia y desfasaje entre la señal de salida y una señal de entrada sinusoidal entregada por un fuente de alimentación programable.} % ACA VA EL RESUMEN


\end{center}


%----------------------------------------------------------------------------------------
%	LOGO SECTION
%----------------------------------------------------------------------------------------

%\includegraphics{Logo}\\[1cm] % Include a department/university logo - this will require the graphicx package
 
%----------------------------------------------------------------------------------------

\vfill % Fill the rest of the page with whitespace

\end{titlepage}
%\end{document} %incluir en el mismo directorio que este archivo. Equivalente a un copiar-pegar, nada de andar diciendo \begin{document} en la portada. Dejar el nombre de Caratula a la caratula.

%%%%%%%%%%%%%%%%%%%%%%%%%%%%%%%%%%%%%%%%%%%%%%%%%%%%%%%%%%%%%%%%%%%%%%%%%%%%%%%%%%%%%%%%%%%%%%%%%%%%%%%%%%%%%%%%%%%%%%%%%%%%%%%%%
% 	ENCABEZADO Y PIE DE PÁGINA.
%%%%%%%%%%%%%%%%%%%%%%%%%%%%%%%%%%%%%%%%%%%%%%%%%%%%%%%%%%%%%%%%%%%%%%%%%%%%%%%%%%%%%%%%%%%%%%%%%%%%%%%%%%%%%%%%%%%%%%%%%%%%%%%%%

\lhead{}
\chead{}
\rhead{Laboratorio 3}
\lfoot{}
\cfoot{}
\rfoot{\thepage}
\renewcommand{\headrulewidth}{1pt}
\renewcommand{\footrulewidth}{1pt}


%%%%%%%%%%%%%%%%%%%%%%%%%%%%%%%%%%%%%%%%%%%%%%%%%%%%%%%%%%%%%%%%%%%%%%%%%%%%%%%%%%%%%%%%%%%%%%%%%%%%%%%%%%%%%%%%%%%%%%%%%%%%%%%
% Página en blanco. Cita, agradecimiento, dedicación, lo que sea pero que sea algo.
%%%%%%%%%%%%%%%%%%%%%%%%%%%%%%%%%%%%%%%%%%%%%%%%%%%%%%%%%%%%%%%%%%%%%%%%%%%%%%%%%%%%%%%%%%%%%%%%%%%%%%%%%%%%%%%%%%%%%%%%%%%%%%%


%%%%%%%%%%%%%%%%%%%%%%%%%%%%%%%%%%%%%%%%%%%%%%%%%%%%%%%%%%%%%%%%%%%%%%%%%%%%%%%%%%%%%%%%%%%%%%%%%%%%%%%%%%%%%%%%%%%%%%%%%%%%%%%%%
% 	ÍNDICE
%%%%%%%%%%%%%%%%%%%%%%%%%%%%%%%%%%%%%%%%%%%%%%%%%%%%%%%%%%%%%%%%%%%%%%%%%%%%%%%%%%%%%%%%%%%%%%%%%%%%%%%%%%%%%%%%%%%%%%%%%%%%%%%%%

%\tableofcontents %compilar dos o tres veces para verlo bien. ¡Todo un índice en unas cuantas letras!
%\newpage

%%%%%%%%%%%%%%%%%%%%%%%%%%%%%%%%%%%%%%%%%%%%%%%%%%%%%%%%%%%%%%%%%%%%%%%%%%%%%%%%%%%%%%%%%%%%%%%%%%%%%%%%%%%%%%%%%%%%%%%%%%%%%%%
% 1. RESUMEN
%%%%%%%%%%%%%%%%%%%%%%%%%%%%%%%%%%%%%%%%%%%%%%%%%%%%%%%%%%%%%%%%%%%%%%%%%%%%%%%%%%%%%%%%%%%%%%%%%%%%%%%%%%%%%%%%%%%%%%%%%%%%%%%

%\section{Resumen}
%\label{sec:resumen}



%%%%%%%%%%%%%%%%%%%%%%%%%%%%%%%%%%%%%%%%%%%%%%%%%%%%%%%%%%%%%%%%%%%%%%%%%%%%%%%%%%%%%%%%%%%%%%%%%%%%%%%%%%%%%%%%%%%%%%%%%%%%%%%
% 2. INTRODUCCIÓN: ecuaciones aquí, luego se las cita.
%%%%%%%%%%%%%%%%%%%%%%%%%%%%%%%%%%%%%%%%%%%%%%%%%%%%%%%%%%%%%%%%%%%%%%%%%%%%%%%%%%%%%%%%%%%%%%%%%%%%%%%%%%%%%%%%%%%%%%%%%%%%%%%

\section{Introducción}\label{sec:intro}

Para un circuito RLC con una fuente de corriente alterna de la forma $\varepsilon = E_{0}cos(\omega t)$ con una resistencia $R$, una inductancia $L$ y una capacitancia $C$ en serie la ecuacion diferencial que rige la evolución del sistema está dada por $\frac{d^2q}{dt^2}L+\frac{dq}{dt}R+\frac{q}{C} = \varepsilon = E_{0}cos(\omega t)$ donde $q = q(t)$ representa la carga del capacitor $C$. En particular, es posible buscar la solución compleja de la ecuación $\frac{d^2Q}{dt^2}L+\frac{dQ}{dt}R+\frac{Q}{C} = \varepsilon = E_{0}e^{i\omega t}$ donde Q es tal que tal que $Re(Q) = q$. Para este sistema, recordando que $\mathbb{I}(t) = \frac{dQ}{dt}$, la solución resulta $\mathbb{I}(t) = \frac{E_0e^{i\omega t}}{Z}$ donde $Z = R +i(\omega L-\frac{1}{\omega C})$

\begin{equation}
V_{ef} = \frac{I_{ef}}{\sqrt{R^2 + (\omega.L - \frac{1}{\omega.C})^2}}
\label{eq:1}
\end{equation}

\begin{equation}
tan(\phi) = \frac{L\omega}{R} - \frac{1}{RC\omega} 
\label{eq:2}
\end{equation}

\begin{equation}
\omega_{0}= \frac{1}{LC}
\label{eq:F_res}
\end{equation}

\begin{equation}
\Delta\omega = \frac{R}{L}
\label{deltaomega}
\end{equation}

Antiresonancia

\begin{equation}
V_{ef} = \frac{I_{ef}}{\sqrt{R^2 + \frac{1}{(\omega.C - \frac{1}{\omega.L})^2}}}
\label{eq:3}
\end{equation}

\begin{equation}
tan(\phi) = \frac{L}{R}\frac{\omega}{(1- LC\omega^2)} 
\label{eq:4}
\end{equation}

\begin{equation}
\Delta\omega = \frac{1}{RC}
\label{deltaomega2}
\end{equation}

Como puede verse, los circuitos con inductancias $L$ y capacitancias $C$ tienen tal interacción con las frecuencias de una fuente de corriente alterna que es posible combinarlos para que atenuen determinadas frecuencias dejando pasar otras. Esto es lo que se denomina como \texit{filtro}. Los filtros más básicos que existen son el Pasa-Altos y el Pasa-Bajos, que no son más que circuitos RC identicos. Lo que varía es la ubicación de las terminales con las que se conectan al resto del circuito. 

Usando una fuente de corriente alterna $\varepsilon(t) = E_{0}e^{i\omega t}$, para un circuito RC de resistencia $R$ y capacitancia $C$ la impendancia es de la forma $Z = R+i\omega.C$. Por lo tanto, la corriente resulta $I(t) = \frac{\varepsilon(t)}{Z}$. A la constante $E_0$ se la llama \textit{voltaje de entrada} mientras que

%%%%%%%%%%%%%%%%%%%%%%%%%%%%%%%%%%%%%%%%%%%%%%%%%%%%%%%%%%%%%%%%%%%%%%%%%%%%%%%%%%%%%%%%%%%%%%%%%%%%%%%%%%%%%%%%%%%%%%%%%%%%%%%
% 3. DISPOSITIVO EXPERIMENTAL: armado del modelo, como se midio, consideraciones a la hora de medir.
%%%%%%%%%%%%%%%%%%%%%%%%%%%%%%%%%%%%%%%%%%%%%%%%%%%%%%%%%%%%%%%%%%%%%%%%%%%%%%%%%%%%%%%%%%%%%%%%%%%%%%%%%%%%%%%%%%%%%%%%%%%%%%%

\section{Desarrollo experimental}

Durante esta experiencia se utilizó como fuente un generador de funciones que se programó para que generara un diferencia de potencial que variara en el tiempo con la forma $\varepsilon = E_{0}cos(\omega t)$, donde $E_{0}$ es la amplitud maxima y en el informe se referira a ella simplemente como \textit{amplitud}. Este generador es capaz de emitir frecuencias con un error relativo del $0,01\%$ en un rango entre $1\mu Hz$ y $5MHz$ cuyo voltaje pico-pico tiene un error relativo del $1\%$ para el rango de voltaje utilizado ($2V-20V$). Además, se utilizó una capacitancia fija $C = (100,0 \pm 0,2)nF$ y una resistencia variable por décadas cuyo error fue a priori desconocido. Usando un multimetro digital se midieron las resistencias utilizadas junto con su error, que era de la forma $\pm(1\%+2d)$ para el rango de resistencias utilizadas (mayores a $100\Omega$). La resistencia del capacitor resultó despreciable. También, se utilizó una inductancia fija $L = (1.000 \pm 0.002) H$ que poseía una resistencia interna (medida por el multímetro) $R_L = (294 \pm 3) \Omega$.
Finalmente, se utilizó un osciloscopio digital que en sus dos canales de entrada era capaz de medir diferencias de potencial entre las dos terminales que dispone en un rango de 2mV a 5V con un error relativo del $3\%$. A la hora de medir voltaje, fue necesario asegurarse que el cable a tierra del osciloscopio estuvera conectado al cable a tierra el generador de funciones. 



\subsection{Resonancia y Anti-Resonancia}
Durante la experiencia se estudió el comportamiento de circuitos RCL sometidos a corrientes con distintas frecuencias. En primer lugar se quiso estudiar el efecto de resonacia, por lo cual se construyó un circuito cerrado que constaba de la fuente $\varepsilon$, la resistencia variable por decadas fijada en un valor $R = (5 \pm 0,05)K\Omega$, la inductancia $L = (1 \pm 0.002)H$ con una resistencia $R_{L}= (296\pm 3)\Omega$ que fue despreciada frente al valor de la mencionada anteriormente, y una capacitacia $C = (9.95 \pm 0.07)nF$, conectados en serie como muestra la \textbf{Figura \ref{fig:RCL-Res}}. Cabe destacar que, previamente a la construccion el dispositivo,  se utilizó el multimetro para asegurar la continuidad de todos los cables utilizados, y que esta misma no se viera comprometida por movimientos aleatorios, a fin de poder reducir una fuente de posibles incertezas.

\begin{figure}[h]
\centering
\includegraphics[scale=0.7]{Circuito-RCL-Resonante}
  \caption{Circuito RCL resonante con una fuente sinusoidal de amplitud (voltaje de entrada), capacitancia e inductancia fijas. Se obtuvo desfasaje y voltaje de salida en función de la frecuencia para dos resistencias distintas.}
  \label{fig:RCL-Res}
\end{figure}

Para medir la diferencia de tension se conectó, en paralelo, un canal del osciloscopio a la resistencia. Se utilizo una llave T para conectar en paralelo a la fuente el segundo canal del osciloscopio, logrando de esta manera, que se desplegaran en la pantalla las dos señales al mismo tiempo y fuera posibile medir la diferencia de fase entre las señales. Además, tambien se utilizó la frecuencia de esa segunda señal como \textit{trigger externo} para asegurar una imagen estatica en la pantalla del osciloscopio. Se fijo una amplitud $E_{0} = (8,00 \pm 0,08)V$ y se fue variando la frecuencia. Para cada frecuencia se tomó nota del desfasaje y de la amplitud calculados por el osciloscopio para la señal de salida. Una vez terminada la adquisicion de datos, se repitio el proceso con un cicuito con los mismos parametros, a excepcion de la resistencia, a la cual se le cambio el valor a $R = (500 \pm 5)$. Vale la pena aclarar que para este segundo circuito, la resistencia producida por la inductancia no era despreciable y fue tenida en cuenta.

Luego, se estudió el caso de la anti-resonancia, y para esto se diseñó un circuito RCL similar al anterior, con la salvedad que en este caso el capacitor se encontraba conectado en paralelo a la inductancia como ilustra la \textbf{Figura \ref{fig:RCL-ARes}}. De la misma forma que en el caso de resonancia, se realizaron dos juegos de mediciones y para ambas se utilizó una inductancia con un valor $L = (1 \pm0.002)H$ y una capacitacia $C = (9.95 \pm 0.07)nF$ mientras que la fuente se fijó en una amplitud $E_{0} = (8,00 \pm 0,08)V$ y la resistencia tuvo un valor $R = (5 \pm 0,05)K\Omega$ durante la primer medición, y en $R = (1 \pm 0,01)K\Omega$ durante la segunda.

\begin{figure}[h]
\centering
\includegraphics[scale=0.7]{Circuito-RCL-AntiResonante}
  \caption{Circuito RCL antiresonante con una fuente sinusoidal de amplitud (voltaje de entrada), capacitancia e inductancia fijas. Se obtuvo desfasaje y voltaje de salida en función de la frecuencia para dos resistencias distintas.}
  \label{fig:RCL-ARes}
\end{figure}

De manera analoga al metodo utilizado con el circuito resonante, el osciloscopio se conectó de forma paralela a la resistencia y a la fuente, y se tomaron nota de los desfasajes y amplitudes de la corriente de salida.


\subsection{Filtros}

El objetivo de esta parte fue experimentar con distintos tipos de filtros, para esto se contruyó un cirtuito que constaba de una resistencia $R = (5 \pm 0.05)k\Omega$, una capacitancia $C = (10.00 \pm 0.03)nF$ y un generador de funciones con una amplitud $E_{0} = (9.4 \pm 0.2)V$, colocados en serie. Para un primer analisis, se utilizaron como terminales de salida los extremos del capacitor obteniendo así, un filtro Pasa-Bajos, como se muestra en la \textbf{Figura \ref{fig:RC-PB}}. Cabe destacar, que de la misma manera que se hizo durante los experimentos de resonancia, se aseguro la continuidad de todos los cables a utilizar.

\begin{figure}[h]
\centering
\includegraphics[scale=0.9]{Circuito-RC-Pasa-Bajos}
  \caption{Circuito RC con un osciloscopio conectado en paralelo a la fuente y al capacitor para medir desfasaje y tensión sobre este último para distintas frecuencias. Tanto la resistencia como la capacitancia son fijos}
  \label{fig:RC-PB}
\end{figure}

Antes de realizar las mediciones se calculó la frecuencia de corte del filtro para asegurar que se relevaran datos que correspondieran tanto a las señales que eran atenuadas como a las que no. Una vez tomado esto en consideración, se utilizó el mismo procedimiento de medicion que se utilizo con el circuito resonante. 

Una vez finalizadas las mediciones sobre el dispositivo, se procedio a ver el caso del filtro Pasa-Altos; para lo cual se cambiaron las terminales de salida, colocandose en los extremos de la resistencia como ilustra la \textbf{Figura \ref{fig:RC-PA}}. No se realizaron variaciones en ninguno de los parametros, pero si se cambió la posicion de la descarga a tierra del generador de funciones para que coincidiera con la del osciloscopio.

\begin{figure}[h]
\centering
\includegraphics[scale=0.9]{Circuito-RC-Pasa-Altos}
  \caption{Circuito RC con un osciloscopio conectado en paralelo a la fuente y a la resistencia para medir desfasaje y tensión sobre esta última para distintas frecuencias. Tanto la resistencia como la capacitancia son fijos}
  \label{fig:RC-PA}
\end{figure}

El Procedimiento de medicion para este circuito, fue completamente análogo al realizado anteriormente con el circuito Pasa-Bajos. 

Finalmente se procedió a combinar ambos filtros para formar un Pasa-Banda como se puede ver en la \textbf{Figura \ref{fig:RC-PBD}}. En este caso si se cambiaron los parametros. Se fijarom ambas resistencias en un valor $R = (1 \pm 0.009)k\Omega$ y una capacitancia $C_{1} = (10.02 \pm 0.02)nF$ y una $C_{2} = (100.0 \pm 0.3)nF$. Cabe destacar que estos valores se eligieron de forma tal que la frecuencia de corte del pasabajos sea un orden de magnitud mayor que la del pasa altos y asi poder apreciar la campana de transferencia. 

\begin{figure}[h]
\centering
\includegraphics[scale=0.8]{Circuito-RC-Pasa-Banda}
  \caption{Circuito Pasa-Banda de amplitud fija cuyas resistencias tienen el mismo valor y la capacitancia $c_{1}$ es 10 veces menor que $c_{2}$. Para distintas frecuencias se midió el desfasaje y la tensión de salida con el osciloscopio.}
  \label{fig:RC-PBD}
\end{figure}

Una vez finalizada la construccion del dispositivo, se procedio a medir de la manera detallada anteriormente.


%%%%%%%%%%%%%%%%%%%%%%%%%%%%%%%%%%%%%%%%%%%%%%%%%%%%%%%%%%%%%%%%%%%%%%%%%%%%%%%%%%%%%%%%%%%%%%%%%%%%%%%%%%%%%%%%%%%%%%%%%%%%%%%%
% 4.DISCUSIÓN Y RESULTADOS: todo lo que se obtuvo y explicación. Graficos, tablas.
%%%%%%%%%%%%%%%%%%%%%%%%%%%%%%%%%%%%%%%%%%%%%%%%%%%%%%%%%%%%%%%%%%%%%%%%%%%%%%%%%%%%%%%%%%%%%%%%%%%%%%%%%%%%%%%%%%%%%%%%%%%%%%%%

\section{Resultados}


\subsection{Resonancia y Anti-resonancia}

Sabiendo que, sobre la resistencia, la diferencia de potencial efectiva es proporcional a la corriente efectiva; se midieron estas difencia de potencial sobre la resistencia $R$ de cada circuito para poder ajustarlo con la ecuación \textbf{\eqref{eq:1}} en los circuitos resonantes y \textbf{\eqref{eq:3}} en los circuitos anti-resonantes.  

Puesto que para el estudio de los circuitos resonante y anti-resonantes se utilizaron una capacitancia $C= (9.95 \pm 0.07)nF$ y una inductancia $L= (0.999 \pm 0.008) H$, la valor teórica de la frecuencia angular de resonancia dado por \textbf{\eqref{eq:F_res}} es $\omega_{0}=(10030 \pm 150) Hz$ para todos los circuitos.

En primer lugar se hizo el análisis de resonancia. Para ello se utilizo el sistema que tiene en serie la inductancia $L$, la capacitancia $C$ y la resistencia $R$; por lo que, al conciderarse despreciable la resistencia de $C$, la resistencia total del circuito era $R_{total} = R+R_{L}$ donde $R_{L} = (296 \pm 3)$ es la resistencia que impone la inductancia.

De las mediciones obtenidas para los circuitos con resistencia $R=(5.00 \pm 0.05)K\Omega$ y $R=(500.0 \pm 0.2)\Omega$ se pudieron esbozar los gráficos que muestran las \textbf{Figuras \ref{subfig:RES_I.a}} y \textbf{\ref{subfig:RES_I.b}} respectivamente.

\begin{figure}[h!]

\begin{subfigure}{0.5\textwidth}
\includegraphics[scale=0.3]{RLC_RES_5000_VvsF}
  \caption{Resonancia 5000 $\Omega$}
  \label{subfig:RES_I.a}
\end{subfigure}
\begin{subfigure}{0.5\textwidth}
\includegraphics[scale=0.32]{RLC_RES_500_VvsF}
  \caption{Resonancia 500 $\Omega$}
  \label{subfig:RES_I.b}
\end{subfigure}
  \caption{Circuitos RLC resonancia: $I_{ef}$}
  \label{fig:RES_I}
\end{figure}

Para el primer caso, para $R=(5.00 \pm 0.05)K\Omega$, se realizó un ajuste según \eqref{eq:1}. De manera que el coeficiente $R-Square = 0.99985$ garantizó la bondad de dicho ajuste, y utilizando los parámetros obtenidos se calcularon el máximo de potencial $V_{max}=(4.6 \pm 0.8)$ para una frecuencia $\omega_{res} = (10425 \pm 2500)Hz$, resultando entonces una frecuencia idéntica al valor teórico $\omega_0 = (10030 \pm 150)Hz$. A su vez se obtuvo un ancho de banda $\Delta\omega = (5590 \pm 1200) Hz$, y usando \textbf{\eqref{deltaomega}} se obtiene un valor teórico $\Delta\omega_{teo} = (5300 \pm 90)Hz$ teniendo entonces dos resultados indistinguibles al coincidir en un intervalo. 


Luego, para $R=(500 \pm 2)\Omega$, el coeficiente $R-Square = 0.99981$ garantizó la bondad del ajuste dado por \textbf{\eqref{eq:1}}. Por lo que se pudo calcular la frecuencia de resonancia $\omega_{res}= (10753 \pm 1000) Hz$ con un máximo de amplitud $V_{max}=(2.4 \pm 0.4) V$. Se puede ver entonces que la frecuencia de resonancia obtenida coincide con el resultado teórico $\omega_0 = (10030 \pm 150) Hz$. Ademas se calculó un ancho de banda $\Delta\omega = (697 \pm 200)Hz$, y utilizando \textbf{\eqref{deltaomega}} se obtuvo su valor teórico $\Delta\omega_{teo} = (729 \pm 6) Hz$ por lo cual, al coincidir en un intervalor, se concideran indistinguibles.

Por otro lado también se registraron los valores de la diferencia de fase entre la señal de entrada y la señal de salida, para obtener los gráficos que muestran las \textbf{Figuras \ref{subfig:RES_D.a}}, para $R=(5.00 \pm 0.05) K\Omega$, y \textbf{\ref{subfig:RES_D.b}} para $R=(500 \pm 2)\Omega$.

\begin{figure}[h!]
\begin{subfigure}{0.5\textwidth}
\includegraphics[scale=0.34]{RLC_RES_5000_Tan(fase)vsF}
  \caption{Resonancia 5000 $\Omega$}
  \label{subfig:RES_D.a}
\end{subfigure}
\begin{subfigure}{0.5\textwidth}
\includegraphics[scale=0.34]{RLC_RES_500_Tan(fase)vsF}
  \caption{Resonancia 500 $\Omega$}
  \label{subfig:RES_D.b}
\end{subfigure}
  \caption{Circuitos RLC resonancia: \textit{Desfasaje}}
  \label{fig:RES_D}
\end{figure}

Por lo que, al obtener un $R-Square = 0.99885$ se garantizó la bondad del ajuste de la \textbf{Figura \ref{subfig:RES_D.a}} dado por \textbf{\eqref{eq:2}}. Entonces, despejando de los parametros, se encontró que las dos señales están en fase cuando $\omega = ( 9978 \pm 90) Hz$ coincidiendo con la frecuencia de resonancia $\omega_{res} = (10030 \pm 150) Hz$.

De la misma manera, el ajuste de \textbf{\ref{subfig:RES_D.b}} por la ecuación \eqref{eq:2} dio un $R-Square = 0.99091$, nuevamente garantizando la bondad del ajuste. Y se obtuvo entonces que las dos señales estaban en fase cuando $\omega = (10036 \omega 220) Hz$ siendo indistinguible de la frecuencia de resonancia $\omega_{res} = (10030 \pm 150) Hz$.

\bigskip

De manera análoga se hizo el análisis de anti-resonancia. En este caso se consideraron despreciables las resistencia impuestas por $C$ como por $L$.

A partir de las mediciones obtenidas para los circuitos con resistencia $R=(5.00 \pm 0.05) K\Omega$ y $R=(1.00 \pm 0.01)K\Omega$ se realizaron los gráficos que muestran las \textbf{Figuras \ref{subfig:ARES_I.a}} y \textbf{\ref{subfig:ARES_I.b}} respectivamente.

\begin{figure}[h!]

\begin{subfigure}{0.5\textwidth}
\includegraphics[scale=0.31]{RLC_ARES_5000_VvsF}
  \caption{Anti-resonancia 5000 $\Omega$}
  \label{subfig:ARES_I.a}
\end{subfigure}
\begin{subfigure}{0.5\textwidth}
\includegraphics[scale=0.3]{RLC_ARES_1000_VvsF}
  \caption{Anti-resonancia 1000 $\Omega$}
  \label{subfig:ARES_I.b}
\end{subfigure}
  \caption{Circuitos RLC anti-resonancia: $I_{ef}$}
  \label{fig:ARES_I}
\end{figure}


También se registraron la diferencia de fase entre las señales de entrada y salida y se obtuvieron los graficos que muestras las \textbf{Figuras \ref{subfig:ARES_D.a}}, para $R=(5.00 \pm 0.05)K\Omega$, y \textbf{\ref{subfig:ARES_D.b}} para $R=(1.00 \pm 0.01)K\Omega$.

\begin{figure}[h!]

\begin{subfigure}{0.5\textwidth}
\includegraphics[scale=0.34]{RLC_ARES_5000_Tan(fase)vsF}
  \caption{Anti-resonancia 5000 $\Omega$}
  \label{subfig:ARES_D.a}
\end{subfigure}
\begin{subfigure}{0.5\textwidth}
\includegraphics[scale=0.34]{RLC_ARES_1000_Tan(fase)vsF}
  \caption{Anti-resonancia 1000 $\Omega$}
  \label{subfig:ARES_D.b}
\end{subfigure}
  \caption{Circuitos RLC anti-resonancia: \textit{Desfasaje}}
  \label{fig:ARES_D}
\end{figure}


\label{sec:discusion}



%%%%%%%%%%%%%%%%%%%%%%%%%%%%%%%%%%%%%%%%%%%%%%%%%%%%%%%%%%%%%%%%%%%%%%%%%%%%%%%%%%%%%%%%%%%%%%%%%%%%%%%%%%%%%%%%%%%%%%%%%%%%%%%%
%	CONCLUSIONES
%%%%%%%%%%%%%%%%%%%%%%%%%%%%%%%%%%%%%%%%%%%%%%%%%%%%%%%%%%%%%%%%%%%%%%%%%%%%%%%%%%%%%%%%%%%%%%%%%%%%%%%%%%%%%%%%%%%%%%%%%%%%%%%%

\section{Conclusiones}
\label{sec:conclusiones}




%%%%%%%%%%%%%%%%%%%%%%%%%%%%%%%%%%%%%%%%%%%%%%%%%%%%%%%%%%%%%%%%%%%%%%%%%%%%%%%%%%%%%%%%%%%%%%%%%%%%%%%%%%%%%%%%%%%%%%%%%%%%%%%%%
%	APÉNDICE: esas cosas extras que simplemente no tuvieron lo suficiente como para ganarse una sección propia.
%%%%%%%%%%%%%%%%%%%%%%%%%%%%%%%%%%%%%%%%%%%%%%%%%%%%%%%%%%%%%%%%%%%%%%%%%%%%%%%%%%%%%%%%%%%%%%%%%%%%%%%%%%%%%%%%%%%%%%%%%%%%%%%%%



%%%%%%%%%%%%%%%%%%%%%%%%%%%%%%%%%%%%%%%%%%%%%%%%%%%%%%%%%%%%%%%%%%%%%%%%%%%%%%%%%%%%%%%%%%%%%%%%%%%%%%%%%%%%%%%%%%%%%%%%%%%%%%%%%
%	REFERENCIAS: libros, libros, libros.
%%%%%%%%%%%%%%%%%%%%%%%%%%%%%%%%%%%%%%%%%%%%%%%%%%%%%%%%%%%%%%%%%%%%%%%%%%%%%%%%%%%%%%%%%%%%%%%%%%%%%%%%%%%%%%%%%%%%%%%%%%%%%%%%%

%Ejemplo:
\begin{thebibliography}{1}
 \bibitem{Berkeley} Frank S. Crawford, \textit{Berkeley physics course 3: Ondas}, 1994, Editorial Reverte S.A.
\end{thebibliography}
%Para citar: blablabla \cite{Baird}
 
\end{document}





