\documentclass[11pt,a4paper]{article}

\usepackage[utf8]{inputenc}
\usepackage[spanish]{babel}
\usepackage{amsmath}
\usepackage{float}
\usepackage{amsfonts}
\usepackage{amssymb}
\usepackage{makeidx}
\usepackage{graphicx}
\usepackage{lmodern}
\usepackage{kpfonts}
\usepackage{wrapfig}
\usepackage{caption}
\usepackage{subcaption}
\usepackage{booktabs}
\usepackage[nottoc,numbib]{tocbibind} %agrega la bibliografia al índice.
\usepackage[font={small,it}]{caption}
%\usepackage{fourier}
\usepackage[left=2cm,right=2cm,top=2cm,bottom=2cm,headheight=13.6pt]{geometry}
\usepackage{fancyhdr}
\usepackage{multirow}
\pagestyle{fancy}


%Para los gráficos en general, con las tablas...¡Ja!, arreglate.
%\begin{figure}[h!]
%\centering
%\includegraphics[width=0.7\textwidth]{} %nombre de la imagen, incluirla en el mismo directorio que este archivo.
%\caption*{} %rótulo, el asterico elimina la numeración automática. 
%\label{fig:} % para luego referirse con \ref{fig:}
%\end{figure}


\begin{document}


%%%%%%%%%%%%%%%%%%%%%%%%%%%%%%%%%%%%%%%%%%%%%%%%%%%%%%%%%%%%%%%%%%%%%%%%%%%%%%%%%%%%%%%%%%%%%%%%%%%%%%%%%%%%%%%%%%%%%%%%%%%%%%%%%
% 	TÍTULO
%%%%%%%%%%%%%%%%%%%%%%%%%%%%%%%%%%%%%%%%%%%%%%%%%%%%%%%%%%%%%%%%%%%%%%%%%%%%%%%%%%%%%%%%%%%%%%%%%%%%%%%%%%%%%%%%%%%%%%%%%%%%%%%%%

%%%%%%%%%%%%%%%%%%%%%%%%%%%%%%%%%%%%%%%%%
% University Assignment Title Page 
% LaTeX Template
% Version 1.0 (27/12/12)
%
% This template has been downloaded from:
% http://www.LaTeXTemplates.com
%
% Original author:
% WikiBooks (http://en.wikibooks.org/wiki/LaTeX/Title_Creation)
%
% License:
% CC BY-NC-SA 3.0 (http://creativecommons.org/licenses/by-nc-sa/3.0/)
% 
% Instructions for using this template:
% This title page is capable of being compiled as is. This is not useful for 
% including it in another document. To do this, you have two options: 
%
% 1) Copy/paste everything between \begin{document} and \end{document} 
% starting at \begin{titlepage} and paste this into another LaTeX file where you 
% want your title page.
% OR
% 2) Remove everything outside the \begin{titlepage} and \end{titlepage} and 
% move this file to the same directory as the LaTeX file you wish to add it to. 
% Then add \input{./title_page_1.tex} to your LaTeX file where you want your
% title page.
%
%%%%%%%%%%%%%%%%%%%%%%%%%%%%%%%%%%%%%%%%%

%----------------------------------------------------------------------------------------
%	PACKAGES AND OTHER DOCUMENT CONFIGURATIONS
%----------------------------------------------------------------------------------------

%\documentclass[12pt]{article}
%\usepackage[utf8]{inputenc}
%\usepackage[spanish]{babel}
%\begin{document}

\begin{titlepage}

\newcommand{\HRule}{\rule{\linewidth}{0.5mm}} % Defines a new command for the horizontal lines, change thickness here

\center % Center everything on the page
 
%----------------------------------------------------------------------------------------
%	HEADING SECTIONS
%----------------------------------------------------------------------------------------

\textsc{\Huge Universidad de Buenos Aires}\\[0.5cm]
\textsc{\LARGE Facultad de Ciencias Exactas y Naturales}\\[0.5cm] % Name of your university/college
\textsc{\Large Departamento de Física}\\[0.25cm] % Major heading such as course name

\begin{figure}[h]
  \centering
  \includegraphics[scale=0.15]{Logo_DF}
  \\[0.5cm]
\end{figure}

\textsc{\large Laboratorio 3}\\[0.25cm] % Minor heading such as course title

%----------------------------------------------------------------------------------------
%	TITLE SECTION
%----------------------------------------------------------------------------------------

\HRule \\[0.4cm]
{ \huge \bfseries Estudio de resonancia y antiresonancia en circuitos RLC, y filtros en circuitos RC utilizando señales sinusoidales }\\[0.2cm] % Title of your document
\HRule \\[1cm]
 
%----------------------------------------------------------------------------------------
%	AUTHOR SECTION
%----------------------------------------------------------------------------------------

\begin{minipage}{0.4\textwidth}
\begin{center} \large
\emph{Autores:}\\
\textsc{Andreu}, Gonzalo\\ % Your name
\textsc{Malpartida}, Bryan\\ % Your name
\textsc{Pugliese}, Facundo\\ % Your name


\end{center}
\end{minipage}
~ \\[1.25cm]
%\begin{minipage}{0.4\textwidth}
%\begin{flushright} \large
%\emph{Supervisor:} \\
%Dr. James \textsc{Smith} % Supervisor's Name
%\end{flushright}
%\end{minipage}\\[4cm]

% If you don't want a supervisor, uncomment the two lines below and remove the section above
%\Large \emph{Author:}\\
%John \textsc{Smith}\\[3cm] % Your name

%----------------------------------------------------------------------------------------
%	DATE SECTION
%----------------------------------------------------------------------------------------

%\vspace{\fill}


{\large 24 de Febrero de 2016}\\[1.75cm] % Date, change the \today to a set date if you want to be precise

%----------------------------------------------------------------------------------------
%	SUMMARY SECTION: No más de 15 renglones, no te zarpes
%----------------------------------------------------------------------------------------

\begin{center}
\large{\textbf{Resumen}}

\small{El objetivo del siguiente trabajo fue comprobar empíricamente los fenómenos de resonancia y anti-resonancia presente en circuitos eléctricos RLC, así como también el estudio de filtros pasa-bajos y filtros pasa-alto en circuitos RC y filtros pasa-banda en un circuito compuesto por dos RC conectados en paralelo. Para ello se observó, utilizando un osciloscopio digital, el comportamiento de la transferencia y desfasaje entre la señal de salida y una señal de entrada sinusoidal entregada por un fuente de alimentación programable.} % ACA VA EL RESUMEN


\end{center}


%----------------------------------------------------------------------------------------
%	LOGO SECTION
%----------------------------------------------------------------------------------------

%\includegraphics{Logo}\\[1cm] % Include a department/university logo - this will require the graphicx package
 
%----------------------------------------------------------------------------------------

\vfill % Fill the rest of the page with whitespace

\end{titlepage}
%\end{document} %incluir en el mismo directorio que este archivo. Equivalente a un copiar-pegar, nada de andar diciendo \begin{document} en la portada. Dejar el nombre de Caratula a la caratula.

%%%%%%%%%%%%%%%%%%%%%%%%%%%%%%%%%%%%%%%%%%%%%%%%%%%%%%%%%%%%%%%%%%%%%%%%%%%%%%%%%%%%%%%%%%%%%%%%%%%%%%%%%%%%%%%%%%%%%%%%%%%%%%%%%
% 	ENCABEZADO Y PIE DE PÁGINA.
%%%%%%%%%%%%%%%%%%%%%%%%%%%%%%%%%%%%%%%%%%%%%%%%%%%%%%%%%%%%%%%%%%%%%%%%%%%%%%%%%%%%%%%%%%%%%%%%%%%%%%%%%%%%%%%%%%%%%%%%%%%%%%%%%

\lhead{}
\chead{}
\rhead{Laboratorio 3}
\lfoot{}
\cfoot{}
\rfoot{\thepage}
\renewcommand{\headrulewidth}{1pt}
\renewcommand{\footrulewidth}{1pt}


%%%%%%%%%%%%%%%%%%%%%%%%%%%%%%%%%%%%%%%%%%%%%%%%%%%%%%%%%%%%%%%%%%%%%%%%%%%%%%%%%%%%%%%%%%%%%%%%%%%%%%%%%%%%%%%%%%%%%%%%%%%%%%%
% Página en blanco. Cita, agradecimiento, dedicación, lo que sea pero que sea algo.
%%%%%%%%%%%%%%%%%%%%%%%%%%%%%%%%%%%%%%%%%%%%%%%%%%%%%%%%%%%%%%%%%%%%%%%%%%%%%%%%%%%%%%%%%%%%%%%%%%%%%%%%%%%%%%%%%%%%%%%%%%%%%%%


%%%%%%%%%%%%%%%%%%%%%%%%%%%%%%%%%%%%%%%%%%%%%%%%%%%%%%%%%%%%%%%%%%%%%%%%%%%%%%%%%%%%%%%%%%%%%%%%%%%%%%%%%%%%%%%%%%%%%%%%%%%%%%%%%
% 	ÍNDICE
%%%%%%%%%%%%%%%%%%%%%%%%%%%%%%%%%%%%%%%%%%%%%%%%%%%%%%%%%%%%%%%%%%%%%%%%%%%%%%%%%%%%%%%%%%%%%%%%%%%%%%%%%%%%%%%%%%%%%%%%%%%%%%%%%

%\tableofcontents %compilar dos o tres veces para verlo bien. ¡Todo un índice en unas cuantas letras!
%\newpage

%%%%%%%%%%%%%%%%%%%%%%%%%%%%%%%%%%%%%%%%%%%%%%%%%%%%%%%%%%%%%%%%%%%%%%%%%%%%%%%%%%%%%%%%%%%%%%%%%%%%%%%%%%%%%%%%%%%%%%%%%%%%%%%
% 1. RESUMEN
%%%%%%%%%%%%%%%%%%%%%%%%%%%%%%%%%%%%%%%%%%%%%%%%%%%%%%%%%%%%%%%%%%%%%%%%%%%%%%%%%%%%%%%%%%%%%%%%%%%%%%%%%%%%%%%%%%%%%%%%%%%%%%%

%\section{Resumen}
%\label{sec:resumen}



%%%%%%%%%%%%%%%%%%%%%%%%%%%%%%%%%%%%%%%%%%%%%%%%%%%%%%%%%%%%%%%%%%%%%%%%%%%%%%%%%%%%%%%%%%%%%%%%%%%%%%%%%%%%%%%%%%%%%%%%%%%%%%%
% 2. INTRODUCCIÓN: ecuaciones aquí, luego se las cita.
%%%%%%%%%%%%%%%%%%%%%%%%%%%%%%%%%%%%%%%%%%%%%%%%%%%%%%%%%%%%%%%%%%%%%%%%%%%%%%%%%%%%%%%%%%%%%%%%%%%%%%%%%%%%%%%%%%%%%%%%%%%%%%%

\section{Introducción}\label{sec:intro}

\subsection{Campos magnéticos estacionarios}

El primer campo estacionario que se estudiará será el campo magnético terrestre. Debido al núcleo de hierro en el interior del planeta se genera un campo magnético $\vec{B}$ como se ve en la \textbf{Figura \ref{subfig:Campo_B}}.

\begin{figure}[h]

\begin{subfigure}{0.5\textwidth}
\centering
\includegraphics[scale=.8]{Campo.png}
	\subcaption{Campo magnético terrestre}
	\label{subfig:Campo_B}
\end{subfigure}
\begin{subfigure}{0.5\textwidth}
\centering
\includegraphics[scale=.65]{angulos_campo.jpg}
	\subcaption{Se definen $i$ es la inclinación y $d$ la declinación del campo magnético terrestre.}
	\label{subfig:Angulos}
\end{subfigure}
\caption{Relación angular entra el campo magnético y el eje de rotación.}
\end{figure}

Siendo que el eje de rotación y del campo magnético están inclinados uno respecto del otro, se pueden definir la \textit{inclinación} $i$ como el ángulo que forma $\vec{B}$ y el vector $\vec{z}$ que va al centro del planeta; y la \textit{declinación} $d$ como el ángulo entre $\vec{B}$ y el vector del norte geográfico $\vec{N_g}$ (ver \textbf{ Figura \ref{subfig:Angulos}}). Es claro que estos ángulos depende de la posición geográfica de donde se lo mida.

\bigskip
El siguiente campo estacionario es el generado por una bobina. Si se considera una bobina de largo $L$, radio $r$ y con $N$ cantidad de vueltas, como se ve en la \textbf{Figura \ref{fig:bobina_intro}},

\begin{figure}[h]

\centering
\includegraphics[scale=1]{bobina.png}
	\caption{Bobina de largo $l$, radio $r$ y con $N$ cantidad de vueltas por el cual circula una corriente $I$.}
	\label{fig:bobina_intro}
\end{figure}

al hacer circular una corriente $I$ se genere un campo magnético. Tomando el origen $0$ en el punto medio de la bobina que pasa por el eje de simetría $z$, de modo que ambos extremos se encuentre a distancia $\frac{l}{2}$ de dicho origen; el valor del campo sobre los puntos $z=0$ y $z=\pm \frac{l}{2}$ serán:

\begin{equation}{\label{valores_B}}
\begin{split}
B(0,0, \pm \frac{l}{2})= \frac{\mu I N}{2\sqrt{R^2+l^2}}\\
B(0,0,0)= \frac{\mu I N}{2\sqrt{R^2+(\frac{l}{2})^2}}
\end{split}
\end{equation}

Donde $\mu$ es la permeabilidad magnética del medio. Ademas, la inductancia $L$ de la bobina es

\begin{equation}
L=\frac{2\pi\mu N^2 r}{l}
\end{equation}

\subsection{Transformadores}
Un transformador es un sistema en el cual un circuito activo o \textit{primario}, con inductancia $L_1$, transmite energía a un circuito pasivo o \textit{secundario} con inductancia $L_2$. Este fenómeno se debe a la inductancia mutua $M$ entre ambos circuitos armando el sistema como se ve en la \textbf{Figura \ref{fig:transformador}}.

\begin{figure}[h]
\centering
\includegraphics[scale=.8]{transformador.jpg}
\caption{Transformador}
\label{fig:transformador}	
\end{figure}

Si la fuente $\varepsilon$ varia en el tiempo de la forma $\varepsilon (t) = E_0 sin (\omega t)$, al resolver el circuito utilizando las leyes de Kirchoff, la transferencia tiene una expresión compleja:

\begin{equation}
\frac{V_c}{V_e}= \frac{i\omega M R_c}{(R_1 + i\omega L_1)(R_c + i\omega L_2)+ (\omega M)^2}
\label{transf_1}
\end{equation}

Bajo la condición $R_1 \ll \omega L_1$, la solución anterior se puede simplificar en:

\begin{equation}
\frac{V_c}{V_e}=\frac{MR_c}{L_1R_c-i\omega L_1L_2(1-k^2)}
\label{transf_2}
\end{equation}

En este caso la inductancia mutua se puede expresar como $M^2=K^2L_1L_2$ siendo $\vert k\vert\leq1$. Solo en el caso ideal $\vert k\vert = 1$, por lo que si entre las bobinas se coloca un núcleo de un material ferromagnético, por ejemplo hierro, se tendrá $\vert k\vert \simeq 1$; de modo que, en la aproximación al caso ideal, la transferencia del transformador es:

\begin{equation}
\frac{V_c}{V_e}= \sqrt{\frac{L_1}{L_2}}
\label{trans_3}
\end{equation}



%%%%%%%%%%%%%%%%%%%%%%%%%%%%%%%%%%%%%%%%%%%%%%%%%%%%%%%%%%%%%%%%%%%%%%%%%%%%%%%%%%%%%%%%%%%%%%%%%%%%%%%%%%%%%%%%%%%%%%%%%%%%%%%
% 3. DISPOSITIVO EXPERIMENTAL: armado del modelo, como se midio, consideraciones a la hora de medir.
%%%%%%%%%%%%%%%%%%%%%%%%%%%%%%%%%%%%%%%%%%%%%%%%%%%%%%%%%%%%%%%%%%%%%%%%%%%%%%%%%%%%%%%%%%%%%%%%%%%%%%%%%%%%%%%%%%%%%%%%%%%%%%%

\section{Desarrollo experimental}

Durante esta experiencia se utilizó un generador de funciones de generar diferencias de potencial $V(t)$ a frecuencias con un error relativo del $0,01\%$ en un rango entre $1\mu Hz$ y $5MHz$. El voltaje pico-pico tiene un error relativo del $1\%$ para el rango de voltaje utilizado ($2V-20V$). Además, se utilizó una resistencia variable por décadas en el intervalo $1\Omega-11110\Omega$. Usando un multimetro digital se midieron los valores utilizados en esta resistencia junto con su precisión que era de la forma $\pm(1\%+2d)$ en el rango utilizado (mayor a $100\Omega$). El multimetro permitía también medir diferencias de potencial con una precisión de la forma $(1d+0,5\%)$

Para generar una diferencia de potencial constante se utilizó una fuente de alimentación programable con una precisión de $\pm (0.5\% + 0.02)V$ en un rango de voltaje de salida de $0 \thicksim 32 V$. Se utilizó además un osciloscopio digital que en sus dos canales de entrada era capaz de medir diferencias de potencial entre sus dos terminales en un rango de 2mV a 5V con un error relativo del $3\%$. A la hora de medir voltaje, fue necesario asegurarse que el cable a tierra del osciloscopio estuvera conectado al cable a tierra el generador de funciones. 

Finalmente, se utilizó un sensor Hall capaz de medir el módulo de campos magnéticos $B$ en una dirección particular, expresándolos como una diferencia de potencial $V$. La relación entre $B$ y $V$ está dada por $B = mV+offset$ donde $m$ es una característica del instrumento definida por su ganancia $G$ y su sensibilidad $S$ según $m = S^{-1}$. La ganancia $G$ puede ser x1, x10 o x200 y define una sensibilidad $S_{x1} = (3,06 \pm 0,15)\frac{mV}{G}$,  $S_{x10} = (30,6 \pm 1,2)\frac{mV}{G}$ y $S_{x200} = (612 \pm 24)\frac{mV}{G}$. Por lo tanto, las pendientes resultan $m_{x1} = (327 \pm 5)\frac{G}{V}$, $m_{x10} = (32,7 \pm 1,3)\frac{G}{V}$ y $m_{x200} = (1,63 \pm 0,06)\frac{G}{V}$. La caracterización del $offset$ se cubre a continuación.

\subsection{Caracterización del Sensor Hall}



%%%%%%%%%%%%%%%%%%%%%%%%%%%%%%%%%%%%%%%%%%%%%%%%%%%%%%%%%%%%%%%%%%%%%%%%%%%%%%%%%%%%%%%%%%%%%%%%%%%%%%%%%%%%%%%%%%%%%%%%%%%%%%%%
% 4.DISCUSIÓN Y RESULTADOS: todo lo que se obtuvo y explicación. Graficos, tablas.
%%%%%%%%%%%%%%%%%%%%%%%%%%%%%%%%%%%%%%%%%%%%%%%%%%%%%%%%%%%%%%%%%%%%%%%%%%%%%%%%%%%%%%%%%%%%%%%%%%%%%%%%%%%%%%%%%%%%%%%%%%%%%%%%

\section{Resultados}
\label{sec:discusion}





%%%%%%%%%%%%%%%%%%%%%%%%%%%%%%%%%%%%%%%%%%%%%%%%%%%%%%%%%%%%%%%%%%%%%%%%%%%%%%%%%%%%%%%%%%%%%%%%%%%%%%%%%%%%%%%%%%%%%%%%%%%%%%%%
%	CONCLUSIONES
%%%%%%%%%%%%%%%%%%%%%%%%%%%%%%%%%%%%%%%%%%%%%%%%%%%%%%%%%%%%%%%%%%%%%%%%%%%%%%%%%%%%%%%%%%%%%%%%%%%%%%%%%%%%%%%%%%%%%%%%%%%%%%%%

\section{Conclusiones}
\label{sec:conclusiones}






%%%%%%%%%%%%%%%%%%%%%%%%%%%%%%%%%%%%%%%%%%%%%%%%%%%%%%%%%%%%%%%%%%%%%%%%%%%%%%%%%%%%%%%%%%%%%%%%%%%%%%%%%%%%%%%%%%%%%%%%%%%%%%%%%
%	APÉNDICE: esas cosas extras que simplemente no tuvieron lo suficiente como para ganarse una sección propia.
%%%%%%%%%%%%%%%%%%%%%%%%%%%%%%%%%%%%%%%%%%%%%%%%%%%%%%%%%%%%%%%%%%%%%%%%%%%%%%%%%%%%%%%%%%%%%%%%%%%%%%%%%%%%%%%%%%%%%%%%%%%%%%%%%



%%%%%%%%%%%%%%%%%%%%%%%%%%%%%%%%%%%%%%%%%%%%%%%%%%%%%%%%%%%%%%%%%%%%%%%%%%%%%%%%%%%%%%%%%%%%%%%%%%%%%%%%%%%%%%%%%%%%%%%%%%%%%%%%%
%	REFERENCIAS: libros, libros, libros.
%%%%%%%%%%%%%%%%%%%%%%%%%%%%%%%%%%%%%%%%%%%%%%%%%%%%%%%%%%%%%%%%%%%%%%%%%%%%%%%%%%%%%%%%%%%%%%%%%%%%%%%%%%%%%%%%%%%%%%%%%%%%%%%%%

%Ejemplo:
\begin{thebibliography}{1}
 \bibitem{Berkeley} Frank S. Crawford, \textit{Berkeley physics course 3: Ondas}, 1994, Editorial Reverte S.A.
\end{thebibliography}
%Para citar: blablabla \cite{Baird}
 
\end{document}





